% Options for packages loaded elsewhere
\PassOptionsToPackage{unicode}{hyperref}
\PassOptionsToPackage{hyphens}{url}
%
\documentclass[
  12pt,
]{article}
\usepackage{lmodern}
\usepackage{amssymb,amsmath}
\usepackage{ifxetex,ifluatex}
\ifnum 0\ifxetex 1\fi\ifluatex 1\fi=0 % if pdftex
  \usepackage[T1]{fontenc}
  \usepackage[utf8]{inputenc}
  \usepackage{textcomp} % provide euro and other symbols
\else % if luatex or xetex
  \usepackage{unicode-math}
  \defaultfontfeatures{Scale=MatchLowercase}
  \defaultfontfeatures[\rmfamily]{Ligatures=TeX,Scale=1}
\fi
% Use upquote if available, for straight quotes in verbatim environments
\IfFileExists{upquote.sty}{\usepackage{upquote}}{}
\IfFileExists{microtype.sty}{% use microtype if available
  \usepackage[]{microtype}
  \UseMicrotypeSet[protrusion]{basicmath} % disable protrusion for tt fonts
}{}
\makeatletter
\@ifundefined{KOMAClassName}{% if non-KOMA class
  \IfFileExists{parskip.sty}{%
    \usepackage{parskip}
  }{% else
    \setlength{\parindent}{0pt}
    \setlength{\parskip}{6pt plus 2pt minus 1pt}}
}{% if KOMA class
  \KOMAoptions{parskip=half}}
\makeatother
\usepackage{xcolor}
\IfFileExists{xurl.sty}{\usepackage{xurl}}{} % add URL line breaks if available
\IfFileExists{bookmark.sty}{\usepackage{bookmark}}{\usepackage{hyperref}}
\hypersetup{
  pdftitle={Econometrics II TA Session \#8},
  pdfauthor={Hiroki Kato},
  hidelinks,
  pdfcreator={LaTeX via pandoc}}
\urlstyle{same} % disable monospaced font for URLs
\usepackage[margin=1in]{geometry}
\usepackage{color}
\usepackage{fancyvrb}
\newcommand{\VerbBar}{|}
\newcommand{\VERB}{\Verb[commandchars=\\\{\}]}
\DefineVerbatimEnvironment{Highlighting}{Verbatim}{commandchars=\\\{\}}
% Add ',fontsize=\small' for more characters per line
\usepackage{framed}
\definecolor{shadecolor}{RGB}{248,248,248}
\newenvironment{Shaded}{\begin{snugshade}}{\end{snugshade}}
\newcommand{\AlertTok}[1]{\textcolor[rgb]{0.94,0.16,0.16}{#1}}
\newcommand{\AnnotationTok}[1]{\textcolor[rgb]{0.56,0.35,0.01}{\textbf{\textit{#1}}}}
\newcommand{\AttributeTok}[1]{\textcolor[rgb]{0.77,0.63,0.00}{#1}}
\newcommand{\BaseNTok}[1]{\textcolor[rgb]{0.00,0.00,0.81}{#1}}
\newcommand{\BuiltInTok}[1]{#1}
\newcommand{\CharTok}[1]{\textcolor[rgb]{0.31,0.60,0.02}{#1}}
\newcommand{\CommentTok}[1]{\textcolor[rgb]{0.56,0.35,0.01}{\textit{#1}}}
\newcommand{\CommentVarTok}[1]{\textcolor[rgb]{0.56,0.35,0.01}{\textbf{\textit{#1}}}}
\newcommand{\ConstantTok}[1]{\textcolor[rgb]{0.00,0.00,0.00}{#1}}
\newcommand{\ControlFlowTok}[1]{\textcolor[rgb]{0.13,0.29,0.53}{\textbf{#1}}}
\newcommand{\DataTypeTok}[1]{\textcolor[rgb]{0.13,0.29,0.53}{#1}}
\newcommand{\DecValTok}[1]{\textcolor[rgb]{0.00,0.00,0.81}{#1}}
\newcommand{\DocumentationTok}[1]{\textcolor[rgb]{0.56,0.35,0.01}{\textbf{\textit{#1}}}}
\newcommand{\ErrorTok}[1]{\textcolor[rgb]{0.64,0.00,0.00}{\textbf{#1}}}
\newcommand{\ExtensionTok}[1]{#1}
\newcommand{\FloatTok}[1]{\textcolor[rgb]{0.00,0.00,0.81}{#1}}
\newcommand{\FunctionTok}[1]{\textcolor[rgb]{0.00,0.00,0.00}{#1}}
\newcommand{\ImportTok}[1]{#1}
\newcommand{\InformationTok}[1]{\textcolor[rgb]{0.56,0.35,0.01}{\textbf{\textit{#1}}}}
\newcommand{\KeywordTok}[1]{\textcolor[rgb]{0.13,0.29,0.53}{\textbf{#1}}}
\newcommand{\NormalTok}[1]{#1}
\newcommand{\OperatorTok}[1]{\textcolor[rgb]{0.81,0.36,0.00}{\textbf{#1}}}
\newcommand{\OtherTok}[1]{\textcolor[rgb]{0.56,0.35,0.01}{#1}}
\newcommand{\PreprocessorTok}[1]{\textcolor[rgb]{0.56,0.35,0.01}{\textit{#1}}}
\newcommand{\RegionMarkerTok}[1]{#1}
\newcommand{\SpecialCharTok}[1]{\textcolor[rgb]{0.00,0.00,0.00}{#1}}
\newcommand{\SpecialStringTok}[1]{\textcolor[rgb]{0.31,0.60,0.02}{#1}}
\newcommand{\StringTok}[1]{\textcolor[rgb]{0.31,0.60,0.02}{#1}}
\newcommand{\VariableTok}[1]{\textcolor[rgb]{0.00,0.00,0.00}{#1}}
\newcommand{\VerbatimStringTok}[1]{\textcolor[rgb]{0.31,0.60,0.02}{#1}}
\newcommand{\WarningTok}[1]{\textcolor[rgb]{0.56,0.35,0.01}{\textbf{\textit{#1}}}}
\usepackage{graphicx}
\makeatletter
\def\maxwidth{\ifdim\Gin@nat@width>\linewidth\linewidth\else\Gin@nat@width\fi}
\def\maxheight{\ifdim\Gin@nat@height>\textheight\textheight\else\Gin@nat@height\fi}
\makeatother
% Scale images if necessary, so that they will not overflow the page
% margins by default, and it is still possible to overwrite the defaults
% using explicit options in \includegraphics[width, height, ...]{}
\setkeys{Gin}{width=\maxwidth,height=\maxheight,keepaspectratio}
% Set default figure placement to htbp
\makeatletter
\def\fps@figure{htbp}
\makeatother
\setlength{\emergencystretch}{3em} % prevent overfull lines
\providecommand{\tightlist}{%
  \setlength{\itemsep}{0pt}\setlength{\parskip}{0pt}}
\setcounter{secnumdepth}{5}
\usepackage{zxjatype}
\setCJKmainfont[BoldFont = IPAゴシック]{IPA明朝}
\setCJKsansfont{IPAゴシック}
\setCJKmonofont{IPAゴシック}
\parindent = 1em
\newcommand{\argmax}{\mathop{\rm arg~max}\limits}
\newcommand{\argmin}{\mathop{\rm arg~min}\limits}
\DeclareMathOperator*{\plim}{plim}
\usepackage{xcolor}
\ifluatex
  \usepackage{selnolig}  % disable illegal ligatures
\fi

\title{Econometrics II TA Session \#8}
\author{Hiroki Kato}
\date{}

\begin{document}
\maketitle

\hypertarget{empirical-application-of-panel-data-model-earnings-equation}{%
\section{Empirical Application of Panel Data Model: Earnings
Equation}\label{empirical-application-of-panel-data-model-earnings-equation}}

\hypertarget{backgruond}{%
\subsection{Backgruond}\label{backgruond}}

A researcher wants to estimate the effect of full-time work experience
on wages. He uses a \emph{balanced} panel of 595 individuals from 1976
to 1982, taken from the Panel Study of Income Dynamics (PSID). The
\emph{balanced} panel data means that we can observe all individuals
every year.

\begin{Shaded}
\begin{Highlighting}[]
\NormalTok{dt \textless{}{-}}\StringTok{ }\KeywordTok{read.csv}\NormalTok{(}\StringTok{"./data/wages.csv"}\NormalTok{)}
\KeywordTok{head}\NormalTok{(dt, }\DecValTok{14}\NormalTok{)}
\end{Highlighting}
\end{Shaded}

\begin{verbatim}
##    exp wks bluecol ind south smsa married  sex union ed black   lwage id time
## 1    3  32      no   0   yes   no     yes male    no  9    no 5.56068  1    1
## 2    4  43      no   0   yes   no     yes male    no  9    no 5.72031  1    2
## 3    5  40      no   0   yes   no     yes male    no  9    no 5.99645  1    3
## 4    6  39      no   0   yes   no     yes male    no  9    no 5.99645  1    4
## 5    7  42      no   1   yes   no     yes male    no  9    no 6.06146  1    5
## 6    8  35      no   1   yes   no     yes male    no  9    no 6.17379  1    6
## 7    9  32      no   1   yes   no     yes male    no  9    no 6.24417  1    7
## 8   30  34     yes   0    no   no     yes male    no 11    no 6.16331  2    1
## 9   31  27     yes   0    no   no     yes male    no 11    no 6.21461  2    2
## 10  32  33     yes   1    no   no     yes male   yes 11    no 6.26340  2    3
## 11  33  30     yes   1    no   no     yes male    no 11    no 6.54391  2    4
## 12  34  30     yes   1    no   no     yes male    no 11    no 6.69703  2    5
## 13  35  37     yes   1    no   no     yes male    no 11    no 6.79122  2    6
## 14  36  30     yes   1    no   no     yes male    no 11    no 6.81564  2    7
\end{verbatim}

The variable \texttt{id} and \texttt{time} indicate individual and time
indexs. We use these two variables to apply panel data models.
Additionally, we use the following variables:

\begin{itemize}
\tightlist
\item
  \texttt{exp}: years of full-time work experience
\item
  \texttt{sqexp}: squared value of \texttt{exp}
\item
  \texttt{lwage}: logarithm of wage
\end{itemize}

\begin{Shaded}
\begin{Highlighting}[]
\NormalTok{dt \textless{}{-}}\StringTok{ }\NormalTok{dt[,}\KeywordTok{c}\NormalTok{(}\StringTok{"id"}\NormalTok{, }\StringTok{"time"}\NormalTok{, }\StringTok{"exp"}\NormalTok{, }\StringTok{"lwage"}\NormalTok{)]}
\NormalTok{dt}\OperatorTok{$}\NormalTok{sqexp \textless{}{-}}\StringTok{ }\NormalTok{dt}\OperatorTok{$}\NormalTok{exp}\OperatorTok{\^{}}\DecValTok{2}
\KeywordTok{summary}\NormalTok{(dt)}
\end{Highlighting}
\end{Shaded}

\begin{verbatim}
##        id           time        exp            lwage           sqexp       
##  Min.   :  1   Min.   :1   Min.   : 1.00   Min.   :4.605   Min.   :   1.0  
##  1st Qu.:149   1st Qu.:2   1st Qu.:11.00   1st Qu.:6.395   1st Qu.: 121.0  
##  Median :298   Median :4   Median :18.00   Median :6.685   Median : 324.0  
##  Mean   :298   Mean   :4   Mean   :19.85   Mean   :6.676   Mean   : 514.4  
##  3rd Qu.:447   3rd Qu.:6   3rd Qu.:29.00   3rd Qu.:6.953   3rd Qu.: 841.0  
##  Max.   :595   Max.   :7   Max.   :51.00   Max.   :8.537   Max.   :2601.0
\end{verbatim}

To examine the effect of labor experience on wages, we want to estimate
the following linear panel data model:

\[
  \text{lwage}_{it} = 
  \beta_1 \cdot \text{exp}_{it} +
  \beta_2 \cdot \text{sqexp}_{it} + 
  u_{it}.
\]

We can define the regression equation as the \texttt{formula} object in
\texttt{R}. To exclude the intercept, we must specify \texttt{-1} in the
rhs of regression equation. Thus, in \texttt{R}, we define the linear
panel data model as follows:

\begin{Shaded}
\begin{Highlighting}[]
\NormalTok{model \textless{}{-}}\StringTok{ }\NormalTok{lwage }\OperatorTok{\textasciitilde{}}\StringTok{ }\DecValTok{{-}1} \OperatorTok{+}\StringTok{ }\NormalTok{exp }\OperatorTok{+}\StringTok{ }\NormalTok{sqexp}
\end{Highlighting}
\end{Shaded}

\hypertarget{pooled-ols}{%
\subsection{Pooled OLS}\label{pooled-ols}}

We want to estimate the above regression equation by the OLS method. We
will discuss assumptions for implementation. Let \(\mathbf{X}_{it}\) be
a \(1 \times K\) (stochastic) explanatory vector. This vector contains
\texttt{exp}, \texttt{sqexp}. Let \(Y_{it}\) be a random variable of
outcome, that is \texttt{lwage}. Then, the linear panel data model can
be rewritten as follows:

\[
  Y_{it} = \mathbf{X}_{it} \beta + u_{it}, \quad t = 1, \ldots, T, \quad i = 1, \ldots, n.
\]

Using notations
\(\underline{\mathbf{X}}_i = (\mathbf{X}'_{i1}, \ldots, \mathbf{X}'_{iT})'\)
and \(\underline{Y}_i = (Y_{i1}, \ldots, Y_{iT})'\), and
\(\underline{u}_i = (u_{i1}, \ldots, u_{iT})'\), we can reformulate this
model as follows:

\[
  \underline{Y}_i = \underline{\mathbf{X}}_i \beta + \underline{u}_i, \quad \forall i.
\]

Now, we assume

\begin{enumerate}
\def\labelenumi{\arabic{enumi}.}
\tightlist
\item
  \(E[\mathbf{X}'_{it}u_{it}] = 0\), \(\forall i, t\). This assumption,
  called \emph{(contempraneous) exogneity assumption}, implies that
  \(u_{it}\) and \(\mathbf{X}_{it}\) are orthogonal in the conditional
  mean sence, \(E[u_{it} | \mathbf{X}_{it}] = 0\). However, this
  assumption does not imply \(u_{it}\) is uncorrelated with the
  explanatory variables in all time periods (strictly exogeneity), that
  is, \(E[u_{it} | \mathbf{X}_{i1}, \ldots, \mathbf{X}_{iT}] = 0\). This
  assumption palces no restriction on the relationship between
  \(\mathbf{X}_{is}\) and \(u_{it}\) for \(s\not=t\).
\item
  \(E[\underline{\mathbf{X}}'_i\underline{\mathbf{X}}_i] \succ 0\).
\end{enumerate}

Under these two assumptions, the true parameter is given by \[
  \beta = E[\underline{\mathbf{X}}'_i\underline{\mathbf{X}}_i]^{-1} E[\underline{\mathbf{X}}'_i\underline{Y}_i].
\]

Hence, the OLSE (pooled OLSE) is given by \[
  \hat{\beta} 
  = \left( \frac{1}{n} \sum_{i=1}^n \underline{\mathbf{X}}'_i\underline{\mathbf{X}}_i \right)^{-1}
  \left( \frac{1}{n} \sum_{i=1}^n \underline{\mathbf{X}}'_i\underline{Y}_i \right)
  = \left( \frac{1}{n} \sum_{i=1}^n \sum_{t=1}^T \mathbf{X}'_{it} \mathbf{X}_{it} \right)^{-1}
  \left( \frac{1}{n} \sum_{i=1}^n \sum_{t=1}^T \mathbf{X}'_{it} Y_{it} \right).
\]

Using the full matrix notation, the OLS estimator is \[
  \hat{\beta} = (\mathbf{X}' \mathbf{X})^{-1} (\mathbf{X}' Y),
\] where
\(\mathbf{X} = (\underline{\mathbf{X}}_1, \ldots, \underline{\mathbf{X}}_n)'\)
and \(Y = (\underline{Y}_1, \ldots, \underline{Y}_n)'\).

In \texttt{R} programming, the \texttt{lm} function provides the pooled
OLSE in the context of panel data model. Another way is the \texttt{plm}
function in the package \texttt{plm}. When you want to estimate pooled
OLS by the \texttt{plm} function, you need to specify
\texttt{model\ =\ "pooling"}. Moreover, you should specify individual
and time index using \texttt{index} augment. This augment passes
\texttt{index\ =\ c("individual\ index",\ "time\ index")}.

\begin{Shaded}
\begin{Highlighting}[]
\NormalTok{bols1 \textless{}{-}}\StringTok{ }\KeywordTok{lm}\NormalTok{(model, }\DataTypeTok{data =}\NormalTok{ dt)}
\NormalTok{bols2 \textless{}{-}}\StringTok{ }\KeywordTok{plm}\NormalTok{(model, }\DataTypeTok{data =}\NormalTok{ dt, }\DataTypeTok{model =} \StringTok{"pooling"}\NormalTok{, }\DataTypeTok{index =} \KeywordTok{c}\NormalTok{(}\StringTok{"id"}\NormalTok{, }\StringTok{"time"}\NormalTok{))}
\end{Highlighting}
\end{Shaded}

The pooled OLS estimator is consistent and asymptotically normally
distributed. \[
  \sqrt{n}(\hat{\beta} - \beta) \sim N(0, A^{-1} B A^{-1}),
\] where \(A = E[\underline{\mathbf{X}}'_i\underline{\mathbf{X}}_i]\)
and
\(B = E[\underline{\mathbf{X}}'_i \underline{u}_i \underline{u}'_i \underline{\mathbf{X}}_i]\).
The consistent estimator of the asymptotic variance covariance matrix is
given by \[
  \hat{A}^{-1} \hat{B} \hat{A}^{-1} = 
  \left( \frac{1}{n} \sum_{i=1}^n \underline{\mathbf{X}}'_i\underline{\mathbf{X}}_i \right)^{-1}
  \left( \frac{1}{n} \sum_{i=1}^n \underline{\mathbf{X}}'_i \underline{u}_i \underline{u}'_i \underline{\mathbf{X}}_i \right)
  \left( \frac{1}{n} \sum_{i=1}^n \underline{\mathbf{X}}'_i\underline{\mathbf{X}}_i \right)^{-1}
\] Thus, estimator of asymptotic variance of the pooled OLSE is \[
  \hat{Avar}(\hat{\beta}) =
  \left( \sum_{i=1}^n \underline{\mathbf{X}}'_i\underline{\mathbf{X}}_i \right)^{-1}
  \left( \sum_{i=1}^n \underline{\mathbf{X}}'_i \underline{u}_i \underline{u}'_i \underline{\mathbf{X}}_i \right)
  \left( \sum_{i=1}^n \underline{\mathbf{X}}'_i\underline{\mathbf{X}}_i \right)^{-1}.
\] Using the full matrix notations, we can reformulate \[
  \hat{Avar}(\hat{\beta}) =
  (\mathbf{X}' \mathbf{X})^{-1}
  (\mathbf{X}' \Omega \mathbf{X})
  (\mathbf{X}' \mathbf{X})^{-1},
\] where \[
  \Omega = 
  \begin{pmatrix}
    \underline{u}_1 \underline{u}'_1 & \mathbf{0} & \cdots & \mathbf{0} \\
    \mathbf{0} & \underline{u}_2 \underline{u}'_2 & \cdots & \mathbf{0} \\
    \vdots & \vdots & \cdots & \vdots \\
    \mathbf{0} & \mathbf{0} & \cdots & \underline{u}_n \underline{u}'_n
  \end{pmatrix}.
\] The standard errors calculated by this matrix is called \emph{robust
standard errors clustered by individuals}.

In \texttt{R}, the \texttt{lm} and \texttt{plm} function provide the
standard errors based on
\(\hat{Avar}(\hat{\beta}) = \hat{\sigma}^2 (X'X)^{-1}\), where
\(\hat{\sigma}^2 = \hat{u}\hat{u}'/(nT - K)\) and
\(\hat{u} = Y - X \hat{\beta}\). There are two ways to obtain cluster
robust standard errors. The first way is to caclulate by yourself. The
second way is to use the \texttt{coeftest} function in the package
\texttt{lmtest}. When you use this function, we should use the
\texttt{plm} function to estimate the pooled OLSE, and the
\texttt{vcovHC} function (the package \texttt{sandwich}) in the
\texttt{vcov} augment of \texttt{coeftest} fucntion.

\begin{Shaded}
\begin{Highlighting}[]
\CommentTok{\# Setup}
\NormalTok{N \textless{}{-}}\StringTok{ }\KeywordTok{length}\NormalTok{(}\KeywordTok{unique}\NormalTok{(dt}\OperatorTok{$}\NormalTok{id)); T \textless{}{-}}\StringTok{ }\KeywordTok{length}\NormalTok{(}\KeywordTok{unique}\NormalTok{(dt}\OperatorTok{$}\NormalTok{time))}
\NormalTok{X \textless{}{-}}\StringTok{ }\KeywordTok{model.matrix}\NormalTok{(bols1); k \textless{}{-}}\StringTok{ }\KeywordTok{ncol}\NormalTok{(X)}

\CommentTok{\# Inference}
\NormalTok{uhat \textless{}{-}}\StringTok{ }\NormalTok{bols1}\OperatorTok{$}\NormalTok{residuals}
\NormalTok{uhatset \textless{}{-}}\StringTok{ }\KeywordTok{matrix}\NormalTok{(}\DecValTok{0}\NormalTok{, }\DataTypeTok{nrow =} \KeywordTok{nrow}\NormalTok{(X), }\DataTypeTok{ncol =} \KeywordTok{nrow}\NormalTok{(X))}

\NormalTok{i\_from \textless{}{-}}\StringTok{ }\DecValTok{1}\NormalTok{; j\_from \textless{}{-}}\StringTok{ }\DecValTok{1}
\ControlFlowTok{for}\NormalTok{ (i }\ControlFlowTok{in} \DecValTok{1}\OperatorTok{:}\KeywordTok{max}\NormalTok{(dt}\OperatorTok{$}\NormalTok{id)) \{}
\NormalTok{  x \textless{}{-}}\StringTok{ }\KeywordTok{as.numeric}\NormalTok{(}\KeywordTok{rownames}\NormalTok{(dt))[dt}\OperatorTok{$}\NormalTok{id }\OperatorTok{==}\StringTok{ }\NormalTok{i]}
\NormalTok{  usq \textless{}{-}}\StringTok{ }\NormalTok{uhat[x] }\OperatorTok{\%*\%}\StringTok{ }\KeywordTok{t}\NormalTok{(uhat[x])}
\NormalTok{  i\_to \textless{}{-}}\StringTok{ }\NormalTok{i\_from }\OperatorTok{+}\StringTok{ }\KeywordTok{nrow}\NormalTok{(usq) }\OperatorTok{{-}}\StringTok{ }\DecValTok{1}
\NormalTok{  j\_to \textless{}{-}}\StringTok{ }\NormalTok{j\_from }\OperatorTok{+}\StringTok{ }\KeywordTok{ncol}\NormalTok{(usq) }\OperatorTok{{-}}\StringTok{ }\DecValTok{1}
\NormalTok{  uhatset[i\_from}\OperatorTok{:}\NormalTok{i\_to, j\_from}\OperatorTok{:}\NormalTok{j\_to] \textless{}{-}}\StringTok{ }\NormalTok{usq}
\NormalTok{  i\_from \textless{}{-}}\StringTok{ }\NormalTok{i\_to }\OperatorTok{+}\StringTok{ }\DecValTok{1}\NormalTok{; j\_from \textless{}{-}}\StringTok{ }\NormalTok{j\_to }\OperatorTok{+}\StringTok{ }\DecValTok{1}
\NormalTok{\}}

\NormalTok{Ahat \textless{}{-}}\StringTok{ }\KeywordTok{t}\NormalTok{(X) }\OperatorTok{\%*\%}\StringTok{ }\NormalTok{X}
\NormalTok{Bhat \textless{}{-}}\StringTok{ }\KeywordTok{t}\NormalTok{(X) }\OperatorTok{\%*\%}\StringTok{ }\NormalTok{uhatset }\OperatorTok{\%*\%}\StringTok{ }\NormalTok{X}
\NormalTok{vcovols \textless{}{-}}\StringTok{ }\KeywordTok{solve}\NormalTok{(Ahat) }\OperatorTok{\%*\%}\StringTok{ }\NormalTok{Bhat }\OperatorTok{\%*\%}\StringTok{ }\KeywordTok{solve}\NormalTok{(Ahat)}
\NormalTok{seols \textless{}{-}}\StringTok{ }\KeywordTok{sqrt}\NormalTok{(}\KeywordTok{diag}\NormalTok{(vcovols))}

\CommentTok{\# Easy way}
\KeywordTok{library}\NormalTok{(lmtest)}
\KeywordTok{library}\NormalTok{(sandwich)}
\NormalTok{easy\_cluster \textless{}{-}}\StringTok{ }\KeywordTok{coeftest}\NormalTok{(}
\NormalTok{  bols2, }\DataTypeTok{vcov =} \KeywordTok{vcovHC}\NormalTok{(bols2, }\DataTypeTok{type =} \StringTok{"HC0"}\NormalTok{, }\DataTypeTok{cluster =} \StringTok{"group"}\NormalTok{))}
\end{Highlighting}
\end{Shaded}

The result is shown in the first column of \ref{pdm}. The partial effect
of experience represents the percent change of wages. Thus, \[
  (\text{\% Change of Wage}) = 64.6 - 2 \cdot 1.3 \cdot \text{exp}.
\] For example, wages increase by 12.99\% at a mathematical mean of
labor experience (\texttt{exp}). Moreover, this result implies
diminishing marginal returns of labor experience.

\end{document}
