% Options for packages loaded elsewhere
\PassOptionsToPackage{unicode}{hyperref}
\PassOptionsToPackage{hyphens}{url}
%
\documentclass[
  12pt,
]{article}
\usepackage{lmodern}
\usepackage{amssymb,amsmath}
\usepackage{ifxetex,ifluatex}
\ifnum 0\ifxetex 1\fi\ifluatex 1\fi=0 % if pdftex
  \usepackage[T1]{fontenc}
  \usepackage[utf8]{inputenc}
  \usepackage{textcomp} % provide euro and other symbols
\else % if luatex or xetex
  \usepackage{unicode-math}
  \defaultfontfeatures{Scale=MatchLowercase}
  \defaultfontfeatures[\rmfamily]{Ligatures=TeX,Scale=1}
\fi
% Use upquote if available, for straight quotes in verbatim environments
\IfFileExists{upquote.sty}{\usepackage{upquote}}{}
\IfFileExists{microtype.sty}{% use microtype if available
  \usepackage[]{microtype}
  \UseMicrotypeSet[protrusion]{basicmath} % disable protrusion for tt fonts
}{}
\makeatletter
\@ifundefined{KOMAClassName}{% if non-KOMA class
  \IfFileExists{parskip.sty}{%
    \usepackage{parskip}
  }{% else
    \setlength{\parindent}{0pt}
    \setlength{\parskip}{6pt plus 2pt minus 1pt}}
}{% if KOMA class
  \KOMAoptions{parskip=half}}
\makeatother
\usepackage{xcolor}
\IfFileExists{xurl.sty}{\usepackage{xurl}}{} % add URL line breaks if available
\IfFileExists{bookmark.sty}{\usepackage{bookmark}}{\usepackage{hyperref}}
\hypersetup{
  pdftitle={Econometrics II TA Session \#8},
  pdfauthor={Hiroki Kato},
  hidelinks,
  pdfcreator={LaTeX via pandoc}}
\urlstyle{same} % disable monospaced font for URLs
\usepackage[margin=1in]{geometry}
\usepackage{color}
\usepackage{fancyvrb}
\newcommand{\VerbBar}{|}
\newcommand{\VERB}{\Verb[commandchars=\\\{\}]}
\DefineVerbatimEnvironment{Highlighting}{Verbatim}{commandchars=\\\{\}}
% Add ',fontsize=\small' for more characters per line
\usepackage{framed}
\definecolor{shadecolor}{RGB}{248,248,248}
\newenvironment{Shaded}{\begin{snugshade}}{\end{snugshade}}
\newcommand{\AlertTok}[1]{\textcolor[rgb]{0.94,0.16,0.16}{#1}}
\newcommand{\AnnotationTok}[1]{\textcolor[rgb]{0.56,0.35,0.01}{\textbf{\textit{#1}}}}
\newcommand{\AttributeTok}[1]{\textcolor[rgb]{0.77,0.63,0.00}{#1}}
\newcommand{\BaseNTok}[1]{\textcolor[rgb]{0.00,0.00,0.81}{#1}}
\newcommand{\BuiltInTok}[1]{#1}
\newcommand{\CharTok}[1]{\textcolor[rgb]{0.31,0.60,0.02}{#1}}
\newcommand{\CommentTok}[1]{\textcolor[rgb]{0.56,0.35,0.01}{\textit{#1}}}
\newcommand{\CommentVarTok}[1]{\textcolor[rgb]{0.56,0.35,0.01}{\textbf{\textit{#1}}}}
\newcommand{\ConstantTok}[1]{\textcolor[rgb]{0.00,0.00,0.00}{#1}}
\newcommand{\ControlFlowTok}[1]{\textcolor[rgb]{0.13,0.29,0.53}{\textbf{#1}}}
\newcommand{\DataTypeTok}[1]{\textcolor[rgb]{0.13,0.29,0.53}{#1}}
\newcommand{\DecValTok}[1]{\textcolor[rgb]{0.00,0.00,0.81}{#1}}
\newcommand{\DocumentationTok}[1]{\textcolor[rgb]{0.56,0.35,0.01}{\textbf{\textit{#1}}}}
\newcommand{\ErrorTok}[1]{\textcolor[rgb]{0.64,0.00,0.00}{\textbf{#1}}}
\newcommand{\ExtensionTok}[1]{#1}
\newcommand{\FloatTok}[1]{\textcolor[rgb]{0.00,0.00,0.81}{#1}}
\newcommand{\FunctionTok}[1]{\textcolor[rgb]{0.00,0.00,0.00}{#1}}
\newcommand{\ImportTok}[1]{#1}
\newcommand{\InformationTok}[1]{\textcolor[rgb]{0.56,0.35,0.01}{\textbf{\textit{#1}}}}
\newcommand{\KeywordTok}[1]{\textcolor[rgb]{0.13,0.29,0.53}{\textbf{#1}}}
\newcommand{\NormalTok}[1]{#1}
\newcommand{\OperatorTok}[1]{\textcolor[rgb]{0.81,0.36,0.00}{\textbf{#1}}}
\newcommand{\OtherTok}[1]{\textcolor[rgb]{0.56,0.35,0.01}{#1}}
\newcommand{\PreprocessorTok}[1]{\textcolor[rgb]{0.56,0.35,0.01}{\textit{#1}}}
\newcommand{\RegionMarkerTok}[1]{#1}
\newcommand{\SpecialCharTok}[1]{\textcolor[rgb]{0.00,0.00,0.00}{#1}}
\newcommand{\SpecialStringTok}[1]{\textcolor[rgb]{0.31,0.60,0.02}{#1}}
\newcommand{\StringTok}[1]{\textcolor[rgb]{0.31,0.60,0.02}{#1}}
\newcommand{\VariableTok}[1]{\textcolor[rgb]{0.00,0.00,0.00}{#1}}
\newcommand{\VerbatimStringTok}[1]{\textcolor[rgb]{0.31,0.60,0.02}{#1}}
\newcommand{\WarningTok}[1]{\textcolor[rgb]{0.56,0.35,0.01}{\textbf{\textit{#1}}}}
\usepackage{longtable,booktabs}
% Correct order of tables after \paragraph or \subparagraph
\usepackage{etoolbox}
\makeatletter
\patchcmd\longtable{\par}{\if@noskipsec\mbox{}\fi\par}{}{}
\makeatother
% Allow footnotes in longtable head/foot
\IfFileExists{footnotehyper.sty}{\usepackage{footnotehyper}}{\usepackage{footnote}}
\makesavenoteenv{longtable}
\usepackage{graphicx}
\makeatletter
\def\maxwidth{\ifdim\Gin@nat@width>\linewidth\linewidth\else\Gin@nat@width\fi}
\def\maxheight{\ifdim\Gin@nat@height>\textheight\textheight\else\Gin@nat@height\fi}
\makeatother
% Scale images if necessary, so that they will not overflow the page
% margins by default, and it is still possible to overwrite the defaults
% using explicit options in \includegraphics[width, height, ...]{}
\setkeys{Gin}{width=\maxwidth,height=\maxheight,keepaspectratio}
% Set default figure placement to htbp
\makeatletter
\def\fps@figure{htbp}
\makeatother
\setlength{\emergencystretch}{3em} % prevent overfull lines
\providecommand{\tightlist}{%
  \setlength{\itemsep}{0pt}\setlength{\parskip}{0pt}}
\setcounter{secnumdepth}{5}
\usepackage{zxjatype}
\setCJKmainfont[BoldFont = IPAゴシック]{IPA明朝}
\setCJKsansfont{IPAゴシック}
\setCJKmonofont{IPAゴシック}
\parindent = 1em
\newcommand{\argmax}{\mathop{\rm arg~max}\limits}
\newcommand{\argmin}{\mathop{\rm arg~min}\limits}
\DeclareMathOperator*{\plim}{plim}
\usepackage{xcolor}
\ifluatex
  \usepackage{selnolig}  % disable illegal ligatures
\fi

\title{Econometrics II TA Session \#8}
\author{Hiroki Kato}
\date{}

\begin{document}
\maketitle

\hypertarget{empirical-application-of-panel-data-model-earnings-equation}{%
\section{Empirical Application of Panel Data Model: Earnings
Equation}\label{empirical-application-of-panel-data-model-earnings-equation}}

\hypertarget{backgruond}{%
\subsection{Backgruond}\label{backgruond}}

A researcher wants to estimate the effect of full-time work experience
on wages. He uses a \emph{balanced} panel of 595 individuals from 1976
to 1982, taken from the Panel Study of Income Dynamics (PSID). The
\emph{balanced} panel data means that we can observe all individuals
every year.

\begin{Shaded}
\begin{Highlighting}[]
\NormalTok{dt \textless{}{-}}\StringTok{ }\KeywordTok{read.csv}\NormalTok{(}\StringTok{"./data/wages.csv"}\NormalTok{)}
\KeywordTok{head}\NormalTok{(dt, }\DecValTok{14}\NormalTok{)}
\end{Highlighting}
\end{Shaded}

\begin{verbatim}
##    exp wks bluecol ind south smsa married  sex union ed black   lwage id time
## 1    3  32      no   0   yes   no     yes male    no  9    no 5.56068  1    1
## 2    4  43      no   0   yes   no     yes male    no  9    no 5.72031  1    2
## 3    5  40      no   0   yes   no     yes male    no  9    no 5.99645  1    3
## 4    6  39      no   0   yes   no     yes male    no  9    no 5.99645  1    4
## 5    7  42      no   1   yes   no     yes male    no  9    no 6.06146  1    5
## 6    8  35      no   1   yes   no     yes male    no  9    no 6.17379  1    6
## 7    9  32      no   1   yes   no     yes male    no  9    no 6.24417  1    7
## 8   30  34     yes   0    no   no     yes male    no 11    no 6.16331  2    1
## 9   31  27     yes   0    no   no     yes male    no 11    no 6.21461  2    2
## 10  32  33     yes   1    no   no     yes male   yes 11    no 6.26340  2    3
## 11  33  30     yes   1    no   no     yes male    no 11    no 6.54391  2    4
## 12  34  30     yes   1    no   no     yes male    no 11    no 6.69703  2    5
## 13  35  37     yes   1    no   no     yes male    no 11    no 6.79122  2    6
## 14  36  30     yes   1    no   no     yes male    no 11    no 6.81564  2    7
\end{verbatim}

The variable \texttt{id} and \texttt{time} indicate individual and time
indexs. We use these two variables to apply panel data models.
Additionally, we use the following variables:

\begin{itemize}
\tightlist
\item
  \texttt{exp}: years of full-time work experience
\item
  \texttt{sqexp}: squared value of \texttt{exp}
\item
  \texttt{sex}: a dummy variable taking 1 if an individual is female
\item
  \texttt{ed}: years of education
\item
  \texttt{lwage}: logarithm of wage
\end{itemize}

\begin{Shaded}
\begin{Highlighting}[]
\NormalTok{dt \textless{}{-}}\StringTok{ }\NormalTok{dt[,}\KeywordTok{c}\NormalTok{(}\StringTok{"id"}\NormalTok{, }\StringTok{"time"}\NormalTok{, }\StringTok{"exp"}\NormalTok{, }\StringTok{"sex"}\NormalTok{, }\StringTok{"ed"}\NormalTok{, }\StringTok{"lwage"}\NormalTok{)]}
\NormalTok{dt}\OperatorTok{$}\NormalTok{sqexp \textless{}{-}}\StringTok{ }\NormalTok{dt}\OperatorTok{$}\NormalTok{exp}\OperatorTok{\^{}}\DecValTok{2}
\NormalTok{dt}\OperatorTok{$}\NormalTok{sex \textless{}{-}}\StringTok{ }\KeywordTok{ifelse}\NormalTok{(}\KeywordTok{as.character}\NormalTok{(dt}\OperatorTok{$}\NormalTok{sex) }\OperatorTok{==}\StringTok{ "female"}\NormalTok{, }\DecValTok{1}\NormalTok{, }\DecValTok{0}\NormalTok{)}
\KeywordTok{summary}\NormalTok{(dt)}
\end{Highlighting}
\end{Shaded}

\begin{verbatim}
##        id           time        exp             sex               ed       
##  Min.   :  1   Min.   :1   Min.   : 1.00   Min.   :0.0000   Min.   : 4.00  
##  1st Qu.:149   1st Qu.:2   1st Qu.:11.00   1st Qu.:0.0000   1st Qu.:12.00  
##  Median :298   Median :4   Median :18.00   Median :0.0000   Median :12.00  
##  Mean   :298   Mean   :4   Mean   :19.85   Mean   :0.1126   Mean   :12.85  
##  3rd Qu.:447   3rd Qu.:6   3rd Qu.:29.00   3rd Qu.:0.0000   3rd Qu.:16.00  
##  Max.   :595   Max.   :7   Max.   :51.00   Max.   :1.0000   Max.   :17.00  
##      lwage           sqexp       
##  Min.   :4.605   Min.   :   1.0  
##  1st Qu.:6.395   1st Qu.: 121.0  
##  Median :6.685   Median : 324.0  
##  Mean   :6.676   Mean   : 514.4  
##  3rd Qu.:6.953   3rd Qu.: 841.0  
##  Max.   :8.537   Max.   :2601.0
\end{verbatim}

\hypertarget{pooled-ols}{%
\subsection{Pooled OLS}\label{pooled-ols}}

Using the OLS method, we want to estimate the following linear panel
data model:

\[
  \text{lwage}_{it} = 
  \alpha + \beta_1 \cdot \text{exp}_{it} +
  \beta_2 \cdot \text{sqexp}_{it} +
  \beta_3 \cdot \text{sex}_{it} + 
  \beta_4 \cdot \text{ed}_{it} + u_{it}.
\]

We will discuss assumptions for applying the OLS method. Let
\(\mathbf{X}_{it}\) be a \(1 \times K\) (stochastic) explanatory vector.
This vector contains \texttt{exp}, \texttt{sqexp}, \texttt{sex} and
\texttt{ed}. Let \(Y_{it}\) be a random variable of outcome, that is
\texttt{lwage}. The balanced panel data is given by

\begin{longtable}[]{@{}ccccc@{}}
\toprule
\begin{minipage}[b]{0.17\columnwidth}\centering
\strut
\end{minipage} & \begin{minipage}[b]{0.17\columnwidth}\centering
\(i = 1\)\strut
\end{minipage} & \begin{minipage}[b]{0.17\columnwidth}\centering
\(i = 2\)\strut
\end{minipage} & \begin{minipage}[b]{0.17\columnwidth}\centering
\(\cdots\)\strut
\end{minipage} & \begin{minipage}[b]{0.17\columnwidth}\centering
\(i = n\)\strut
\end{minipage}\tabularnewline
\midrule
\endhead
\begin{minipage}[t]{0.17\columnwidth}\centering
\(t = 1\)\strut
\end{minipage} & \begin{minipage}[t]{0.17\columnwidth}\centering
\((Y_{11}, \mathbf{X}_{11})\)\strut
\end{minipage} & \begin{minipage}[t]{0.17\columnwidth}\centering
\((Y_{21}, \mathbf{X}_{21})\)\strut
\end{minipage} & \begin{minipage}[t]{0.17\columnwidth}\centering
\(\cdots\)\strut
\end{minipage} & \begin{minipage}[t]{0.17\columnwidth}\centering
\((Y_{n1}, \mathbf{X}_{n1})\)\strut
\end{minipage}\tabularnewline
\begin{minipage}[t]{0.17\columnwidth}\centering
\(t = 2\)\strut
\end{minipage} & \begin{minipage}[t]{0.17\columnwidth}\centering
\((Y_{12}, \mathbf{X}_{12})\)\strut
\end{minipage} & \begin{minipage}[t]{0.17\columnwidth}\centering
\((Y_{22}, \mathbf{X}_{22})\)\strut
\end{minipage} & \begin{minipage}[t]{0.17\columnwidth}\centering
\(\cdots\)\strut
\end{minipage} & \begin{minipage}[t]{0.17\columnwidth}\centering
\((Y_{n2}, \mathbf{X}_{n2})\)\strut
\end{minipage}\tabularnewline
\begin{minipage}[t]{0.17\columnwidth}\centering
\(\vdots\)\strut
\end{minipage} & \begin{minipage}[t]{0.17\columnwidth}\centering
\(\vdots\)\strut
\end{minipage} & \begin{minipage}[t]{0.17\columnwidth}\centering
\(\vdots\)\strut
\end{minipage} & \begin{minipage}[t]{0.17\columnwidth}\centering
\(\cdots\)\strut
\end{minipage} & \begin{minipage}[t]{0.17\columnwidth}\centering
\(\vdots\)\strut
\end{minipage}\tabularnewline
\begin{minipage}[t]{0.17\columnwidth}\centering
\(t = T\)\strut
\end{minipage} & \begin{minipage}[t]{0.17\columnwidth}\centering
\((Y_{1T}, \mathbf{X}_{1T})\)\strut
\end{minipage} & \begin{minipage}[t]{0.17\columnwidth}\centering
\((Y_{2T}, \mathbf{X}_{2T})\)\strut
\end{minipage} & \begin{minipage}[t]{0.17\columnwidth}\centering
\(\cdots\)\strut
\end{minipage} & \begin{minipage}[t]{0.17\columnwidth}\centering
\((Y_{nT}, \mathbf{X}_{nT})\)\strut
\end{minipage}\tabularnewline
\bottomrule
\end{longtable}

Then, the linear panel data model can be rewritten as follows:

\[
  Y_{it} = \mathbf{X}_{it} \beta + u_{it}, \quad t = 1, \ldots, T, \quad i = 1, \ldots, n.
\]

Using notations
\(\underline{\mathbf{X}}_i = (\mathbf{X}'_{i1}, \ldots, \mathbf{X}'_{iT})'\)
and \(\underline{Y}_i = (Y_{i1}, \ldots, Y_{iT})'\), and
\(\underline{u}_i = (u_{i1}, \ldots, u_{iT})'\), we can reformulate this
model as follows:

\[
  \underline{Y}_i = \underline{\mathbf{X}}_i \beta + \underline{u}_i, \quad \forall i.
\]

Now, we assume

\begin{enumerate}
\def\labelenumi{\arabic{enumi}.}
\tightlist
\item
  \(E[\mathbf{X}'_{it}u_{it}] = 0\), \(\forall i, t\). This assumption,
  called \emph{(contempraneous) exogneity assumption}, implies that
  \(u_{it}\) and \(\mathbf{X}_{it}\) are orthogonal in the conditional
  mean sence, \(E[u_{it} | \mathbf{X}_{it}] = 0\). However, this
  assumption does not imply \(u_{it}\) is uncorrelated with the
  explanatory variables in all time periods (strictly exogeneity), that
  is, \(E[u_{it} | \mathbf{X}_{i1}, \ldots, \mathbf{X}_{iT}] = 0\). This
  assumption palces no restriction on the relationship between
  \(\mathbf{X}_{is}\) and \(u_{it}\) for \(s\not=t\).
\item
  \(E[\underline{\mathbf{X}}'_i\underline{\mathbf{X}}_i] \succ 0\).
\end{enumerate}

Under these two assumptions, the true parameter can be identified by \[
  \beta = E[\underline{\mathbf{X}}'_i\underline{\mathbf{X}}_i]^{-1} E[\underline{\mathbf{X}}'_i\underline{Y}_i].
\] Hence, the OLSE (pooled OLSE) is given by \[
  \hat{\beta} 
  = \left( \frac{1}{n} \sum_{i=1}^n \underline{\mathbf{X}}'_i\underline{\mathbf{X}}_i \right)
  \left( \frac{1}{n} \sum_{i=1}^n \underline{\mathbf{X}}'_i\underline{Y}_i \right)
  = \left( \frac{1}{n} \sum_{i=1}^n \sum_{t=1}^T \mathbf{X}'_{it} \mathbf{X}_{it} \right)
  \left( \frac{1}{n} \sum_{i=1}^n \sum_{t=1}^T \mathbf{X}'_{it} Y_{it} \right).
\]

The pooled OLS estimator is consistent and asymptotically normally
distributed. \[
  \sqrt{n}(\hat{\beta} - \beta) \sim N(0, A^{-1} B A^{-1}),
\] where \(A = E[\underline{\mathbf{X}}'_i\underline{\mathbf{X}}_i]\)
and
\(B = E[\underline{\mathbf{X}}'_i \underline{u}_i \underline{u}'_i \underline{\mathbf{X}}_i]\).
The consistent estimator of the asymptotic variance covariance matrix is
given by \[
  \hat{A}^{-1} \hat{B} \hat{A}^{-1} = 
  \left( \frac{1}{n} \sum_{i=1}^n \underline{\mathbf{X}}'_i\underline{\mathbf{X}}_i \right)^{-1}
  \left( \frac{1}{n} \sum_{i=1}^n \underline{\mathbf{X}}'_i \underline{u}_i \underline{u}'_i \underline{\mathbf{X}}_i \right)
  \left( \frac{1}{n} \sum_{i=1}^n \underline{\mathbf{X}}'_i\underline{\mathbf{X}}_i \right)^{-1}
\] The standard errors caclulated by this matrix is called \emph{robust
standard errors clustered by individuals}.

In \texttt{R}, the pooled OLSE can be obtained by \texttt{lm} function.
However, the \texttt{lm} function does not return the cluster-robust
standard errors. Thus, you need to caclulate them by yourself. Here is a
sample code.

\begin{Shaded}
\begin{Highlighting}[]
\CommentTok{\# OLSE}
\NormalTok{pool \textless{}{-}}\StringTok{ }\KeywordTok{lm}\NormalTok{(lwage }\OperatorTok{\textasciitilde{}}\StringTok{ }\NormalTok{exp }\OperatorTok{+}\StringTok{ }\NormalTok{sqexp }\OperatorTok{+}\StringTok{ }\NormalTok{sex }\OperatorTok{+}\StringTok{ }\NormalTok{ed, }\DataTypeTok{data =}\NormalTok{ dt)}

\CommentTok{\# Clustered SE}
\NormalTok{X \textless{}{-}}\StringTok{ }\KeywordTok{model.matrix}\NormalTok{(pool); uhat \textless{}{-}}\StringTok{ }\NormalTok{pool}\OperatorTok{$}\NormalTok{residuals}
\NormalTok{uhatset \textless{}{-}}\StringTok{ }\KeywordTok{matrix}\NormalTok{(}\DecValTok{0}\NormalTok{, }\DataTypeTok{nrow =} \KeywordTok{nrow}\NormalTok{(X), }\DataTypeTok{ncol =} \KeywordTok{nrow}\NormalTok{(X))}

\NormalTok{i\_from \textless{}{-}}\StringTok{ }\DecValTok{1}\NormalTok{; j\_from \textless{}{-}}\StringTok{ }\DecValTok{1}
\ControlFlowTok{for}\NormalTok{ (i }\ControlFlowTok{in} \DecValTok{1}\OperatorTok{:}\KeywordTok{max}\NormalTok{(dt}\OperatorTok{$}\NormalTok{id)) \{}
\NormalTok{  x \textless{}{-}}\StringTok{ }\KeywordTok{as.numeric}\NormalTok{(}\KeywordTok{rownames}\NormalTok{(dt))[dt}\OperatorTok{$}\NormalTok{id }\OperatorTok{==}\StringTok{ }\NormalTok{i]}
\NormalTok{  usq \textless{}{-}}\StringTok{ }\NormalTok{uhat[x] }\OperatorTok{\%*\%}\StringTok{ }\KeywordTok{t}\NormalTok{(uhat[x])}
\NormalTok{  i\_to \textless{}{-}}\StringTok{ }\NormalTok{i\_from }\OperatorTok{+}\StringTok{ }\KeywordTok{nrow}\NormalTok{(usq) }\OperatorTok{{-}}\StringTok{ }\DecValTok{1}
\NormalTok{  j\_to \textless{}{-}}\StringTok{ }\NormalTok{j\_from }\OperatorTok{+}\StringTok{ }\KeywordTok{ncol}\NormalTok{(usq) }\OperatorTok{{-}}\StringTok{ }\DecValTok{1}
\NormalTok{  uhatset[i\_from}\OperatorTok{:}\NormalTok{i\_to, j\_from}\OperatorTok{:}\NormalTok{j\_to] \textless{}{-}}\StringTok{ }\NormalTok{usq}
\NormalTok{  i\_from \textless{}{-}}\StringTok{ }\NormalTok{i\_to }\OperatorTok{+}\StringTok{ }\DecValTok{1}\NormalTok{; j\_from \textless{}{-}}\StringTok{ }\NormalTok{j\_to }\OperatorTok{+}\StringTok{ }\DecValTok{1}
\NormalTok{\}}

\NormalTok{Ahat \textless{}{-}}\StringTok{ }\KeywordTok{t}\NormalTok{(X) }\OperatorTok{\%*\%}\StringTok{ }\NormalTok{X}
\NormalTok{Bhat \textless{}{-}}\StringTok{ }\KeywordTok{t}\NormalTok{(X) }\OperatorTok{\%*\%}\StringTok{ }\NormalTok{uhatset }\OperatorTok{\%*\%}\StringTok{ }\NormalTok{X}
\NormalTok{clust\_vcov \textless{}{-}}\StringTok{ }\KeywordTok{solve}\NormalTok{(Ahat) }\OperatorTok{\%*\%}\StringTok{ }\NormalTok{Bhat }\OperatorTok{\%*\%}\StringTok{ }\KeywordTok{solve}\NormalTok{(Ahat)}
\NormalTok{clust\_se \textless{}{-}}\StringTok{ }\KeywordTok{sqrt}\NormalTok{(}\KeywordTok{diag}\NormalTok{(clust\_vcov))}

\KeywordTok{print}\NormalTok{(}\StringTok{"Pooled OLSE"}\NormalTok{); }\KeywordTok{coef}\NormalTok{(pool)}
\end{Highlighting}
\end{Shaded}

\begin{verbatim}
## [1] "Pooled OLSE"
\end{verbatim}

\begin{verbatim}
##   (Intercept)           exp         sqexp           sex            ed 
##  5.2759322858  0.0427794655 -0.0007022517 -0.4305537695  0.0747976640
\end{verbatim}

\begin{Shaded}
\begin{Highlighting}[]
\KeywordTok{print}\NormalTok{(}\StringTok{"SE of pooled OLSE"}\NormalTok{); clust\_se}
\end{Highlighting}
\end{Shaded}

\begin{verbatim}
## [1] "SE of pooled OLSE"
\end{verbatim}

\begin{verbatim}
##  (Intercept)          exp        sqexp          sex           ed 
## 0.0846130179 0.0047667140 0.0001103059 0.0339740598 0.0048139358
\end{verbatim}

Alternatively, using the \texttt{plm} function (the package
\texttt{plm}) and the \texttt{coeftest} function (the package
\texttt{lmtest}), you can obtain the asymptotic variance covariance
matrix of pooled OLSE easily. The \texttt{plm} function provides the
panel data model. When you want to estimate pooled OLS, you need to
specify \texttt{model\ =\ "pooling"}. Moreover, you should specify
individual and time index using \texttt{index} augment. This augment
passes \texttt{index\ =\ c("individual\ index",\ "time\ index")}. After
estimating the pooled OLS by the \texttt{plm} function, you must use the
\texttt{coeftest} function to obtain the cluster-robust standard errors.
To caclulate the clustered standard errors, you should use the
\texttt{vcovHC} function in the \texttt{vcov} augment.

\begin{Shaded}
\begin{Highlighting}[]
\KeywordTok{library}\NormalTok{(plm)}
\KeywordTok{library}\NormalTok{(lmtest)}
\KeywordTok{library}\NormalTok{(sandwich)}
\NormalTok{test \textless{}{-}}\StringTok{ }\KeywordTok{plm}\NormalTok{(}
\NormalTok{  lwage }\OperatorTok{\textasciitilde{}}\StringTok{ }\NormalTok{exp }\OperatorTok{+}\StringTok{ }\NormalTok{sqexp }\OperatorTok{+}\StringTok{ }\NormalTok{sex }\OperatorTok{+}\StringTok{ }\NormalTok{ed, }
  \DataTypeTok{data =}\NormalTok{ dt, }\DataTypeTok{model =} \StringTok{"pooling"}\NormalTok{, }\DataTypeTok{index =} \KeywordTok{c}\NormalTok{(}\StringTok{"id"}\NormalTok{, }\StringTok{"time"}\NormalTok{))}
\KeywordTok{coeftest}\NormalTok{(test, }\DataTypeTok{vcov =} \KeywordTok{vcovHC}\NormalTok{(test, }\DataTypeTok{type =} \StringTok{"HC0"}\NormalTok{, }\DataTypeTok{cluster =} \StringTok{"group"}\NormalTok{))}
\end{Highlighting}
\end{Shaded}

\begin{verbatim}
## 
## t test of coefficients:
## 
##                Estimate  Std. Error  t value  Pr(>|t|)    
## (Intercept)  5.27593229  0.08461302  62.3537 < 2.2e-16 ***
## exp          0.04277947  0.00476671   8.9746 < 2.2e-16 ***
## sqexp       -0.00070225  0.00011031  -6.3664 2.145e-10 ***
## sex         -0.43055377  0.03397406 -12.6730 < 2.2e-16 ***
## ed           0.07479766  0.00481394  15.5377 < 2.2e-16 ***
## ---
## Signif. codes:  0 '***' 0.001 '**' 0.01 '*' 0.05 '.' 0.1 ' ' 1
\end{verbatim}

\end{document}
