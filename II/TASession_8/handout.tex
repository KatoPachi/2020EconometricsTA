% Options for packages loaded elsewhere
\PassOptionsToPackage{unicode}{hyperref}
\PassOptionsToPackage{hyphens}{url}
%
\documentclass[
  12pt,
]{article}
\usepackage{lmodern}
\usepackage{amssymb,amsmath}
\usepackage{ifxetex,ifluatex}
\ifnum 0\ifxetex 1\fi\ifluatex 1\fi=0 % if pdftex
  \usepackage[T1]{fontenc}
  \usepackage[utf8]{inputenc}
  \usepackage{textcomp} % provide euro and other symbols
\else % if luatex or xetex
  \usepackage{unicode-math}
  \defaultfontfeatures{Scale=MatchLowercase}
  \defaultfontfeatures[\rmfamily]{Ligatures=TeX,Scale=1}
\fi
% Use upquote if available, for straight quotes in verbatim environments
\IfFileExists{upquote.sty}{\usepackage{upquote}}{}
\IfFileExists{microtype.sty}{% use microtype if available
  \usepackage[]{microtype}
  \UseMicrotypeSet[protrusion]{basicmath} % disable protrusion for tt fonts
}{}
\makeatletter
\@ifundefined{KOMAClassName}{% if non-KOMA class
  \IfFileExists{parskip.sty}{%
    \usepackage{parskip}
  }{% else
    \setlength{\parindent}{0pt}
    \setlength{\parskip}{6pt plus 2pt minus 1pt}}
}{% if KOMA class
  \KOMAoptions{parskip=half}}
\makeatother
\usepackage{xcolor}
\IfFileExists{xurl.sty}{\usepackage{xurl}}{} % add URL line breaks if available
\IfFileExists{bookmark.sty}{\usepackage{bookmark}}{\usepackage{hyperref}}
\hypersetup{
  pdftitle={Econometrics II TA Session \#8},
  pdfauthor={Hiroki Kato},
  hidelinks,
  pdfcreator={LaTeX via pandoc}}
\urlstyle{same} % disable monospaced font for URLs
\usepackage[margin=1in]{geometry}
\usepackage{color}
\usepackage{fancyvrb}
\newcommand{\VerbBar}{|}
\newcommand{\VERB}{\Verb[commandchars=\\\{\}]}
\DefineVerbatimEnvironment{Highlighting}{Verbatim}{commandchars=\\\{\}}
% Add ',fontsize=\small' for more characters per line
\usepackage{framed}
\definecolor{shadecolor}{RGB}{248,248,248}
\newenvironment{Shaded}{\begin{snugshade}}{\end{snugshade}}
\newcommand{\AlertTok}[1]{\textcolor[rgb]{0.94,0.16,0.16}{#1}}
\newcommand{\AnnotationTok}[1]{\textcolor[rgb]{0.56,0.35,0.01}{\textbf{\textit{#1}}}}
\newcommand{\AttributeTok}[1]{\textcolor[rgb]{0.77,0.63,0.00}{#1}}
\newcommand{\BaseNTok}[1]{\textcolor[rgb]{0.00,0.00,0.81}{#1}}
\newcommand{\BuiltInTok}[1]{#1}
\newcommand{\CharTok}[1]{\textcolor[rgb]{0.31,0.60,0.02}{#1}}
\newcommand{\CommentTok}[1]{\textcolor[rgb]{0.56,0.35,0.01}{\textit{#1}}}
\newcommand{\CommentVarTok}[1]{\textcolor[rgb]{0.56,0.35,0.01}{\textbf{\textit{#1}}}}
\newcommand{\ConstantTok}[1]{\textcolor[rgb]{0.00,0.00,0.00}{#1}}
\newcommand{\ControlFlowTok}[1]{\textcolor[rgb]{0.13,0.29,0.53}{\textbf{#1}}}
\newcommand{\DataTypeTok}[1]{\textcolor[rgb]{0.13,0.29,0.53}{#1}}
\newcommand{\DecValTok}[1]{\textcolor[rgb]{0.00,0.00,0.81}{#1}}
\newcommand{\DocumentationTok}[1]{\textcolor[rgb]{0.56,0.35,0.01}{\textbf{\textit{#1}}}}
\newcommand{\ErrorTok}[1]{\textcolor[rgb]{0.64,0.00,0.00}{\textbf{#1}}}
\newcommand{\ExtensionTok}[1]{#1}
\newcommand{\FloatTok}[1]{\textcolor[rgb]{0.00,0.00,0.81}{#1}}
\newcommand{\FunctionTok}[1]{\textcolor[rgb]{0.00,0.00,0.00}{#1}}
\newcommand{\ImportTok}[1]{#1}
\newcommand{\InformationTok}[1]{\textcolor[rgb]{0.56,0.35,0.01}{\textbf{\textit{#1}}}}
\newcommand{\KeywordTok}[1]{\textcolor[rgb]{0.13,0.29,0.53}{\textbf{#1}}}
\newcommand{\NormalTok}[1]{#1}
\newcommand{\OperatorTok}[1]{\textcolor[rgb]{0.81,0.36,0.00}{\textbf{#1}}}
\newcommand{\OtherTok}[1]{\textcolor[rgb]{0.56,0.35,0.01}{#1}}
\newcommand{\PreprocessorTok}[1]{\textcolor[rgb]{0.56,0.35,0.01}{\textit{#1}}}
\newcommand{\RegionMarkerTok}[1]{#1}
\newcommand{\SpecialCharTok}[1]{\textcolor[rgb]{0.00,0.00,0.00}{#1}}
\newcommand{\SpecialStringTok}[1]{\textcolor[rgb]{0.31,0.60,0.02}{#1}}
\newcommand{\StringTok}[1]{\textcolor[rgb]{0.31,0.60,0.02}{#1}}
\newcommand{\VariableTok}[1]{\textcolor[rgb]{0.00,0.00,0.00}{#1}}
\newcommand{\VerbatimStringTok}[1]{\textcolor[rgb]{0.31,0.60,0.02}{#1}}
\newcommand{\WarningTok}[1]{\textcolor[rgb]{0.56,0.35,0.01}{\textbf{\textit{#1}}}}
\usepackage{graphicx}
\makeatletter
\def\maxwidth{\ifdim\Gin@nat@width>\linewidth\linewidth\else\Gin@nat@width\fi}
\def\maxheight{\ifdim\Gin@nat@height>\textheight\textheight\else\Gin@nat@height\fi}
\makeatother
% Scale images if necessary, so that they will not overflow the page
% margins by default, and it is still possible to overwrite the defaults
% using explicit options in \includegraphics[width, height, ...]{}
\setkeys{Gin}{width=\maxwidth,height=\maxheight,keepaspectratio}
% Set default figure placement to htbp
\makeatletter
\def\fps@figure{htbp}
\makeatother
\setlength{\emergencystretch}{3em} % prevent overfull lines
\providecommand{\tightlist}{%
  \setlength{\itemsep}{0pt}\setlength{\parskip}{0pt}}
\setcounter{secnumdepth}{5}
\usepackage{zxjatype}
\setCJKmainfont[BoldFont = IPAゴシック]{IPA明朝}
\setCJKsansfont{IPAゴシック}
\setCJKmonofont{IPAゴシック}
\parindent = 1em
\newcommand{\argmax}{\mathop{\rm arg~max}\limits}
\newcommand{\argmin}{\mathop{\rm arg~min}\limits}
\DeclareMathOperator*{\plim}{plim}
\usepackage{xcolor}
\ifluatex
  \usepackage{selnolig}  % disable illegal ligatures
\fi

\title{Econometrics II TA Session \#8}
\author{Hiroki Kato}
\date{}

\begin{document}
\maketitle

\hypertarget{empirical-application-of-panel-data-model-earnings-equation}{%
\section{Empirical Application of Panel Data Model: Earnings
Equation}\label{empirical-application-of-panel-data-model-earnings-equation}}

\hypertarget{backgruond}{%
\subsection{Backgruond}\label{backgruond}}

A researcher wants to estimate the effect of full-time work experience
on wages. He uses a \emph{balanced} panel of 595 individuals from 1976
to 1982, taken from the Panel Study of Income Dynamics (PSID). The
\emph{balanced} panel data means that we can observe all individuals
every year.

\begin{Shaded}
\begin{Highlighting}[]
\NormalTok{dt \textless{}{-}}\StringTok{ }\KeywordTok{read.csv}\NormalTok{(}\StringTok{"./data/wages.csv"}\NormalTok{)}
\KeywordTok{head}\NormalTok{(dt, }\DecValTok{14}\NormalTok{)}
\end{Highlighting}
\end{Shaded}

\begin{verbatim}
##    exp wks bluecol ind south smsa married  sex union ed black   lwage id time
## 1    3  32      no   0   yes   no     yes male    no  9    no 5.56068  1    1
## 2    4  43      no   0   yes   no     yes male    no  9    no 5.72031  1    2
## 3    5  40      no   0   yes   no     yes male    no  9    no 5.99645  1    3
## 4    6  39      no   0   yes   no     yes male    no  9    no 5.99645  1    4
## 5    7  42      no   1   yes   no     yes male    no  9    no 6.06146  1    5
## 6    8  35      no   1   yes   no     yes male    no  9    no 6.17379  1    6
## 7    9  32      no   1   yes   no     yes male    no  9    no 6.24417  1    7
## 8   30  34     yes   0    no   no     yes male    no 11    no 6.16331  2    1
## 9   31  27     yes   0    no   no     yes male    no 11    no 6.21461  2    2
## 10  32  33     yes   1    no   no     yes male   yes 11    no 6.26340  2    3
## 11  33  30     yes   1    no   no     yes male    no 11    no 6.54391  2    4
## 12  34  30     yes   1    no   no     yes male    no 11    no 6.69703  2    5
## 13  35  37     yes   1    no   no     yes male    no 11    no 6.79122  2    6
## 14  36  30     yes   1    no   no     yes male    no 11    no 6.81564  2    7
\end{verbatim}

The variable \texttt{id} and \texttt{time} indicate individual and time
indexs. We use these two variables to apply panel data models.
Additionally, we use the following variables:

\begin{itemize}
\tightlist
\item
  \texttt{exp}: years of full-time work experience
\item
  \texttt{sqexp}: squared value of \texttt{exp}
\item
  \texttt{lwage}: logarithm of wage
\end{itemize}

\begin{Shaded}
\begin{Highlighting}[]
\NormalTok{dt \textless{}{-}}\StringTok{ }\NormalTok{dt[,}\KeywordTok{c}\NormalTok{(}\StringTok{"id"}\NormalTok{, }\StringTok{"time"}\NormalTok{, }\StringTok{"exp"}\NormalTok{, }\StringTok{"lwage"}\NormalTok{)]}
\NormalTok{dt}\OperatorTok{$}\NormalTok{sqexp \textless{}{-}}\StringTok{ }\NormalTok{dt}\OperatorTok{$}\NormalTok{exp}\OperatorTok{\^{}}\DecValTok{2}
\KeywordTok{summary}\NormalTok{(dt)}
\end{Highlighting}
\end{Shaded}

\begin{verbatim}
##        id           time        exp            lwage           sqexp       
##  Min.   :  1   Min.   :1   Min.   : 1.00   Min.   :4.605   Min.   :   1.0  
##  1st Qu.:149   1st Qu.:2   1st Qu.:11.00   1st Qu.:6.395   1st Qu.: 121.0  
##  Median :298   Median :4   Median :18.00   Median :6.685   Median : 324.0  
##  Mean   :298   Mean   :4   Mean   :19.85   Mean   :6.676   Mean   : 514.4  
##  3rd Qu.:447   3rd Qu.:6   3rd Qu.:29.00   3rd Qu.:6.953   3rd Qu.: 841.0  
##  Max.   :595   Max.   :7   Max.   :51.00   Max.   :8.537   Max.   :2601.0
\end{verbatim}

To examine the effect of labor experience on wages, we want to estimate
the following linear panel data model:

\[
  \text{lwage}_{it} = 
  \beta_1 \cdot \text{exp}_{it} +
  \beta_2 \cdot \text{sqexp}_{it} + 
  u_{it}.
\]

We can define the regression equation as the \texttt{formula} object in
\texttt{R}. To exclude the intercept, we must specify \texttt{-1} in the
rhs of regression equation. Thus, in \texttt{R}, we define the linear
panel data model as follows:

\begin{Shaded}
\begin{Highlighting}[]
\NormalTok{model \textless{}{-}}\StringTok{ }\NormalTok{lwage }\OperatorTok{\textasciitilde{}}\StringTok{ }\DecValTok{{-}1} \OperatorTok{+}\StringTok{ }\NormalTok{exp }\OperatorTok{+}\StringTok{ }\NormalTok{sqexp}
\end{Highlighting}
\end{Shaded}

\hypertarget{pooled-olse}{%
\subsection{Pooled OLSE}\label{pooled-olse}}

We want to estimate the above regression equation by the OLS method. We
will discuss assumptions for implementation. Let \(\mathbf{X}_{it}\) be
a \(1 \times K\) (stochastic) explanatory vector. This vector contains
\texttt{exp} and \texttt{sqexp}. Let \(Y_{it}\) be a random variable of
outcome, that is, \texttt{lwage}. Then, the linear panel data model can
be rewritten as follows:

\[
  Y_{it} = \mathbf{X}_{it} \beta + u_{it}, \quad t = 1, \ldots, T, \quad i = 1, \ldots, n.
\]

Using notations
\(\underline{\mathbf{X}}_i = (\mathbf{X}'_{i1}, \ldots, \mathbf{X}'_{iT})'\)
and \(\underline{Y}_i = (Y_{i1}, \ldots, Y_{iT})'\), and
\(\underline{u}_i = (u_{i1}, \ldots, u_{iT})'\), we can reformulate this
model as follows:

\[
  \underline{Y}_i = \underline{\mathbf{X}}_i \beta + \underline{u}_i, \quad \forall i.
\]

Now, we assume

\begin{enumerate}
\def\labelenumi{\arabic{enumi}.}
\tightlist
\item
  (contempraneous) exogneity assumption:
  \(E[\mathbf{X}'_{it}u_{it}] = 0\), \(\forall i, t\).

  \begin{itemize}
  \tightlist
  \item
    This assumption,implies that \(u_{it}\) and \(\mathbf{X}_{it}\) are
    orthogonal in the conditional mean sence,
    \(E[u_{it} | \mathbf{X}_{it}] = 0\). However, this assumption does
    not imply \(u_{it}\) is uncorrelated with the explanatory variables
    in all time periods (strictly exogeneity), that is,
    \(E[u_{it} | \mathbf{X}_{i1}, \ldots, \mathbf{X}_{iT}] = 0\). This
    assumption palces no restriction on the relationship between
    \(\mathbf{X}_{is}\) and \(u_{it}\) for \(s\not=t\).
  \end{itemize}
\item
  \(E[\underline{\mathbf{X}}'_i\underline{\mathbf{X}}_i] \succ 0\).
\end{enumerate}

Under these two assumptions, the true parameter is given by \[
  \beta = E[\underline{\mathbf{X}}'_i\underline{\mathbf{X}}_i]^{-1} E[\underline{\mathbf{X}}'_i\underline{Y}_i].
\]

Hence, the OLSE (pooled OLSE) is given by \[
  \hat{\beta} 
  = \left( \frac{1}{n} \sum_{i=1}^n \underline{\mathbf{X}}'_i\underline{\mathbf{X}}_i \right)^{-1}
  \left( \frac{1}{n} \sum_{i=1}^n \underline{\mathbf{X}}'_i\underline{Y}_i \right)
  = \left( \frac{1}{n} \sum_{i=1}^n \sum_{t=1}^T \mathbf{X}'_{it} \mathbf{X}_{it} \right)^{-1}
  \left( \frac{1}{n} \sum_{i=1}^n \sum_{t=1}^T \mathbf{X}'_{it} Y_{it} \right).
\]

Using the full matrix notation, the OLS estimator is \[
  \hat{\beta} = (\mathbf{X}' \mathbf{X})^{-1} (\mathbf{X}' Y),
\] where
\(\mathbf{X} = (\underline{\mathbf{X}}_1, \ldots, \underline{\mathbf{X}}_n)'\)
and \(Y = (\underline{Y}_1, \ldots, \underline{Y}_n)'\).

In \texttt{R} programming, the \texttt{lm} function provides the pooled
OLSE in the context of panel data model. Another way is the \texttt{plm}
function in the package \texttt{plm}. When you want to estimate pooled
OLS by the \texttt{plm} function, you need to specify
\texttt{model\ =\ "pooling"}. Moreover, you should specify individual
and time index using \texttt{index} augment. This augment passes
\texttt{index\ =\ c("individual\ index",\ "time\ index")}.

\begin{Shaded}
\begin{Highlighting}[]
\NormalTok{bols1 \textless{}{-}}\StringTok{ }\KeywordTok{lm}\NormalTok{(model, }\DataTypeTok{data =}\NormalTok{ dt)}

\KeywordTok{library}\NormalTok{(plm)}
\NormalTok{bols2 \textless{}{-}}\StringTok{ }\KeywordTok{plm}\NormalTok{(model, }\DataTypeTok{data =}\NormalTok{ dt, }\DataTypeTok{model =} \StringTok{"pooling"}\NormalTok{, }\DataTypeTok{index =} \KeywordTok{c}\NormalTok{(}\StringTok{"id"}\NormalTok{, }\StringTok{"time"}\NormalTok{))}
\end{Highlighting}
\end{Shaded}

The pooled OLS estimator is consistent and asymptotically normally
distributed. \[
  \sqrt{n}(\hat{\beta} - \beta) \sim N_{\mathbb{R}^K}(0, A^{-1} B A^{-1}),
\] where \(A = E[\underline{\mathbf{X}}'_i\underline{\mathbf{X}}_i]\)
and
\(B = E[\underline{\mathbf{X}}'_i \underline{u}_i \underline{u}'_i \underline{\mathbf{X}}_i]\).
The consistent estimator of A and B is given by \[
  \hat{A} = \frac{1}{n} \sum_{i=1}^n \underline{\mathbf{X}}'_i\underline{\mathbf{X}}_i,
\] \[
  \hat{B} = \frac{1}{n} \sum_{i=1}^n \underline{\mathbf{X}}'_i \underline{\hat{u}}_i \underline{\hat{u}}'_i \underline{\mathbf{X}}_i.
\] Thus, estimator of asymptotic variance of the pooled OLSE is \[
  \widehat{Asyvar}(\hat{\beta}) =
  \left( \sum_{i=1}^n \underline{\mathbf{X}}'_i\underline{\mathbf{X}}_i \right)^{-1}
  \left( \sum_{i=1}^n \underline{\mathbf{X}}'_i \underline{u}_i \underline{u}'_i \underline{\mathbf{X}}_i \right)
  \left( \sum_{i=1}^n \underline{\mathbf{X}}'_i\underline{\mathbf{X}}_i \right)^{-1}.
\] Using the full matrix notations, we can reformulate \[
  \widehat{Asyvar}(\hat{\beta}) =
  (\mathbf{X}' \mathbf{X})^{-1}
  (\mathbf{X}' U \mathbf{X})
  (\mathbf{X}' \mathbf{X})^{-1},
\] where \[
  U = 
  \begin{pmatrix}
    \underline{\hat{u}}_1 \underline{\hat{u}}'_1 & \mathbf{0} & \cdots & \mathbf{0} \\
    \mathbf{0} & \underline{\hat{u}}_2 \underline{\hat{u}}'_2 & \cdots & \mathbf{0} \\
    \vdots & \vdots & \cdots & \vdots \\
    \mathbf{0} & \mathbf{0} & \cdots & \underline{\hat{u}}_n \underline{\hat{u}}'_n
  \end{pmatrix}.
\] The standard errors calculated by this matrix is called \emph{robust
standard errors clustered by individuals}.

In \texttt{R}, the \texttt{lm} and \texttt{plm} function provide the
standard errors based on
\(\widehat{Asyvar}(\hat{\beta}) = \hat{\sigma}^2 (X'X)^{-1}\), where
\(\hat{\sigma}^2 = \hat{u}\hat{u}'/(nT - K)\) and
\(\hat{u} = Y - X \hat{\beta}\). There are two ways to obtain cluster
robust standard errors. The first way is to caclulate by yourself. The
second way is to use the \texttt{coeftest} function in the package
\texttt{lmtest}. When you use this function, we should use the
\texttt{plm} function to estimate the pooled OLSE, and the
\texttt{vcovHC} function (the package \texttt{sandwich}) in the
\texttt{vcov} augment of \texttt{coeftest} fucntion.

\begin{Shaded}
\begin{Highlighting}[]
\CommentTok{\# Setup}
\NormalTok{N \textless{}{-}}\StringTok{ }\KeywordTok{length}\NormalTok{(}\KeywordTok{unique}\NormalTok{(dt}\OperatorTok{$}\NormalTok{id)); T \textless{}{-}}\StringTok{ }\KeywordTok{length}\NormalTok{(}\KeywordTok{unique}\NormalTok{(dt}\OperatorTok{$}\NormalTok{time))}
\NormalTok{X \textless{}{-}}\StringTok{ }\KeywordTok{model.matrix}\NormalTok{(bols1); k \textless{}{-}}\StringTok{ }\KeywordTok{ncol}\NormalTok{(X)}

\CommentTok{\# Inference}
\NormalTok{uhat \textless{}{-}}\StringTok{ }\NormalTok{bols1}\OperatorTok{$}\NormalTok{residuals}
\NormalTok{uhatset \textless{}{-}}\StringTok{ }\KeywordTok{matrix}\NormalTok{(}\DecValTok{0}\NormalTok{, }\DataTypeTok{nrow =} \KeywordTok{nrow}\NormalTok{(X), }\DataTypeTok{ncol =} \KeywordTok{nrow}\NormalTok{(X))}

\NormalTok{i\_from \textless{}{-}}\StringTok{ }\DecValTok{1}\NormalTok{; j\_from \textless{}{-}}\StringTok{ }\DecValTok{1}
\ControlFlowTok{for}\NormalTok{ (i }\ControlFlowTok{in} \DecValTok{1}\OperatorTok{:}\KeywordTok{max}\NormalTok{(dt}\OperatorTok{$}\NormalTok{id)) \{}
\NormalTok{  x \textless{}{-}}\StringTok{ }\KeywordTok{as.numeric}\NormalTok{(}\KeywordTok{rownames}\NormalTok{(dt))[dt}\OperatorTok{$}\NormalTok{id }\OperatorTok{==}\StringTok{ }\NormalTok{i]}
\NormalTok{  usq \textless{}{-}}\StringTok{ }\NormalTok{uhat[x] }\OperatorTok{\%*\%}\StringTok{ }\KeywordTok{t}\NormalTok{(uhat[x])}
\NormalTok{  i\_to \textless{}{-}}\StringTok{ }\NormalTok{i\_from }\OperatorTok{+}\StringTok{ }\KeywordTok{nrow}\NormalTok{(usq) }\OperatorTok{{-}}\StringTok{ }\DecValTok{1}
\NormalTok{  j\_to \textless{}{-}}\StringTok{ }\NormalTok{j\_from }\OperatorTok{+}\StringTok{ }\KeywordTok{ncol}\NormalTok{(usq) }\OperatorTok{{-}}\StringTok{ }\DecValTok{1}
\NormalTok{  uhatset[i\_from}\OperatorTok{:}\NormalTok{i\_to, j\_from}\OperatorTok{:}\NormalTok{j\_to] \textless{}{-}}\StringTok{ }\NormalTok{usq}
\NormalTok{  i\_from \textless{}{-}}\StringTok{ }\NormalTok{i\_to }\OperatorTok{+}\StringTok{ }\DecValTok{1}\NormalTok{; j\_from \textless{}{-}}\StringTok{ }\NormalTok{j\_to }\OperatorTok{+}\StringTok{ }\DecValTok{1}
\NormalTok{\}}

\NormalTok{Ahat \textless{}{-}}\StringTok{ }\KeywordTok{t}\NormalTok{(X) }\OperatorTok{\%*\%}\StringTok{ }\NormalTok{X}
\NormalTok{Bhat \textless{}{-}}\StringTok{ }\KeywordTok{t}\NormalTok{(X) }\OperatorTok{\%*\%}\StringTok{ }\NormalTok{uhatset }\OperatorTok{\%*\%}\StringTok{ }\NormalTok{X}
\NormalTok{vcovols \textless{}{-}}\StringTok{ }\KeywordTok{solve}\NormalTok{(Ahat) }\OperatorTok{\%*\%}\StringTok{ }\NormalTok{Bhat }\OperatorTok{\%*\%}\StringTok{ }\KeywordTok{solve}\NormalTok{(Ahat)}
\NormalTok{seols \textless{}{-}}\StringTok{ }\KeywordTok{sqrt}\NormalTok{(}\KeywordTok{diag}\NormalTok{(vcovols))}

\CommentTok{\# Easy way}
\KeywordTok{library}\NormalTok{(lmtest)}
\KeywordTok{library}\NormalTok{(sandwich)}
\NormalTok{easy\_cluster \textless{}{-}}\StringTok{ }\KeywordTok{coeftest}\NormalTok{(}
\NormalTok{  bols2, }\DataTypeTok{vcov =} \KeywordTok{vcovHC}\NormalTok{(bols2, }\DataTypeTok{type =} \StringTok{"HC0"}\NormalTok{, }\DataTypeTok{cluster =} \StringTok{"group"}\NormalTok{))}
\end{Highlighting}
\end{Shaded}

The result is shown in the first column of Table \ref{pdm}. The partial
effect of experience represents the percent change of wages. Thus, \[
  (\text{\% Change of Wage}) = 64.6 - 2 \cdot 1.3 \cdot \text{exp}.
\] For example, wages increase by 12.99\% at a mathematical mean of
labor experience (\texttt{exp}). Moreover, this result implies
diminishing marginal returns of labor experience.

\hypertarget{feasible-glse}{%
\subsection{Feasible GLSE}\label{feasible-glse}}

Adding and assumption of the conditional variance of \(\underline{u}_i\)
allows for using the Generalized Ordinary Squares method. To implement,
we assume

\begin{enumerate}
\def\labelenumi{\arabic{enumi}.}
\tightlist
\item
  \(E[\underline{X}_i \otimes \underline{u}_i] = 0\). A sufficient
  condition to satisfy this relationship is
  \(E[ \underline{u}_i | \underline{X}_i] = 0\). This assumption implies
  \(E[\underline{X}'_i \Omega^{-1} \underline{u}_i] = 0\) where
  \(\Omega = E[\underline{u}_i\underline{u}'_i]\) is \(T \times T\)
  matrix.
\item
  \(\Omega \succ 0\) and
  \(E[\underline{X}'_i \Omega^{-1} \underline{X}'_i] \succ 0\).
\end{enumerate}

The GLS estimator is given by \[
  \hat{\beta}_{GLS} 
  = \left( \frac{1}{n} \sum_{i=1}^n \underline{\mathbf{X}}'_i \Omega^{-1} \underline{\mathbf{X}}_i \right)^{-1}
  \left( \frac{1}{n} \sum_{i=1}^n \underline{\mathbf{X}}'_i \Omega^{-1} \underline{Y}_i \right).
\] Under two assumptions, this estimator is weakly consistent.

In the feasible GLS method, we replace the unknown \(\Omega\) with a
consistent estimator. Here, we consider the two-step FGLS: obtain the
OLS estimator and residuals; replace \(\Omega\) by it. Then, the unknown
\(\Omega\) is replaced by \[
  \hat{\Omega} = \frac{1}{n} \sum_{i=1}^n \underline{\hat{u}}_i\underline{\hat{u}}'_i,
\] where
\(\underline{\hat{u}}_i = \underline{Y}_i - \underline{\mathbf{X}}_i \hat{\beta}_{OLS}\).

Thus, the FGLS estimator is \[
  \hat{\beta}_{FGLS} 
  = \left( \frac{1}{n} \sum_{i=1}^n \underline{\mathbf{X}}'_i \hat{\Omega}^{-1} \underline{\mathbf{X}}_i \right)^{-1}
  \left( \frac{1}{n} \sum_{i=1}^n \underline{\mathbf{X}}'_i \hat{\Omega}^{-1} \underline{Y}_i \right).
\] Using the full matrix notations, \[
  \hat{\beta}_{FGLS}
  = \{ \mathbf{X}'(I_n \otimes \hat{\Omega}^{-1}) \mathbf{X} \}^{-1}
  \{ \mathbf{X}'(I_n \otimes \hat{\Omega}^{-1}) Y \}.
\]

In the \texttt{R} programming, there are two ways to obtain the FGLS
estimator. The first way is to caclulate by yourself. The second way is
to use the \texttt{pggls} function in the package \texttt{plm}. When you
use the \texttt{pggls} function, you should specify individual and time
indexs using \texttt{index} augment, and type in
\texttt{model\ =\ "pooling"}.

\begin{Shaded}
\begin{Highlighting}[]
\CommentTok{\# Setup}
\NormalTok{X \textless{}{-}}\StringTok{ }\KeywordTok{model.matrix}\NormalTok{(model, dt); k \textless{}{-}}\StringTok{ }\KeywordTok{ncol}\NormalTok{(X)}
\NormalTok{y \textless{}{-}}\StringTok{ }\NormalTok{dt}\OperatorTok{$}\NormalTok{lwage}
\NormalTok{N \textless{}{-}}\StringTok{ }\KeywordTok{length}\NormalTok{(}\KeywordTok{unique}\NormalTok{(dt}\OperatorTok{$}\NormalTok{id)); T \textless{}{-}}\StringTok{ }\KeywordTok{length}\NormalTok{(}\KeywordTok{unique}\NormalTok{(dt}\OperatorTok{$}\NormalTok{time))}

\CommentTok{\# Estimator of Omega}
\NormalTok{uhat \textless{}{-}}\StringTok{ }\NormalTok{bols1}\OperatorTok{$}\NormalTok{residuals}

\NormalTok{Omega\_sum \textless{}{-}}\StringTok{ }\KeywordTok{matrix}\NormalTok{(}\DecValTok{0}\NormalTok{, }\DataTypeTok{ncol =}\NormalTok{ T, }\DataTypeTok{nrow =}\NormalTok{ T)}
\ControlFlowTok{for}\NormalTok{ (i }\ControlFlowTok{in} \DecValTok{1}\OperatorTok{:}\NormalTok{N) \{}
\NormalTok{  x \textless{}{-}}\StringTok{ }\KeywordTok{as.numeric}\NormalTok{(}\KeywordTok{rownames}\NormalTok{(dt))[dt}\OperatorTok{$}\NormalTok{id }\OperatorTok{==}\StringTok{ }\NormalTok{i]}
\NormalTok{  Omega\_sum \textless{}{-}}\StringTok{ }\NormalTok{uhat[x] }\OperatorTok{\%*\%}\StringTok{ }\KeywordTok{t}\NormalTok{(uhat[x]) }\OperatorTok{+}\StringTok{ }\NormalTok{Omega\_sum}
\NormalTok{\}}
\NormalTok{Omega \textless{}{-}}\StringTok{ }\NormalTok{Omega\_sum}\OperatorTok{/}\NormalTok{N}

\CommentTok{\# FGLS estimator}
\NormalTok{kroOmega \textless{}{-}}\StringTok{ }\KeywordTok{diag}\NormalTok{(N) }\OperatorTok{\%x\%}\StringTok{ }\KeywordTok{solve}\NormalTok{(Omega)}
\NormalTok{bfgls \textless{}{-}}\StringTok{ }\KeywordTok{solve}\NormalTok{(}\KeywordTok{t}\NormalTok{(X) }\OperatorTok{\%*\%}\StringTok{ }\NormalTok{kroOmega }\OperatorTok{\%*\%}\StringTok{ }\NormalTok{X) }\OperatorTok{\%*\%}\StringTok{ }\NormalTok{(}\KeywordTok{t}\NormalTok{(X) }\OperatorTok{\%*\%}\StringTok{ }\NormalTok{kroOmega }\OperatorTok{\%*\%}\StringTok{ }\NormalTok{y)}

\CommentTok{\# Easy way!!!}
\NormalTok{easy\_fgls \textless{}{-}}\StringTok{ }\KeywordTok{pggls}\NormalTok{(}
\NormalTok{  model, }\DataTypeTok{data =}\NormalTok{ dt, }\DataTypeTok{index =} \KeywordTok{c}\NormalTok{(}\StringTok{"id"}\NormalTok{, }\StringTok{"time"}\NormalTok{), }\DataTypeTok{model =} \StringTok{"pooling"}\NormalTok{)}
\end{Highlighting}
\end{Shaded}

The asymptotic distribution of the FGLS estimator is given by \[
  \sqrt{n}(\hat{\beta}_{FGLS} - \beta) \sim N_{\mathbb{R}^K}(0, A^{-1} B A^{-1}),
\] where
\(A = E[\underline{\mathbf{X}}'_i \Omega^{-1} \underline{\mathbf{X}}_i]\)
and
\(B = E[\underline{\mathbf{X}}'_i \Omega^{-1} \underline{u}_i \underline{u}'_i \Omega^{-1} \underline{\mathbf{X}}_i]\).
The consistent estimator of \(A\) and \(B\) is \[
  \hat{A} = \frac{1}{n} \sum_{i=1}^n \underline{\mathbf{X}}'_i \hat{\Omega}^{-1} \underline{\mathbf{X}}_i,
\] \[
  \hat{B} 
  = \frac{1}{n} \sum_{i=1}^n
  \underline{\mathbf{X}}'_i \hat{\Omega}^{-1} 
  \underline{\hat{u}}^{FLGS}_i \underline{\hat{u}}^{FGLS'}_i 
  \hat{\Omega}^{-1} \underline{\mathbf{X}}_i,
\] where
\(\underline{u}^{FLGS}_i = \underline{Y}_i - \underline{\mathbf{X}}_i \hat{\beta}_{FGLS}\).
Thus, estimator of asymptotic variance of the FGLS estimator is \[
  \widehat{Asyvar}(\hat{\beta}_{FGLS}) =
  \left( \sum_{i=1}^n \underline{\mathbf{X}}'_i \hat{\Omega}^{-1} \underline{\mathbf{X}}_i \right)^{-1}
  \left( \sum_{i=1}^n \underline{\mathbf{X}}'_i \hat{\Omega}^{-1} 
  \underline{u}^{FLGS}_i \underline{u}^{FGLS'}_i 
  \hat{\Omega}^{-1} \underline{\mathbf{X}}_i \right)
  \left( \sum_{i=1}^n \underline{\mathbf{X}}'_i \hat{\Omega}^{-1} \underline{\mathbf{X}}_i \right)^{-1}.
\] Using the full matrix notations, \[
  \widehat{Asyvar}(\hat{\beta}_{FGLS}) =
  \{ \mathbf{X}'(I_n \otimes \hat{\Omega}^{-1}) \mathbf{X} \}^{-1}
  \{ \mathbf{X}'(I_n \otimes \hat{\Omega}^{-1}) U (I_n \otimes \hat{\Omega}^{-1}) \mathbf{X} \}^{-1}
  \{ \mathbf{X}'(I_n \otimes \hat{\Omega}^{-1}) \mathbf{X} \}^{-1},
\] where \[
  U = 
  \begin{pmatrix}
    \underline{\hat{u}}^{FLGS}_1 \underline{\hat{u}}^{FGLS'}_1 & \mathbf{0} & \cdots & \mathbf{0} \\
    \mathbf{0} & \underline{\hat{u}}^{FLGS}_2 \underline{\hat{u}}^{FGLS'}_2 & \cdots & \mathbf{0} \\
    \vdots & \vdots & \cdots & \vdots \\
    \mathbf{0} & \mathbf{0} & \cdots & \underline{\hat{u}}^{FLGS}_n \underline{\hat{u}}^{FGLS'}_n
  \end{pmatrix}.
\]

In the \texttt{R} programming, you need to caclculate by yourself. The
\texttt{pggls} function provides the FGLS estimator. However, this
function calculates standard errors, assuming \emph{system
homoskedasticity}, that is,
\(E[\underline{\mathbf{X}}'_i \Omega^{-1} \underline{u}_i \underline{u}'_i \Omega^{-1} \underline{\mathbf{X}}_i] = E[\underline{\mathbf{X}}'_i \Omega^{-1} \underline{\mathbf{X}}_i]\).
If you can rationale this assumption, the \texttt{bggls} fucntion is the
easiest way to carry out statistical inference.

\begin{Shaded}
\begin{Highlighting}[]
\NormalTok{ufgls \textless{}{-}}\StringTok{ }\NormalTok{y }\OperatorTok{{-}}\StringTok{ }\NormalTok{X }\OperatorTok{\%*\%}\StringTok{ }\NormalTok{bfgls}
\NormalTok{uhatset \textless{}{-}}\StringTok{ }\KeywordTok{matrix}\NormalTok{(}\DecValTok{0}\NormalTok{, }\DataTypeTok{nrow =} \KeywordTok{nrow}\NormalTok{(X), }\DataTypeTok{ncol =} \KeywordTok{nrow}\NormalTok{(X))}
\NormalTok{i\_from \textless{}{-}}\StringTok{ }\DecValTok{1}\NormalTok{; j\_from \textless{}{-}}\StringTok{ }\DecValTok{1}
\ControlFlowTok{for}\NormalTok{ (i }\ControlFlowTok{in} \DecValTok{1}\OperatorTok{:}\KeywordTok{max}\NormalTok{(dt}\OperatorTok{$}\NormalTok{id)) \{}
\NormalTok{  x \textless{}{-}}\StringTok{ }\KeywordTok{as.numeric}\NormalTok{(}\KeywordTok{rownames}\NormalTok{(dt))[dt}\OperatorTok{$}\NormalTok{id }\OperatorTok{==}\StringTok{ }\NormalTok{i]}
\NormalTok{  usq \textless{}{-}}\StringTok{ }\NormalTok{uhat[x] }\OperatorTok{\%*\%}\StringTok{ }\KeywordTok{t}\NormalTok{(uhat[x])}
\NormalTok{  i\_to \textless{}{-}}\StringTok{ }\NormalTok{i\_from }\OperatorTok{+}\StringTok{ }\KeywordTok{nrow}\NormalTok{(usq) }\OperatorTok{{-}}\StringTok{ }\DecValTok{1}
\NormalTok{  j\_to \textless{}{-}}\StringTok{ }\NormalTok{j\_from }\OperatorTok{+}\StringTok{ }\KeywordTok{ncol}\NormalTok{(usq) }\OperatorTok{{-}}\StringTok{ }\DecValTok{1}
\NormalTok{  uhatset[i\_from}\OperatorTok{:}\NormalTok{i\_to, j\_from}\OperatorTok{:}\NormalTok{j\_to] \textless{}{-}}\StringTok{ }\NormalTok{usq}
\NormalTok{  i\_from \textless{}{-}}\StringTok{ }\NormalTok{i\_to }\OperatorTok{+}\StringTok{ }\DecValTok{1}\NormalTok{; j\_from \textless{}{-}}\StringTok{ }\NormalTok{j\_to }\OperatorTok{+}\StringTok{ }\DecValTok{1}
\NormalTok{\}}

\NormalTok{Ahat \textless{}{-}}\StringTok{ }\KeywordTok{t}\NormalTok{(X) }\OperatorTok{\%*\%}\StringTok{ }\NormalTok{kroOmega }\OperatorTok{\%*\%}\StringTok{ }\NormalTok{X}
\NormalTok{Bhat \textless{}{-}}\StringTok{ }\KeywordTok{t}\NormalTok{(X) }\OperatorTok{\%*\%}\StringTok{ }\NormalTok{kroOmega }\OperatorTok{\%*\%}\StringTok{ }\NormalTok{uhatset }\OperatorTok{\%*\%}\StringTok{ }\NormalTok{kroOmega }\OperatorTok{\%*\%}\StringTok{ }\NormalTok{X}
\NormalTok{vcovfgls \textless{}{-}}\StringTok{ }\KeywordTok{solve}\NormalTok{(Ahat) }\OperatorTok{\%*\%}\StringTok{ }\NormalTok{Bhat }\OperatorTok{\%*\%}\StringTok{ }\KeywordTok{solve}\NormalTok{(Ahat)}
\NormalTok{sefgls \textless{}{-}}\StringTok{ }\KeywordTok{sqrt}\NormalTok{(}\KeywordTok{diag}\NormalTok{(vcovfgls))}
\end{Highlighting}
\end{Shaded}

The result is shown in the second column of Table \ref{pdm}. The partial
effect of experience represents the percent change of wages. Thus, \[
  (\text{\% Change of Wage}) = 52.9 - 2 \cdot 0.9 \cdot \text{exp}.
\] For example, wages increase by 17.17\% at a mathematical mean of
labor experience (\texttt{exp}).

\hypertarget{fixed-effect-model}{%
\subsection{Fixed Effect Model}\label{fixed-effect-model}}

To examine the effect of labor experience on wages, we introduce
unobserved heterogeneity such as ability. The unobserved effects model
is given by \[
  \text{lwage}_{it} = 
  \beta_1 \cdot \text{exp}_{it} +
  \beta_2 \cdot \text{sqexp}_{it} + 
  c_i + u_{it},
\] where \(c_i\) is unobserved component which is constant over time,
\(u_{it}\) is the idiosyncratic error term. The fixed effect model
treats \(c_i\) as a parameter to be estimated for each cross section
unit \(i\).

We generalize the unobserved effects model as follows: \[
  Y_{it} = \mathbf{X}_{it} \beta + c_i + u_{it}, \quad t = 1, \ldots, T, \quad i = 1, \ldots, n.
\] Using notations
\(\underline{\mathbf{X}}_i = (\mathbf{X}'_{i1}, \ldots, \mathbf{X}'_{iT})'\)
and \(\underline{Y}_i = (Y_{i1}, \ldots, Y_{iT})'\), and
\(\underline{u}_i = (u_{i1}, \ldots, u_{iT})'\), we can reformulate this
model as follows: \[
  \underline{Y}_i = \underline{\mathbf{X}}_i \beta + \iota c_i + \underline{u}_i, \quad \forall i,
\] where \(\iota = (1, \ldots, 1)'\) is \(T \times 1\) vector.

To implement the fixed effect model, we assume the following three
assumptions:

\begin{enumerate}
\def\labelenumi{\arabic{enumi}.}
\tightlist
\item
  Strict exogeneity:
  \(E[u_{it} | \mathbf{X}_{i1}, \ldots, \mathbf{X}_{iT}, c_i] = 0\).
\item
  Full rank:
  \(rank(\sum_t E[\ddot{\mathbf{X}}'_{it}\ddot{\mathbf{X}}_{it}]) = rank(E[\ddot{\underline{\mathbf{X}}}'_i\ddot{\underline{\mathbf{X}}}_i]) = K\)
  where
  \(\ddot{\mathbf{X}}_{it} = \mathbf{X}_{it} - T^{-1}\sum_t \mathbf{X}_{it}\).
\item
  homoskedasticity:
  \(E[\underline{u}_i \underline{u}'_i|\mathbf{X}_{i1}, \ldots, \mathbf{X}_{iT}, c_i] = \sigma^2_u I_T\).
\end{enumerate}

To obtain the FE estimator, we consider the within transformation first.
Averaging the unobserved effects model for individual \(i\) and time
\(t\) over time yields \[
  \bar{Y}_i = \bar{\mathbf{X}}_i \beta + c_i + \bar{u}_i,
\] where \(\bar{Y}_i = T^{-1} \sum_t Y_{it}\),
\(\bar{\mathbf{X}}_i = T^{-1} \sum_t \mathbf{X}_{it}\), and
\(\bar{u}_i = T^{-1} \sum_t u_{it}\). Subtracting this equation from the
original one for each t yields \[
  Y_{it} - \bar{Y}_i = (\mathbf{X}_{it} - \bar{\mathbf{X}}_i) \beta + (u_{it} - \bar{u}_i)
  \Leftrightarrow
  \ddot{Y}_{it} = \ddot{\mathbf{X}}_{it} \beta + \ddot{u}_{it}.
\] Note that
\(E[\ddot{u}_{it} | \ddot{\mathbf{X}}_{i1}, \ldots, \ddot{\mathbf{X}}_{iT}] = 0\)
under the first assumption. Using the \(T\) system of equation, the
within transformation is \[
  Q_T \underline{Y}_i = Q_T \underline{\mathbf{X}}_i \beta + Q_T \underline{u}_i
  \Leftrightarrow
  \ddot{\underline{Y}}_i = \ddot{\underline{\mathbf{X}}}_i \beta + \ddot{\underline{u}}_i.
\] where \(Q_T = I_T - \iota (\iota' \iota)^{-1} \iota\) is
\emph{time-demeaning matrix}, and \(Q_T \iota = 0\). Using the matrix
notations, the within transformation is \[
  (I_n \otimes Q_t) Y = (I_n \otimes Q_t) X \beta + (I_n \otimes Q_t) u
  \Leftrightarrow
  \ddot{Y} = \ddot{X} \beta + \ddot{u}.
\]

Before showing the FE estimator, I will show
\(Q_T \underline{Y}_i = Y_{it} - T^{-1} \sum_t Y_{it}\), using
\texttt{R}. As an illustration, we caclulate time-demeaned outcome
variable for \(i = 1\), \(\ddot{Y}_{1t}\). R snippet is as follows:

\begin{Shaded}
\begin{Highlighting}[]
\CommentTok{\# extract outcome variables for i = 1}
\NormalTok{i \textless{}{-}}\StringTok{ }\KeywordTok{as.numeric}\NormalTok{(}\KeywordTok{rownames}\NormalTok{(dt))[dt}\OperatorTok{$}\NormalTok{id }\OperatorTok{==}\StringTok{ }\DecValTok{1}\NormalTok{]}
\NormalTok{y1 \textless{}{-}}\StringTok{ }\NormalTok{dt}\OperatorTok{$}\NormalTok{lwage[i]}

\CommentTok{\# deviation from mean}
\NormalTok{Ydev1 \textless{}{-}}\StringTok{ }\NormalTok{y1 }\OperatorTok{{-}}\StringTok{ }\KeywordTok{mean}\NormalTok{(y1)}
\KeywordTok{print}\NormalTok{(}\StringTok{"Deviation from mean across time"}\NormalTok{); Ydev1}
\end{Highlighting}
\end{Shaded}

\begin{verbatim}
## [1] "Deviation from mean across time"
\end{verbatim}

\begin{verbatim}
## [1] -0.40407857 -0.24444857  0.03169143  0.03169143  0.09670143  0.20903143
## [7]  0.27941143
\end{verbatim}

\begin{Shaded}
\begin{Highlighting}[]
\CommentTok{\# time demean{-}matrix}
\NormalTok{T \textless{}{-}}\StringTok{ }\KeywordTok{length}\NormalTok{(y1)}
\NormalTok{vec1 \textless{}{-}}\StringTok{ }\KeywordTok{rep}\NormalTok{(}\DecValTok{1}\NormalTok{, T)}
\NormalTok{Qt \textless{}{-}}\StringTok{ }\KeywordTok{diag}\NormalTok{(T) }\OperatorTok{{-}}\StringTok{ }\NormalTok{vec1 }\OperatorTok{\%*\%}\StringTok{ }\KeywordTok{solve}\NormalTok{(}\KeywordTok{t}\NormalTok{(vec1) }\OperatorTok{\%*\%}\StringTok{ }\NormalTok{vec1) }\OperatorTok{\%*\%}\StringTok{ }\KeywordTok{t}\NormalTok{(vec1)}
\NormalTok{Ydev2 \textless{}{-}}\StringTok{ }\NormalTok{Qt }\OperatorTok{\%*\%}\StringTok{ }\NormalTok{y1}
\KeywordTok{print}\NormalTok{(}\StringTok{"Time{-}demeaning matrix"}\NormalTok{); Ydev2}
\end{Highlighting}
\end{Shaded}

\begin{verbatim}
## [1] "Time-demeaning matrix"
\end{verbatim}

\begin{verbatim}
##             [,1]
## [1,] -0.40407857
## [2,] -0.24444857
## [3,]  0.03169143
## [4,]  0.03169143
## [5,]  0.09670143
## [6,]  0.20903143
## [7,]  0.27941143
\end{verbatim}

The FE estimator is given by \[
  \hat{\beta}_{FE} = 
  \left( \frac{1}{n} \sum_{i=1}^n \ddot{\underline{\mathbf{X}}}'_i 
  \ddot{\underline{\mathbf{X}}}_i \right)^{-1}
  \left( \frac{1}{n} \sum_{i=1}^n \ddot{\underline{\mathbf{X}}}'_i \ddot{\underline{Y}}_i \right)
  = (\ddot{X}' \ddot{X})^{-1}(\ddot{X}' \ddot{Y}).
\]

In the \texttt{R} programming, there are two ways to obtain the FE
estimator. The first way is to caclulate by yourself. The second way is
to use the \texttt{plm} function. When you use the \texttt{plm}
function, you need to specify \texttt{model\ =\ "within"} to implement
the FE model.

\begin{Shaded}
\begin{Highlighting}[]
\CommentTok{\# Setup}
\NormalTok{X \textless{}{-}}\StringTok{ }\KeywordTok{model.matrix}\NormalTok{(model, dt); k \textless{}{-}}\StringTok{ }\KeywordTok{ncol}\NormalTok{(X)}
\NormalTok{y \textless{}{-}}\StringTok{ }\NormalTok{dt}\OperatorTok{$}\NormalTok{lwage}
\NormalTok{N \textless{}{-}}\StringTok{ }\KeywordTok{length}\NormalTok{(}\KeywordTok{unique}\NormalTok{(dt}\OperatorTok{$}\NormalTok{id)); T \textless{}{-}}\StringTok{ }\KeywordTok{length}\NormalTok{(}\KeywordTok{unique}\NormalTok{(dt}\OperatorTok{$}\NormalTok{time))}

\CommentTok{\# FE estimator}
\NormalTok{i \textless{}{-}}\StringTok{ }\KeywordTok{rep}\NormalTok{(}\DecValTok{1}\NormalTok{, T)}
\NormalTok{Qt \textless{}{-}}\StringTok{ }\KeywordTok{diag}\NormalTok{(T) }\OperatorTok{{-}}\StringTok{ }\NormalTok{i }\OperatorTok{\%*\%}\StringTok{ }\KeywordTok{solve}\NormalTok{(}\KeywordTok{t}\NormalTok{(i) }\OperatorTok{\%*\%}\StringTok{ }\NormalTok{i) }\OperatorTok{\%*\%}\StringTok{ }\KeywordTok{t}\NormalTok{(i)}
\NormalTok{Ydev \textless{}{-}}\StringTok{ }\KeywordTok{diag}\NormalTok{(N) }\OperatorTok{\%x\%}\StringTok{ }\NormalTok{Qt }\OperatorTok{\%*\%}\StringTok{ }\NormalTok{y}
\NormalTok{Xdev \textless{}{-}}\StringTok{ }\KeywordTok{diag}\NormalTok{(N) }\OperatorTok{\%x\%}\StringTok{ }\NormalTok{Qt }\OperatorTok{\%*\%}\StringTok{ }\NormalTok{X}
\NormalTok{bfe \textless{}{-}}\StringTok{ }\KeywordTok{solve}\NormalTok{(}\KeywordTok{t}\NormalTok{(Xdev) }\OperatorTok{\%*\%}\StringTok{ }\NormalTok{Xdev) }\OperatorTok{\%*\%}\StringTok{ }\KeywordTok{t}\NormalTok{(Xdev) }\OperatorTok{\%*\%}\StringTok{ }\NormalTok{Ydev}

\CommentTok{\# Awesome way !!!}
\NormalTok{plmfe \textless{}{-}}\StringTok{ }\KeywordTok{plm}\NormalTok{(model, }\DataTypeTok{data =}\NormalTok{ dt, }\DataTypeTok{index =} \KeywordTok{c}\NormalTok{(}\StringTok{"id"}\NormalTok{, }\StringTok{"time"}\NormalTok{), }\DataTypeTok{model =} \StringTok{"within"}\NormalTok{)}
\end{Highlighting}
\end{Shaded}

Under the third assumption, asymptotic distribution of the FE estimator
is given by \[
  \sqrt{n}(\hat{\beta}_{FE} - \beta) 
  \sim N_{\mathbb{R}^K}(0, \sigma^2_u E[\ddot{\underline{\mathbf{X}}}'_i \ddot{\underline{\mathbf{X}}}_i]).
\] The consistent estimator of the asymptotic variance of the FE
estimator is \[
  \widehat{Asyvar}(\hat{\beta}_{FE}) =
  \hat{\sigma}_u^2 
  \left( \sum_{i=1}^n \ddot{\underline{\mathbf{X}}}'_i \ddot{\underline{\mathbf{X}}}_i \right)^{-1} =
  \hat{\sigma}_u^2 (\ddot{X}' \ddot{X})^{-1},
\] where
\(\hat{\sigma}_u^2 = \frac{1}{n(T-1)-K} \sum_i \sum_t \hat{\ddot{u}}_{it}\),
and
\(\hat{\ddot{u}}_{it} = \ddot{Y}_{it} - \ddot{\mathbf{X}}_{it} \hat{\beta}_{FE}\).

In the \texttt{R} programming, the \texttt{plm} function also returns
standard errors, \(\hat{\sigma}_u^2 (\ddot{X}' \ddot{X})^{-1}\). Of
course, you can compute the standard errors manually. The sample code is
as follows:

\begin{Shaded}
\begin{Highlighting}[]
\NormalTok{uhat \textless{}{-}}\StringTok{ }\NormalTok{Ydev }\OperatorTok{{-}}\StringTok{ }\NormalTok{Xdev }\OperatorTok{\%*\%}\StringTok{ }\NormalTok{bfe}
\NormalTok{sigmahat \textless{}{-}}\StringTok{ }\KeywordTok{sum}\NormalTok{(uhat}\OperatorTok{\^{}}\DecValTok{2}\NormalTok{)}\OperatorTok{/}\NormalTok{(N}\OperatorTok{*}\NormalTok{(T}\DecValTok{{-}1}\NormalTok{)}\OperatorTok{{-}}\NormalTok{k)}
\NormalTok{vcovfe \textless{}{-}}\StringTok{ }\NormalTok{sigmahat }\OperatorTok{*}\StringTok{ }\KeywordTok{solve}\NormalTok{(}\KeywordTok{t}\NormalTok{(Xdev) }\OperatorTok{\%*\%}\StringTok{ }\NormalTok{Xdev)}
\NormalTok{sefe \textless{}{-}}\StringTok{ }\KeywordTok{sqrt}\NormalTok{(}\KeywordTok{diag}\NormalTok{(vcovfe))}
\end{Highlighting}
\end{Shaded}

The result is shown in the third column in Table \ref{pdm}. The partial
effect of experience represents the percent change of wages. Thus, \[
  (\text{\% Change of Wage}) = 11.4 - 2 \cdot 0.04 \cdot \text{exp}.
\] For example, wages increase by 9.812\% at a mathematical mean of
labor experience (\texttt{exp}).

\hypertarget{random-effect-model}{%
\subsection{Random Effect Model}\label{random-effect-model}}

Again, consider the unobserved effects model: \[
  Y_{it} = \mathbf{X}_{it} \beta + c_i + u_{it}, \quad t = 1, \ldots, T, \quad i = 1, \ldots, n.
\] The random effect model treats \(c_i\) as a random variable. Thus,
the variable \(c_i\) is put into the error term. We reformulate the
model as follows: \[
  Y_{it} = \mathbf{X}_{it} \beta + v_{it},
\] where \(v_{it} = c_i + u_{it}\). Using notations
\(\underline{\mathbf{X}}_i = (\mathbf{X}'_{i1}, \ldots, \mathbf{X}'_{iT})'\)
and \(\underline{Y}_i = (Y_{i1}, \ldots, Y_{iT})'\), and
\(\underline{u}_i = (u_{i1}, \ldots, u_{iT})'\), we can reformulate this
model as follows: \[
  \underline{Y}_i = \underline{\mathbf{X}}_i \beta + \underline{v}_i,
\] where \(\underline{v}_i = \iota c_i + \underline{u}_i\), and
\(\iota = (1, \ldots, 1)'\) is \(T \times 1\) vector.

To implement the RE model, we assume

\begin{enumerate}
\def\labelenumi{\arabic{enumi}.}
\tightlist
\item
  Strict exogeneity:
  \(E[u_{it} | \mathbf{X}_{i1}, \ldots, \mathbf{X}_{iT}, c_i] = 0\).
\item
  Orthogonality between \(c_i\) and \(\mathbf{X}_{it}\):
  \(E[c_i | \mathbf{X}_{i1}, \ldots, \mathbf{X}_{iT}] = 0\).
\item
  Full rank:
  \(rank(E[\underline{\mathbf{X}}'_i \Omega^{-1} \underline{\mathbf{X}}_i]) = K\).
\item
  \(E[\underline{u}_i\underline{u}'_i | \mathbf{X}_{i1}, \ldots, \mathbf{X}_{iT}, c_i] = \sigma_u^2 I_T\),
  and
  \(E[c_i^2 | \mathbf{X}_{i1}, \ldots, \mathbf{X}_{iT}] = \sigma_c^2\).
\end{enumerate}

Using the FGLS method through the introduction of \(\Sigma\), we can
obtain the FGLS-type RE estimator as follows: \[
  \hat{\beta}_{RE} = 
  \left( \frac{1}{n} \sum_{i=1}^n \underline{\mathbf{X}}'_i \hat{\Omega}^{-1} \underline{\mathbf{X}}_i \right)^{-1}
  \left( \frac{1}{n} \sum_{i=1}^n \underline{\mathbf{X}}'_i \hat{\Omega}^{-1} \underline{Y}_i \right),
\] where \[
  \hat{\Omega} = \hat{\sigma}_u^2 I_T + \hat{\sigma}_c^2 \iota \iota' =
  \begin{pmatrix}
    \hat{\sigma}_c^2 + \hat{\sigma}_u^2 & \hat{\sigma}_c^2 & \cdots & \hat{\sigma}_c^2 \\
    \hat{\sigma}_c^2 & \hat{\sigma}_c^2 + \hat{\sigma}_u^2 & \cdots & \hat{\sigma}_c^2 \\
    \vdots & \vdots & \cdots & \vdots \\
    \hat{\sigma}_c^2 & \hat{\sigma}_c^2 & \cdots & \hat{\sigma}_c^2 + \hat{\sigma}_u^2
  \end{pmatrix}.
\]

The estimator \(\hat{\sigma}_u^2\) and \(\hat{\sigma}_c^2\) can be
obtained by \[
  \hat{\sigma}_u^2 = \hat{\sigma}_v^2 - \hat{\sigma}_c^2,
\] \[
  \hat{\sigma}_v^2 = \frac{1}{nT - K} \sum_{i=1}^n \sum_{t=1}^T \hat{v}_{it}^2,
\] \[
  \hat{\sigma}_c^2 = 
  \frac{1}{nT(T-1)/2 - K} \sum_{i=1}^n \sum_{t=1}^{T-1} \sum_{s=t+1}^T \hat{v}_{it} \hat{v}_{is},
\] \[
  \hat{v}_{it} = Y_{it} - X_{it} \hat{\beta}_{OLS}.
\]

In the \texttt{R} programming, the \texttt{plm} function provides the
random effect model. However, the procedure is not the feasible GLS
method, but the OLS method on a dataset in which all variables are
subject to quasi-demeaning \footnote{The RE estimator by the
  quasi-demeaning method is simple. First, we calculate quasi-demeaned
  variables as in \(\tilde{Y}_{it} = Y_{it} - \theta \bar{Y}_i\) where
  \(\theta = 1 - (\sigma_u^2/(\sigma_u^2 + T \sigma_c^2))^{1/2}\). Using
  the matrix notations,
  \(\tilde{\underline{Y}}_i = \tilde{Q}_T \underline{Y}_i\) where
  \(\tilde{Q}_T\) is the quasi-demeaning matrix, which is given by
  \(\tilde{Q}_T = I_T - \theta \iota (\iota' \iota)^{-1} \iota\). After
  transforming all variables, we estimate
  \(\tilde{\underline{Y}}_i = \tilde{\underline{X}}_i \beta + \tilde{\underline{u}}_i\)
  by OLS method. The variance-covariance matrix is
  \(\hat{\sigma} (\sum_i \tilde{\underline{X}}_i \tilde{\underline{X}}'_i)^{-1}\)
  where
  \(\hat{\sigma} = \tilde{\underline{Y}}_i - \tilde{\underline{X}}_i \hat{\beta}\).
  See
  \url{http://ricardo.ecn.wfu.edu/~cottrell/gretl/random-effects.pdf} in
  detail.}. The two procedures generate the same RE estimator. Moreover,
the idiosyncratic error and the unobserved component are obtained by
differenct approach. To implement the RE model described above, we
compute manually.

\begin{Shaded}
\begin{Highlighting}[]
\CommentTok{\# Setup}
\NormalTok{X \textless{}{-}}\StringTok{ }\KeywordTok{model.matrix}\NormalTok{(model, dt)}
\NormalTok{y \textless{}{-}}\StringTok{ }\NormalTok{dt}\OperatorTok{$}\NormalTok{lwage}
\NormalTok{k \textless{}{-}}\StringTok{ }\KeywordTok{ncol}\NormalTok{(X)}
\NormalTok{N \textless{}{-}}\StringTok{ }\KeywordTok{length}\NormalTok{(}\KeywordTok{unique}\NormalTok{(dt}\OperatorTok{$}\NormalTok{id))}
\NormalTok{T \textless{}{-}}\StringTok{ }\KeywordTok{length}\NormalTok{(}\KeywordTok{unique}\NormalTok{(dt}\OperatorTok{$}\NormalTok{time))}

\CommentTok{\# estimator of Omega}
\NormalTok{pols \textless{}{-}}\StringTok{ }\KeywordTok{lm}\NormalTok{(model, dt)}
\NormalTok{vhat \textless{}{-}}\StringTok{ }\NormalTok{pols}\OperatorTok{$}\NormalTok{residuals}
\NormalTok{sigmav \textless{}{-}}\StringTok{ }\KeywordTok{sum}\NormalTok{(vhat}\OperatorTok{\^{}}\DecValTok{2}\NormalTok{)}\OperatorTok{/}\NormalTok{(N}\OperatorTok{*}\NormalTok{T }\OperatorTok{{-}}\StringTok{ }\NormalTok{k)}

\NormalTok{sumuc \textless{}{-}}\StringTok{ }\KeywordTok{matrix}\NormalTok{(}\DecValTok{0}\NormalTok{, }\DataTypeTok{nrow =}\NormalTok{ N, }\DataTypeTok{ncol =}\NormalTok{ T}\DecValTok{{-}1}\NormalTok{)}
\ControlFlowTok{for}\NormalTok{ (i }\ControlFlowTok{in} \DecValTok{1}\OperatorTok{:}\NormalTok{N) \{}
  \ControlFlowTok{for}\NormalTok{ (t }\ControlFlowTok{in} \DecValTok{1}\OperatorTok{:}\NormalTok{T}\DecValTok{{-}1}\NormalTok{) \{}
\NormalTok{    it \textless{}{-}}\StringTok{ }\KeywordTok{as.numeric}\NormalTok{(}\KeywordTok{rownames}\NormalTok{(dt))[dt}\OperatorTok{$}\NormalTok{id }\OperatorTok{==}\StringTok{ }\NormalTok{i }\OperatorTok{\&}\StringTok{ }\NormalTok{dt}\OperatorTok{$}\NormalTok{time }\OperatorTok{==}\StringTok{ }\NormalTok{t]}
\NormalTok{    is \textless{}{-}}\StringTok{ }\KeywordTok{as.numeric}\NormalTok{(}\KeywordTok{rownames}\NormalTok{(dt))[dt}\OperatorTok{$}\NormalTok{id }\OperatorTok{==}\StringTok{ }\NormalTok{i }\OperatorTok{\&}\StringTok{ }\NormalTok{dt}\OperatorTok{$}\NormalTok{time }\OperatorTok{\textgreater{}}\StringTok{ }\NormalTok{t]}
\NormalTok{    sumuc[i,t] \textless{}{-}}\StringTok{ }\NormalTok{vhat[it] }\OperatorTok{*}\StringTok{ }\KeywordTok{sum}\NormalTok{(vhat[is])}
\NormalTok{  \}}
\NormalTok{\}}
\NormalTok{sigmac \textless{}{-}}\StringTok{ }\KeywordTok{sum}\NormalTok{(}\KeywordTok{colSums}\NormalTok{(sumuc))}\OperatorTok{/}\NormalTok{((N}\OperatorTok{*}\NormalTok{T}\OperatorTok{*}\NormalTok{(T}\DecValTok{{-}1}\NormalTok{))}\OperatorTok{/}\DecValTok{2}\OperatorTok{{-}}\NormalTok{k)}
\NormalTok{sigmau \textless{}{-}}\StringTok{ }\NormalTok{sigmav }\OperatorTok{{-}}\StringTok{ }\NormalTok{sigmac}

\NormalTok{i \textless{}{-}}\StringTok{ }\KeywordTok{rep}\NormalTok{(}\DecValTok{1}\NormalTok{, T)}
\NormalTok{Omega \textless{}{-}}\StringTok{ }\NormalTok{sigmau }\OperatorTok{*}\StringTok{ }\KeywordTok{diag}\NormalTok{(T) }\OperatorTok{+}\StringTok{ }\NormalTok{sigmac }\OperatorTok{*}\StringTok{ }\NormalTok{i }\OperatorTok{\%*\%}\StringTok{ }\KeywordTok{t}\NormalTok{(i)}
\NormalTok{kroOmega \textless{}{-}}\StringTok{ }\KeywordTok{diag}\NormalTok{(N) }\OperatorTok{\%x\%}\StringTok{ }\KeywordTok{solve}\NormalTok{(Omega)}

\CommentTok{\# Random effect}
\NormalTok{bre \textless{}{-}}\StringTok{ }\KeywordTok{solve}\NormalTok{(}\KeywordTok{t}\NormalTok{(X) }\OperatorTok{\%*\%}\StringTok{ }\NormalTok{kroOmega }\OperatorTok{\%*\%}\StringTok{ }\NormalTok{X) }\OperatorTok{\%*\%}\StringTok{ }\KeywordTok{t}\NormalTok{(X) }\OperatorTok{\%*\%}\StringTok{ }\NormalTok{kroOmega }\OperatorTok{\%*\%}\StringTok{ }\NormalTok{y}
\end{Highlighting}
\end{Shaded}

A consistent estimator of asymptotic variance of the RE estimator is
given by \[
  \widehat{Asyvar}(\hat{\beta}_{RE}) = 
  \left( \underline{\mathbf{X}}'_i \hat{\Omega}^{-1} \underline{\mathbf{X}}_i \right)^{-1}.
\]

In the \texttt{R} programming, the \texttt{plm} function returns
standard errors calculated by variance-covariance matrix of OLS on a
quasi-demeaned data. To obtain the FGLS-type standard errors, we
compuate manually. The sample code is as follows:

\begin{Shaded}
\begin{Highlighting}[]
\NormalTok{vcovre \textless{}{-}}\StringTok{ }\KeywordTok{solve}\NormalTok{(}\KeywordTok{t}\NormalTok{(X) }\OperatorTok{\%*\%}\StringTok{ }\NormalTok{kroOmega }\OperatorTok{\%*\%}\StringTok{ }\NormalTok{X)}
\NormalTok{sere \textless{}{-}}\StringTok{ }\KeywordTok{sqrt}\NormalTok{(}\KeywordTok{diag}\NormalTok{(vcovre))}
\end{Highlighting}
\end{Shaded}

The result is shown in the fourth column in Table \ref{pdm}. The partial
effect of experience represents the percent change of wages. Thus, \[
  (\text{\% Change of Wage}) = 39.5 - 2 \cdot 0.6 \cdot \text{exp}.
\] For example, wages increase by 15.68\% at a mathematical mean of
labor experience (\texttt{exp}).

\begin{table}[t] \centering 
  \caption{Effect of Experience on Wages (Standard errors are in parentheses)} 
  \label{pdm} 
\begin{tabular}{@{\extracolsep{5pt}}lcccc} 
\\[-1.8ex]\hline 
\hline \\[-1.8ex] 
 & \multicolumn{4}{c}{\textit{Dependent variable:}} \\ 
\cline{2-5} 
\\[-1.8ex] & \multicolumn{4}{c}{lwage} \\ 
 & Pooled OLS & FGLS & Fixed Effect & Random Effect \\ 
\\[-1.8ex] & (1) & (2) & (3) & (4)\\ 
\hline \\[-1.8ex] 
 exp & 0.646 & 0.529 & 0.114 & 0.395 \\ 
  & (0.011) & (0.010) & (0.002) & (0.006) \\ 
  & & & & \\ 
 sqexp & $-$0.013 & $-$0.009 & $-$0.0004 & $-$0.006 \\ 
  & (0.0004) & (0.0004) & (0.0001) & (0.0002) \\ 
  & & & & \\ 
\hline \\[-1.8ex] 
Observations & 4,165 & 4,165 & 4,165 & 4,165 \\ 
\hline 
\hline \\[-1.8ex] 
\end{tabular} 
\end{table}

\hypertarget{hausman-test}{%
\subsection{Hausman Test}\label{hausman-test}}

The Hausman test provides empirical evidence on choosing between FE and
RE model. The null hypothesis of this test is \(\mathbf{X}_{it}\) and
\(c_i\) are independent. If we can reject the null hypothesis, then the
FE model is preferred. If we cannot reject the null hypothesis, then the
RE model should be used.

The test statistics is \[
  \hat{H} = 
  (\hat{\beta}_{RE} - \hat{\beta}_{FE})'
  \{ \widehat{Var}(\hat{\beta}_{RE}) - \widehat{Var}(\hat{\beta}_{FE}) \}^{-1}
  (\hat{\beta}_{RE} - \hat{\beta}_{FE}).
\] The limiting distribution of this test statistics is
\(\hat{H} \to \chi^2(K)\).

In the \texttt{R} programming, the manual computation is very easy.
Alternatively, the \texttt{phtest} function in the package \texttt{plm}
provides the Hausman test. To use the \texttt{phtest}, we need to
estimate the FE and RE model by the \texttt{plm} function.

\begin{Shaded}
\begin{Highlighting}[]
\NormalTok{delta \textless{}{-}}\StringTok{ }\NormalTok{bre }\OperatorTok{{-}}\StringTok{ }\NormalTok{bfe}
\NormalTok{diffv \textless{}{-}}\StringTok{ }\NormalTok{vcovre }\OperatorTok{{-}}\StringTok{ }\NormalTok{vcovfe}
\NormalTok{H \textless{}{-}}\StringTok{ }\KeywordTok{t}\NormalTok{(delta) }\OperatorTok{\%*\%}\StringTok{ }\KeywordTok{solve}\NormalTok{(diffv) }\OperatorTok{\%*\%}\StringTok{ }\NormalTok{delta}
\NormalTok{qtchi \textless{}{-}}\StringTok{ }\KeywordTok{qchisq}\NormalTok{(}\FloatTok{0.99}\NormalTok{, }\KeywordTok{nrow}\NormalTok{(delta))}
\KeywordTok{paste}\NormalTok{(}\StringTok{"The test statistics of Hausman test is "}\NormalTok{, }\KeywordTok{round}\NormalTok{(H, }\DecValTok{3}\NormalTok{))}
\end{Highlighting}
\end{Shaded}

\begin{verbatim}
## [1] "The test statistics of Hausman test is  3999.537"
\end{verbatim}

\begin{Shaded}
\begin{Highlighting}[]
\KeywordTok{paste}\NormalTok{(}\StringTok{"The 1\% quantile value of chi{-}sq dist is"}\NormalTok{, }\KeywordTok{round}\NormalTok{(qtchi, }\DecValTok{3}\NormalTok{))}
\end{Highlighting}
\end{Shaded}

\begin{verbatim}
## [1] "The 1% quantile value of chi-sq dist is 9.21"
\end{verbatim}

In this empirical application, we can reject the null hypothesis at 1\%
significance level. This implies that we should use the FE model in this
application beucase observed covariates and unobserved component are not
independent.

\end{document}
