% Options for packages loaded elsewhere
\PassOptionsToPackage{unicode}{hyperref}
\PassOptionsToPackage{hyphens}{url}
%
\documentclass[
  12pt,
]{article}
\usepackage{lmodern}
\usepackage{amsmath}
\usepackage{ifxetex,ifluatex}
\ifnum 0\ifxetex 1\fi\ifluatex 1\fi=0 % if pdftex
  \usepackage[T1]{fontenc}
  \usepackage[utf8]{inputenc}
  \usepackage{textcomp} % provide euro and other symbols
  \usepackage{amssymb}
\else % if luatex or xetex
  \usepackage{unicode-math}
  \defaultfontfeatures{Scale=MatchLowercase}
  \defaultfontfeatures[\rmfamily]{Ligatures=TeX,Scale=1}
\fi
% Use upquote if available, for straight quotes in verbatim environments
\IfFileExists{upquote.sty}{\usepackage{upquote}}{}
\IfFileExists{microtype.sty}{% use microtype if available
  \usepackage[]{microtype}
  \UseMicrotypeSet[protrusion]{basicmath} % disable protrusion for tt fonts
}{}
\makeatletter
\@ifundefined{KOMAClassName}{% if non-KOMA class
  \IfFileExists{parskip.sty}{%
    \usepackage{parskip}
  }{% else
    \setlength{\parindent}{0pt}
    \setlength{\parskip}{6pt plus 2pt minus 1pt}}
}{% if KOMA class
  \KOMAoptions{parskip=half}}
\makeatother
\usepackage{xcolor}
\IfFileExists{xurl.sty}{\usepackage{xurl}}{} % add URL line breaks if available
\IfFileExists{bookmark.sty}{\usepackage{bookmark}}{\usepackage{hyperref}}
\hypersetup{
  pdftitle={Econometrics II TA Session \#8},
  pdfauthor={Hiroki Kato},
  hidelinks,
  pdfcreator={LaTeX via pandoc}}
\urlstyle{same} % disable monospaced font for URLs
\usepackage[margin=1in]{geometry}
\usepackage{color}
\usepackage{fancyvrb}
\newcommand{\VerbBar}{|}
\newcommand{\VERB}{\Verb[commandchars=\\\{\}]}
\DefineVerbatimEnvironment{Highlighting}{Verbatim}{commandchars=\\\{\}}
% Add ',fontsize=\small' for more characters per line
\usepackage{framed}
\definecolor{shadecolor}{RGB}{248,248,248}
\newenvironment{Shaded}{\begin{snugshade}}{\end{snugshade}}
\newcommand{\AlertTok}[1]{\textcolor[rgb]{0.94,0.16,0.16}{#1}}
\newcommand{\AnnotationTok}[1]{\textcolor[rgb]{0.56,0.35,0.01}{\textbf{\textit{#1}}}}
\newcommand{\AttributeTok}[1]{\textcolor[rgb]{0.77,0.63,0.00}{#1}}
\newcommand{\BaseNTok}[1]{\textcolor[rgb]{0.00,0.00,0.81}{#1}}
\newcommand{\BuiltInTok}[1]{#1}
\newcommand{\CharTok}[1]{\textcolor[rgb]{0.31,0.60,0.02}{#1}}
\newcommand{\CommentTok}[1]{\textcolor[rgb]{0.56,0.35,0.01}{\textit{#1}}}
\newcommand{\CommentVarTok}[1]{\textcolor[rgb]{0.56,0.35,0.01}{\textbf{\textit{#1}}}}
\newcommand{\ConstantTok}[1]{\textcolor[rgb]{0.00,0.00,0.00}{#1}}
\newcommand{\ControlFlowTok}[1]{\textcolor[rgb]{0.13,0.29,0.53}{\textbf{#1}}}
\newcommand{\DataTypeTok}[1]{\textcolor[rgb]{0.13,0.29,0.53}{#1}}
\newcommand{\DecValTok}[1]{\textcolor[rgb]{0.00,0.00,0.81}{#1}}
\newcommand{\DocumentationTok}[1]{\textcolor[rgb]{0.56,0.35,0.01}{\textbf{\textit{#1}}}}
\newcommand{\ErrorTok}[1]{\textcolor[rgb]{0.64,0.00,0.00}{\textbf{#1}}}
\newcommand{\ExtensionTok}[1]{#1}
\newcommand{\FloatTok}[1]{\textcolor[rgb]{0.00,0.00,0.81}{#1}}
\newcommand{\FunctionTok}[1]{\textcolor[rgb]{0.00,0.00,0.00}{#1}}
\newcommand{\ImportTok}[1]{#1}
\newcommand{\InformationTok}[1]{\textcolor[rgb]{0.56,0.35,0.01}{\textbf{\textit{#1}}}}
\newcommand{\KeywordTok}[1]{\textcolor[rgb]{0.13,0.29,0.53}{\textbf{#1}}}
\newcommand{\NormalTok}[1]{#1}
\newcommand{\OperatorTok}[1]{\textcolor[rgb]{0.81,0.36,0.00}{\textbf{#1}}}
\newcommand{\OtherTok}[1]{\textcolor[rgb]{0.56,0.35,0.01}{#1}}
\newcommand{\PreprocessorTok}[1]{\textcolor[rgb]{0.56,0.35,0.01}{\textit{#1}}}
\newcommand{\RegionMarkerTok}[1]{#1}
\newcommand{\SpecialCharTok}[1]{\textcolor[rgb]{0.00,0.00,0.00}{#1}}
\newcommand{\SpecialStringTok}[1]{\textcolor[rgb]{0.31,0.60,0.02}{#1}}
\newcommand{\StringTok}[1]{\textcolor[rgb]{0.31,0.60,0.02}{#1}}
\newcommand{\VariableTok}[1]{\textcolor[rgb]{0.00,0.00,0.00}{#1}}
\newcommand{\VerbatimStringTok}[1]{\textcolor[rgb]{0.31,0.60,0.02}{#1}}
\newcommand{\WarningTok}[1]{\textcolor[rgb]{0.56,0.35,0.01}{\textbf{\textit{#1}}}}
\usepackage{graphicx}
\makeatletter
\def\maxwidth{\ifdim\Gin@nat@width>\linewidth\linewidth\else\Gin@nat@width\fi}
\def\maxheight{\ifdim\Gin@nat@height>\textheight\textheight\else\Gin@nat@height\fi}
\makeatother
% Scale images if necessary, so that they will not overflow the page
% margins by default, and it is still possible to overwrite the defaults
% using explicit options in \includegraphics[width, height, ...]{}
\setkeys{Gin}{width=\maxwidth,height=\maxheight,keepaspectratio}
% Set default figure placement to htbp
\makeatletter
\def\fps@figure{htbp}
\makeatother
\setlength{\emergencystretch}{3em} % prevent overfull lines
\providecommand{\tightlist}{%
  \setlength{\itemsep}{0pt}\setlength{\parskip}{0pt}}
\setcounter{secnumdepth}{5}
\usepackage{zxjatype}
\setCJKmainfont[BoldFont = IPAゴシック]{IPA明朝}
\setCJKsansfont{IPAゴシック}
\setCJKmonofont{IPAゴシック}
\parindent = 1em
\newcommand{\argmax}{\mathop{\rm arg~max}\limits}
\newcommand{\argmin}{\mathop{\rm arg~min}\limits}
\DeclareMathOperator*{\plim}{plim}
\usepackage{xcolor}
\ifluatex
  \usepackage{selnolig}  % disable illegal ligatures
\fi

\title{Econometrics II TA Session \#8}
\author{Hiroki Kato}
\date{}

\begin{document}
\maketitle

\hypertarget{empirical-application-of-panel-data-model-earnings-equation}{%
\section{Empirical Application of Panel Data Model: Earnings
Equation}\label{empirical-application-of-panel-data-model-earnings-equation}}

\hypertarget{backgruond}{%
\subsection{Backgruond}\label{backgruond}}

A researcher wants to estimate the effect of full-time work experience
on wages. He uses a \emph{balanced} panel of 595 individuals from 1976
to 1982, taken from the Panel Study of Income Dynamics (PSID). The
\emph{balanced} panel data means that we can observe all individuals
every year.

\begin{Shaded}
\begin{Highlighting}[]
\NormalTok{dt }\OtherTok{\textless{}{-}} \FunctionTok{read.csv}\NormalTok{(}\StringTok{"./data/wages.csv"}\NormalTok{)}
\FunctionTok{head}\NormalTok{(dt, }\DecValTok{14}\NormalTok{)}
\end{Highlighting}
\end{Shaded}

\begin{verbatim}
##    exp wks bluecol ind south smsa married  sex union ed black   lwage id time
## 1    3  32      no   0   yes   no     yes male    no  9    no 5.56068  1    1
## 2    4  43      no   0   yes   no     yes male    no  9    no 5.72031  1    2
## 3    5  40      no   0   yes   no     yes male    no  9    no 5.99645  1    3
## 4    6  39      no   0   yes   no     yes male    no  9    no 5.99645  1    4
## 5    7  42      no   1   yes   no     yes male    no  9    no 6.06146  1    5
## 6    8  35      no   1   yes   no     yes male    no  9    no 6.17379  1    6
## 7    9  32      no   1   yes   no     yes male    no  9    no 6.24417  1    7
## 8   30  34     yes   0    no   no     yes male    no 11    no 6.16331  2    1
## 9   31  27     yes   0    no   no     yes male    no 11    no 6.21461  2    2
## 10  32  33     yes   1    no   no     yes male   yes 11    no 6.26340  2    3
## 11  33  30     yes   1    no   no     yes male    no 11    no 6.54391  2    4
## 12  34  30     yes   1    no   no     yes male    no 11    no 6.69703  2    5
## 13  35  37     yes   1    no   no     yes male    no 11    no 6.79122  2    6
## 14  36  30     yes   1    no   no     yes male    no 11    no 6.81564  2    7
\end{verbatim}

The variable \texttt{id} and \texttt{time} indicate individual and time
indexs. We use these two variables to apply panel data models.
Additionally, we use the following variables:

\begin{itemize}
\tightlist
\item
  \texttt{exp}: years of full-time work experience
\item
  \texttt{sex}: an indicator of gender
\item
  \texttt{ed}: years of education
\item
  \texttt{lwage}: logarithm of wage
\end{itemize}

\begin{Shaded}
\begin{Highlighting}[]
\NormalTok{dt }\OtherTok{\textless{}{-}}\NormalTok{ dt[,}\FunctionTok{c}\NormalTok{(}\StringTok{"id"}\NormalTok{, }\StringTok{"time"}\NormalTok{, }\StringTok{"exp"}\NormalTok{, }\StringTok{"sex"}\NormalTok{, }\StringTok{"ed"}\NormalTok{, }\StringTok{"lwage"}\NormalTok{)]}
\FunctionTok{summary}\NormalTok{(dt)}
\end{Highlighting}
\end{Shaded}

\begin{verbatim}
##        id           time        exp            sex             ed       
##  Min.   :  1   Min.   :1   Min.   : 1.00   female: 469   Min.   : 4.00  
##  1st Qu.:149   1st Qu.:2   1st Qu.:11.00   male  :3696   1st Qu.:12.00  
##  Median :298   Median :4   Median :18.00                 Median :12.00  
##  Mean   :298   Mean   :4   Mean   :19.85                 Mean   :12.85  
##  3rd Qu.:447   3rd Qu.:6   3rd Qu.:29.00                 3rd Qu.:16.00  
##  Max.   :595   Max.   :7   Max.   :51.00                 Max.   :17.00  
##      lwage      
##  Min.   :4.605  
##  1st Qu.:6.395  
##  Median :6.685  
##  Mean   :6.676  
##  3rd Qu.:6.953  
##  Max.   :8.537
\end{verbatim}

\end{document}
