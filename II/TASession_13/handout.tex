% Options for packages loaded elsewhere
\PassOptionsToPackage{unicode}{hyperref}
\PassOptionsToPackage{hyphens}{url}
%
\documentclass[
  12pt,
]{article}
\usepackage{lmodern}
\usepackage{amssymb,amsmath}
\usepackage{ifxetex,ifluatex}
\ifnum 0\ifxetex 1\fi\ifluatex 1\fi=0 % if pdftex
  \usepackage[T1]{fontenc}
  \usepackage[utf8]{inputenc}
  \usepackage{textcomp} % provide euro and other symbols
\else % if luatex or xetex
  \usepackage{unicode-math}
  \defaultfontfeatures{Scale=MatchLowercase}
  \defaultfontfeatures[\rmfamily]{Ligatures=TeX,Scale=1}
\fi
% Use upquote if available, for straight quotes in verbatim environments
\IfFileExists{upquote.sty}{\usepackage{upquote}}{}
\IfFileExists{microtype.sty}{% use microtype if available
  \usepackage[]{microtype}
  \UseMicrotypeSet[protrusion]{basicmath} % disable protrusion for tt fonts
}{}
\makeatletter
\@ifundefined{KOMAClassName}{% if non-KOMA class
  \IfFileExists{parskip.sty}{%
    \usepackage{parskip}
  }{% else
    \setlength{\parindent}{0pt}
    \setlength{\parskip}{6pt plus 2pt minus 1pt}}
}{% if KOMA class
  \KOMAoptions{parskip=half}}
\makeatother
\usepackage{xcolor}
\IfFileExists{xurl.sty}{\usepackage{xurl}}{} % add URL line breaks if available
\IfFileExists{bookmark.sty}{\usepackage{bookmark}}{\usepackage{hyperref}}
\hypersetup{
  pdftitle={Econometrics II TA Session \#13},
  pdfauthor={Hiroki Kato},
  hidelinks,
  pdfcreator={LaTeX via pandoc}}
\urlstyle{same} % disable monospaced font for URLs
\usepackage[margin=1in]{geometry}
\usepackage{color}
\usepackage{fancyvrb}
\newcommand{\VerbBar}{|}
\newcommand{\VERB}{\Verb[commandchars=\\\{\}]}
\DefineVerbatimEnvironment{Highlighting}{Verbatim}{commandchars=\\\{\}}
% Add ',fontsize=\small' for more characters per line
\usepackage{framed}
\definecolor{shadecolor}{RGB}{248,248,248}
\newenvironment{Shaded}{\begin{snugshade}}{\end{snugshade}}
\newcommand{\AlertTok}[1]{\textcolor[rgb]{0.94,0.16,0.16}{#1}}
\newcommand{\AnnotationTok}[1]{\textcolor[rgb]{0.56,0.35,0.01}{\textbf{\textit{#1}}}}
\newcommand{\AttributeTok}[1]{\textcolor[rgb]{0.77,0.63,0.00}{#1}}
\newcommand{\BaseNTok}[1]{\textcolor[rgb]{0.00,0.00,0.81}{#1}}
\newcommand{\BuiltInTok}[1]{#1}
\newcommand{\CharTok}[1]{\textcolor[rgb]{0.31,0.60,0.02}{#1}}
\newcommand{\CommentTok}[1]{\textcolor[rgb]{0.56,0.35,0.01}{\textit{#1}}}
\newcommand{\CommentVarTok}[1]{\textcolor[rgb]{0.56,0.35,0.01}{\textbf{\textit{#1}}}}
\newcommand{\ConstantTok}[1]{\textcolor[rgb]{0.00,0.00,0.00}{#1}}
\newcommand{\ControlFlowTok}[1]{\textcolor[rgb]{0.13,0.29,0.53}{\textbf{#1}}}
\newcommand{\DataTypeTok}[1]{\textcolor[rgb]{0.13,0.29,0.53}{#1}}
\newcommand{\DecValTok}[1]{\textcolor[rgb]{0.00,0.00,0.81}{#1}}
\newcommand{\DocumentationTok}[1]{\textcolor[rgb]{0.56,0.35,0.01}{\textbf{\textit{#1}}}}
\newcommand{\ErrorTok}[1]{\textcolor[rgb]{0.64,0.00,0.00}{\textbf{#1}}}
\newcommand{\ExtensionTok}[1]{#1}
\newcommand{\FloatTok}[1]{\textcolor[rgb]{0.00,0.00,0.81}{#1}}
\newcommand{\FunctionTok}[1]{\textcolor[rgb]{0.00,0.00,0.00}{#1}}
\newcommand{\ImportTok}[1]{#1}
\newcommand{\InformationTok}[1]{\textcolor[rgb]{0.56,0.35,0.01}{\textbf{\textit{#1}}}}
\newcommand{\KeywordTok}[1]{\textcolor[rgb]{0.13,0.29,0.53}{\textbf{#1}}}
\newcommand{\NormalTok}[1]{#1}
\newcommand{\OperatorTok}[1]{\textcolor[rgb]{0.81,0.36,0.00}{\textbf{#1}}}
\newcommand{\OtherTok}[1]{\textcolor[rgb]{0.56,0.35,0.01}{#1}}
\newcommand{\PreprocessorTok}[1]{\textcolor[rgb]{0.56,0.35,0.01}{\textit{#1}}}
\newcommand{\RegionMarkerTok}[1]{#1}
\newcommand{\SpecialCharTok}[1]{\textcolor[rgb]{0.00,0.00,0.00}{#1}}
\newcommand{\SpecialStringTok}[1]{\textcolor[rgb]{0.31,0.60,0.02}{#1}}
\newcommand{\StringTok}[1]{\textcolor[rgb]{0.31,0.60,0.02}{#1}}
\newcommand{\VariableTok}[1]{\textcolor[rgb]{0.00,0.00,0.00}{#1}}
\newcommand{\VerbatimStringTok}[1]{\textcolor[rgb]{0.31,0.60,0.02}{#1}}
\newcommand{\WarningTok}[1]{\textcolor[rgb]{0.56,0.35,0.01}{\textbf{\textit{#1}}}}
\usepackage{graphicx}
\makeatletter
\def\maxwidth{\ifdim\Gin@nat@width>\linewidth\linewidth\else\Gin@nat@width\fi}
\def\maxheight{\ifdim\Gin@nat@height>\textheight\textheight\else\Gin@nat@height\fi}
\makeatother
% Scale images if necessary, so that they will not overflow the page
% margins by default, and it is still possible to overwrite the defaults
% using explicit options in \includegraphics[width, height, ...]{}
\setkeys{Gin}{width=\maxwidth,height=\maxheight,keepaspectratio}
% Set default figure placement to htbp
\makeatletter
\def\fps@figure{htbp}
\makeatother
\setlength{\emergencystretch}{3em} % prevent overfull lines
\providecommand{\tightlist}{%
  \setlength{\itemsep}{0pt}\setlength{\parskip}{0pt}}
\setcounter{secnumdepth}{5}
\usepackage{zxjatype}
\setCJKmainfont[BoldFont = IPAゴシック]{IPA明朝}
\setCJKsansfont{IPAゴシック}
\setCJKmonofont{IPAゴシック}
\parindent = 1em
\newcommand{\argmax}{\mathop{\rm arg~max}\limits}
\newcommand{\argmin}{\mathop{\rm arg~min}\limits}
\DeclareMathOperator*{\plim}{plim}
\usepackage{xcolor}
\ifluatex
  \usepackage{selnolig}  % disable illegal ligatures
\fi

\title{Econometrics II TA Session \#13}
\author{Hiroki Kato}
\date{}

\begin{document}
\maketitle

\hypertarget{empirical-application-of-time-series-model-nikkei-225}{%
\section{Empirical Application of Time Series Model: Nikkei
225}\label{empirical-application-of-time-series-model-nikkei-225}}

\hypertarget{background-and-data}{%
\subsection{Background and Data}\label{background-and-data}}

The ``Nikkei225'' is a stock price index published by Nihon Keizai
Shimbun (hereafter, NIKKEI). NIKKEI calculates this price index based on
225 high liquid brands listed with first section of the Tokyo Stock
Exchange. We use daily data of the Nikkei 225 index taken from the yahoo
finance
(\url{https://stocks.finance.yahoo.co.jp/stocks/detail/?code=998407.O}).
The time length is from January 4th 2020 to January 22 2021. We have 498
observations. We call a csv data which recodes the Nikkei 225 dairy
index, using the \texttt{read.csv} function in \texttt{R}. Since
\texttt{R} recognize a time variable (e.g., \texttt{2021/01/22}) as a
character string, we need to define a time variable, using
\texttt{as.Date()} function. The data structure is as follows:

\begin{Shaded}
\begin{Highlighting}[]
\NormalTok{dt \textless{}{-}}\StringTok{ }\KeywordTok{read.csv}\NormalTok{(}\StringTok{"data/nikkei225.csv"}\NormalTok{, }\DataTypeTok{stringsAsFactor =} \OtherTok{FALSE}\NormalTok{)}
\NormalTok{dt}\OperatorTok{$}\NormalTok{date \textless{}{-}}\StringTok{ }\KeywordTok{as.Date}\NormalTok{(dt}\OperatorTok{$}\NormalTok{date, }\DataTypeTok{format =} \StringTok{"\%Y/\%m/\%d"}\NormalTok{)}
\KeywordTok{head}\NormalTok{(dt)}
\end{Highlighting}
\end{Shaded}

\begin{verbatim}
##         date open_price high_price low_price close_price
## 1 2021-01-22   28580.20   28698.18  28527.16    28631.45
## 2 2021-01-21   28710.41   28846.15  28677.61    28756.86
## 3 2021-01-20   28798.74   28801.19  28402.11    28523.26
## 4 2021-01-19   28405.49   28720.91  28373.34    28633.46
## 5 2021-01-18   28238.68   28349.97  28111.54    28242.21
## 6 2021-01-15   28777.47   28820.50  28477.03    28519.18
\end{verbatim}

There are five variables:

\begin{itemize}
\tightlist
\item
  \texttt{date}: date variable
\item
  \texttt{open\_price}: open price in day \(t\)
\item
  \texttt{high\_price}: high price in day \(t\)
\item
  \texttt{low\_price}: low price in day \(t\)
\item
  \texttt{close\_price}: close price in day \(t\)
\end{itemize}

Mainly, we use the \texttt{date} and \texttt{close\_price}.

\begin{figure}[h]

{\centering \includegraphics[width=0.9\linewidth]{C:/Users/katoo/Desktop/2020EconometricsTA/II/TAsession_13/handout_files/figure-latex/TimeSeriesPlot-1} 

}

\caption{Time Series Data of Nikkei 225}\label{fig:TimeSeriesPlot}
\end{figure}

Figure 1 shows the time-series of close price of the Nikkei225. We
summarize some features of this data as follows:

\begin{itemize}
\tightlist
\item
  After the COVID-19 occured in Japan and Chaina, the Nikkei225 has
  drastically decreased.
\item
  During the first declaration of a state of emergency in Japan, the
  Nikkei225's performance has been a V-shaped recovery.
\item
  The Nikkei225 has sharply increased immediaterly before and after the
  U.S. presidential election.
\end{itemize}

Some may wonder if the negative shock of COVID-19 reflects the
Nikkei225. To discuss it, we need to consider two potential concerns.

\begin{enumerate}
\def\labelenumi{\arabic{enumi}.}
\tightlist
\item
  unlisted companies (such as restaurant business) suffers heavily from
  the negative shock of COVID-19.
\item
  the Nikkei225 does not represent a variation of price index of 225
  brands. In principle, the Nikkei225 is a mathematical mean of stock
  price of 225 brands. The the stock prices of top five brands which
  contribute to the Nikkei225 have increased at 70\%. On the other hand,
  the stock price of other brands have decreased at 5\% \footnote{See
    \url{https://news.yahoo.co.jp/articles/f63a4627b298857a62ac329b1ed41a88c2721bd4}.}.
\end{enumerate}

Anyway, we estimate the time series model, using this dataset.

\end{document}
