% Options for packages loaded elsewhere
\PassOptionsToPackage{unicode}{hyperref}
\PassOptionsToPackage{hyphens}{url}
%
\documentclass[
  12pt,
]{article}
\usepackage{lmodern}
\usepackage{amssymb,amsmath}
\usepackage{ifxetex,ifluatex}
\ifnum 0\ifxetex 1\fi\ifluatex 1\fi=0 % if pdftex
  \usepackage[T1]{fontenc}
  \usepackage[utf8]{inputenc}
  \usepackage{textcomp} % provide euro and other symbols
\else % if luatex or xetex
  \usepackage{unicode-math}
  \defaultfontfeatures{Scale=MatchLowercase}
  \defaultfontfeatures[\rmfamily]{Ligatures=TeX,Scale=1}
\fi
% Use upquote if available, for straight quotes in verbatim environments
\IfFileExists{upquote.sty}{\usepackage{upquote}}{}
\IfFileExists{microtype.sty}{% use microtype if available
  \usepackage[]{microtype}
  \UseMicrotypeSet[protrusion]{basicmath} % disable protrusion for tt fonts
}{}
\makeatletter
\@ifundefined{KOMAClassName}{% if non-KOMA class
  \IfFileExists{parskip.sty}{%
    \usepackage{parskip}
  }{% else
    \setlength{\parindent}{0pt}
    \setlength{\parskip}{6pt plus 2pt minus 1pt}}
}{% if KOMA class
  \KOMAoptions{parskip=half}}
\makeatother
\usepackage{xcolor}
\IfFileExists{xurl.sty}{\usepackage{xurl}}{} % add URL line breaks if available
\IfFileExists{bookmark.sty}{\usepackage{bookmark}}{\usepackage{hyperref}}
\hypersetup{
  pdftitle={Econometrics II TA Session \#3},
  pdfauthor={Hiroki Kato},
  hidelinks,
  pdfcreator={LaTeX via pandoc}}
\urlstyle{same} % disable monospaced font for URLs
\usepackage[margin=1in]{geometry}
\usepackage{color}
\usepackage{fancyvrb}
\newcommand{\VerbBar}{|}
\newcommand{\VERB}{\Verb[commandchars=\\\{\}]}
\DefineVerbatimEnvironment{Highlighting}{Verbatim}{commandchars=\\\{\}}
% Add ',fontsize=\small' for more characters per line
\usepackage{framed}
\definecolor{shadecolor}{RGB}{248,248,248}
\newenvironment{Shaded}{\begin{snugshade}}{\end{snugshade}}
\newcommand{\AlertTok}[1]{\textcolor[rgb]{0.94,0.16,0.16}{#1}}
\newcommand{\AnnotationTok}[1]{\textcolor[rgb]{0.56,0.35,0.01}{\textbf{\textit{#1}}}}
\newcommand{\AttributeTok}[1]{\textcolor[rgb]{0.77,0.63,0.00}{#1}}
\newcommand{\BaseNTok}[1]{\textcolor[rgb]{0.00,0.00,0.81}{#1}}
\newcommand{\BuiltInTok}[1]{#1}
\newcommand{\CharTok}[1]{\textcolor[rgb]{0.31,0.60,0.02}{#1}}
\newcommand{\CommentTok}[1]{\textcolor[rgb]{0.56,0.35,0.01}{\textit{#1}}}
\newcommand{\CommentVarTok}[1]{\textcolor[rgb]{0.56,0.35,0.01}{\textbf{\textit{#1}}}}
\newcommand{\ConstantTok}[1]{\textcolor[rgb]{0.00,0.00,0.00}{#1}}
\newcommand{\ControlFlowTok}[1]{\textcolor[rgb]{0.13,0.29,0.53}{\textbf{#1}}}
\newcommand{\DataTypeTok}[1]{\textcolor[rgb]{0.13,0.29,0.53}{#1}}
\newcommand{\DecValTok}[1]{\textcolor[rgb]{0.00,0.00,0.81}{#1}}
\newcommand{\DocumentationTok}[1]{\textcolor[rgb]{0.56,0.35,0.01}{\textbf{\textit{#1}}}}
\newcommand{\ErrorTok}[1]{\textcolor[rgb]{0.64,0.00,0.00}{\textbf{#1}}}
\newcommand{\ExtensionTok}[1]{#1}
\newcommand{\FloatTok}[1]{\textcolor[rgb]{0.00,0.00,0.81}{#1}}
\newcommand{\FunctionTok}[1]{\textcolor[rgb]{0.00,0.00,0.00}{#1}}
\newcommand{\ImportTok}[1]{#1}
\newcommand{\InformationTok}[1]{\textcolor[rgb]{0.56,0.35,0.01}{\textbf{\textit{#1}}}}
\newcommand{\KeywordTok}[1]{\textcolor[rgb]{0.13,0.29,0.53}{\textbf{#1}}}
\newcommand{\NormalTok}[1]{#1}
\newcommand{\OperatorTok}[1]{\textcolor[rgb]{0.81,0.36,0.00}{\textbf{#1}}}
\newcommand{\OtherTok}[1]{\textcolor[rgb]{0.56,0.35,0.01}{#1}}
\newcommand{\PreprocessorTok}[1]{\textcolor[rgb]{0.56,0.35,0.01}{\textit{#1}}}
\newcommand{\RegionMarkerTok}[1]{#1}
\newcommand{\SpecialCharTok}[1]{\textcolor[rgb]{0.00,0.00,0.00}{#1}}
\newcommand{\SpecialStringTok}[1]{\textcolor[rgb]{0.31,0.60,0.02}{#1}}
\newcommand{\StringTok}[1]{\textcolor[rgb]{0.31,0.60,0.02}{#1}}
\newcommand{\VariableTok}[1]{\textcolor[rgb]{0.00,0.00,0.00}{#1}}
\newcommand{\VerbatimStringTok}[1]{\textcolor[rgb]{0.31,0.60,0.02}{#1}}
\newcommand{\WarningTok}[1]{\textcolor[rgb]{0.56,0.35,0.01}{\textbf{\textit{#1}}}}
\usepackage{graphicx}
\makeatletter
\def\maxwidth{\ifdim\Gin@nat@width>\linewidth\linewidth\else\Gin@nat@width\fi}
\def\maxheight{\ifdim\Gin@nat@height>\textheight\textheight\else\Gin@nat@height\fi}
\makeatother
% Scale images if necessary, so that they will not overflow the page
% margins by default, and it is still possible to overwrite the defaults
% using explicit options in \includegraphics[width, height, ...]{}
\setkeys{Gin}{width=\maxwidth,height=\maxheight,keepaspectratio}
% Set default figure placement to htbp
\makeatletter
\def\fps@figure{htbp}
\makeatother
\setlength{\emergencystretch}{3em} % prevent overfull lines
\providecommand{\tightlist}{%
  \setlength{\itemsep}{0pt}\setlength{\parskip}{0pt}}
\setcounter{secnumdepth}{5}
\usepackage{zxjatype}
\setCJKmainfont[BoldFont = IPAゴシック]{IPA明朝}
\setCJKsansfont{IPAゴシック}
\setCJKmonofont{IPAゴシック}
\parindent = 1em
\newcommand{\argmax}{\mathop{\rm arg~max}\limits}
\newcommand{\argmin}{\mathop{\rm arg~min}\limits}
\DeclareMathOperator*{\plim}{plim}
\usepackage{xcolor}
\ifluatex
  \usepackage{selnolig}  % disable illegal ligatures
\fi

\title{Econometrics II TA Session \#3}
\author{Hiroki Kato}
\date{}

\begin{document}
\maketitle

\hypertarget{empirical-application-of-binary-model-titanic-survivors}{%
\section{Empirical Application of Binary Model: Titanic
Survivors}\label{empirical-application-of-binary-model-titanic-survivors}}

\textbf{Brief Background}. ``Women and children first'' is a behavioral
norm, which women and children are saved first in a life-threatening
situation. This code was made famous by the sinking of the Titanic in
1912. An empirical application investigates characteristics of survivors
of Titanic to answer whether crews obeyed the code or not.

\noindent \textbf{Data}. We use an open data about Titanic survivors
\footnote{data source: \url{http://biostat.mc.vanderbilt.edu/DataSets}.}.
Although this dataset contains many variables, we use only four
variables: \texttt{survived}, \texttt{age}, \texttt{fare}, and
\texttt{sex}. We summarize descriptions of variables as follows:

\begin{itemize}
\tightlist
\item
  \texttt{survived}: a binary variable taking 1 if a passenger survived.
\item
  \texttt{age}: a continuous variable representing passeger's age.
\item
  \texttt{fare}: a continuous variable representing how much passeger
  paid.
\item
  \texttt{sex}: a string variable representing passenger's sex.
\end{itemize}

Using \texttt{sex}, we will make a binary variable, called
\texttt{female}, taking 1 if passeger is female. Intead of \texttt{sex},
we use \texttt{female} variable in regression.

\begin{Shaded}
\begin{Highlighting}[]
\NormalTok{dt \textless{}{-}}\StringTok{ }\KeywordTok{read.csv}\NormalTok{(}
  \DataTypeTok{file =} \StringTok{"./data/titanic.csv"}\NormalTok{, }
  \DataTypeTok{header =} \OtherTok{TRUE}\NormalTok{,  }\DataTypeTok{sep =} \StringTok{","}\NormalTok{, }\DataTypeTok{row.names =} \OtherTok{NULL}\NormalTok{,  }\DataTypeTok{stringsAsFactors =} \OtherTok{FALSE}\NormalTok{)}

\NormalTok{dt}\OperatorTok{$}\NormalTok{female \textless{}{-}}\StringTok{ }\KeywordTok{ifelse}\NormalTok{(dt}\OperatorTok{$}\NormalTok{sex }\OperatorTok{==}\StringTok{ "female"}\NormalTok{, }\DecValTok{1}\NormalTok{, }\DecValTok{0}\NormalTok{)}
\NormalTok{dt \textless{}{-}}\StringTok{ }\KeywordTok{subset}\NormalTok{(dt, }\OperatorTok{!}\KeywordTok{is.na}\NormalTok{(survived)}\OperatorTok{\&!}\KeywordTok{is.na}\NormalTok{(age)}\OperatorTok{\&!}\KeywordTok{is.na}\NormalTok{(fare)}\OperatorTok{\&!}\KeywordTok{is.na}\NormalTok{(female))}

\NormalTok{dt \textless{}{-}}\StringTok{ }\NormalTok{dt[,}\KeywordTok{c}\NormalTok{(}\StringTok{"survived"}\NormalTok{, }\StringTok{"age"}\NormalTok{, }\StringTok{"fare"}\NormalTok{, }\StringTok{"female"}\NormalTok{)]}
\KeywordTok{head}\NormalTok{(dt)}
\end{Highlighting}
\end{Shaded}

\begin{verbatim}
##   survived   age     fare female
## 1        1 29.00 211.3375      1
## 2        1  0.92 151.5500      0
## 3        0  2.00 151.5500      1
## 4        0 30.00 151.5500      0
## 5        0 25.00 151.5500      1
## 6        1 48.00  26.5500      0
\end{verbatim}

\noindent \textbf{Model}. In a binary model, a dependent (outcome)
variable \(y_i\) takes only two values, i.e., \(y_i \in \{0, 1\}\). A
binary variable is sometimes called a \emph{dummy} variable. In this
application, the outcome variable is \texttt{survived}. Explanatory
variables are \texttt{female}, \texttt{age}, and \texttt{fare}. The
regression function is \begin{equation*}
  \begin{split}
    &E[survived | female, age, fare] \\
    =& \mathbb{P}[survived = 1 | female, age, fare]
    = G(\beta_0 + \beta_1 female + \beta_2 age + \beta_3 fare).
  \end{split}
\end{equation*} The function \(G(\cdot)\) is arbitrary function. In
practice, we often use following three specifications:

\begin{itemize}
\tightlist
\item
  Linear probability model (LPM):
  \(G(\mathbf{x}_i \beta) = \mathbf{x}_i \beta\).
\item
  Probit model: \(G(\mathbf{x}_i \beta) = \Phi(\mathbf{x}_i \beta)\)
  where \(\Phi(\cdot)\) is the standard Gaussian cumulative function.
\item
  Logit model:
  \(G(\mathbf{x}_i \beta) = 1/(1 + \exp(-\mathbf{x}_i \beta))\).
\end{itemize}

\hypertarget{linear-probability-model}{%
\subsection{Linear Probability Model}\label{linear-probability-model}}

The linear probability model specifys that \(G(a)\) is linear in \(a\),
that is, \begin{equation*}
  \mathbb{P}[survived = 1 | female, age, fare]
  = \beta_0 + \beta_1 female + \beta_2 age + \beta_3 fare.
\end{equation*} This model can be estimated using the OLS method. In
\texttt{R}, we can use the OLS method, running \texttt{lm()} function.

\begin{Shaded}
\begin{Highlighting}[]
\NormalTok{model \textless{}{-}}\StringTok{ }\NormalTok{survived }\OperatorTok{\textasciitilde{}}\StringTok{ }\KeywordTok{factor}\NormalTok{(female) }\OperatorTok{+}\StringTok{ }\NormalTok{age }\OperatorTok{+}\StringTok{ }\NormalTok{fare}
\NormalTok{LPM \textless{}{-}}\StringTok{ }\KeywordTok{lm}\NormalTok{(model, }\DataTypeTok{data =}\NormalTok{ dt)}
\end{Highlighting}
\end{Shaded}

However, \texttt{lm()} function does not deal with heteroskedasticity
problem. To resolve it, we need to claculate heteroskedasticity-robust
standard errors using the White method. \begin{equation*}
  \hat{V}(\hat{\beta}) =
  \left( \frac{1}{n} \sum_i \mathbf{x}'_i \mathbf{x}_i  \right)^{-1}
  \left( \frac{1}{n} \sum_i \hat{u}_i^2 \mathbf{x}'_i \mathbf{x}_i \right)
  \left( \frac{1}{n} \sum_i \mathbf{x}'_i \mathbf{x}_i \right)^{-1}
\end{equation*}

\begin{Shaded}
\begin{Highlighting}[]
\CommentTok{\# heteroskedasticity{-}robust standard errors}
\NormalTok{dt}\OperatorTok{$}\StringTok{"(Intercept)"}\NormalTok{ \textless{}{-}}\StringTok{ }\DecValTok{1}
\NormalTok{X \textless{}{-}}\StringTok{ }\KeywordTok{as.matrix}\NormalTok{(dt[,}\KeywordTok{c}\NormalTok{(}\StringTok{"(Intercept)"}\NormalTok{, }\StringTok{"female"}\NormalTok{, }\StringTok{"age"}\NormalTok{, }\StringTok{"fare"}\NormalTok{)])}
\NormalTok{u \textless{}{-}}\StringTok{ }\KeywordTok{diag}\NormalTok{(LPM}\OperatorTok{$}\NormalTok{residuals}\OperatorTok{\^{}}\DecValTok{2}\NormalTok{)}

\NormalTok{XX \textless{}{-}}\StringTok{ }\KeywordTok{t}\NormalTok{(X) }\OperatorTok{\%*\%}\StringTok{ }\NormalTok{X}
\NormalTok{avgXX \textless{}{-}}\StringTok{ }\NormalTok{XX }\OperatorTok{*}\StringTok{ }\KeywordTok{nrow}\NormalTok{(X)}\OperatorTok{\^{}}\NormalTok{\{}\OperatorTok{{-}}\DecValTok{1}\NormalTok{\}}
\NormalTok{inv\_avgXX \textless{}{-}}\StringTok{ }\KeywordTok{solve}\NormalTok{(avgXX)}

\NormalTok{uXX \textless{}{-}}\StringTok{ }\KeywordTok{t}\NormalTok{(X) }\OperatorTok{\%*\%}\StringTok{ }\NormalTok{u }\OperatorTok{\%*\%}\StringTok{ }\NormalTok{X}
\NormalTok{avguXX \textless{}{-}}\StringTok{ }\NormalTok{uXX }\OperatorTok{*}\StringTok{ }\KeywordTok{nrow}\NormalTok{(X)}\OperatorTok{\^{}}\NormalTok{\{}\OperatorTok{{-}}\DecValTok{1}\NormalTok{\} }

\NormalTok{vcov\_b \textless{}{-}}\StringTok{ }\NormalTok{(inv\_avgXX }\OperatorTok{\%*\%}\StringTok{ }\NormalTok{avguXX }\OperatorTok{\%*\%}\StringTok{ }\NormalTok{inv\_avgXX) }\OperatorTok{*}\StringTok{ }\KeywordTok{nrow}\NormalTok{(X)}\OperatorTok{\^{}}\NormalTok{\{}\OperatorTok{{-}}\DecValTok{1}\NormalTok{\}}
\NormalTok{rse\_b \textless{}{-}}\StringTok{ }\KeywordTok{sqrt}\NormalTok{(}\KeywordTok{diag}\NormalTok{(vcov\_b))}

\NormalTok{label \textless{}{-}}\StringTok{ }\KeywordTok{c}\NormalTok{(}\StringTok{"(Intercept)"}\NormalTok{, }\StringTok{"factor(female)1"}\NormalTok{, }\StringTok{"age"}\NormalTok{, }\StringTok{"fare"}\NormalTok{)}
\KeywordTok{names}\NormalTok{(rse\_b) \textless{}{-}}\StringTok{ }\NormalTok{label}

\CommentTok{\# homoskedasticity{-}based standard errors}
\NormalTok{se\_b \textless{}{-}}\StringTok{ }\KeywordTok{sqrt}\NormalTok{(}\KeywordTok{diag}\NormalTok{(}\KeywordTok{vcov}\NormalTok{(LPM)))}

\KeywordTok{print}\NormalTok{(}\StringTok{"The Variance of OLS"}\NormalTok{); }\KeywordTok{vcov}\NormalTok{(LPM)}
\end{Highlighting}
\end{Shaded}

\begin{verbatim}
## [1] "The Variance of OLS"
\end{verbatim}

\begin{verbatim}
##                   (Intercept) factor(female)1           age          fare
## (Intercept)      9.754357e-04   -2.891381e-04 -2.333963e-05 -3.329763e-07
## factor(female)1 -2.891381e-04    7.136865e-04  2.373259e-06 -1.272800e-06
## age             -2.333963e-05    2.373259e-06  8.026024e-07 -4.090649e-08
## fare            -3.329763e-07   -1.272800e-06 -4.090649e-08  5.524412e-08
\end{verbatim}

\begin{Shaded}
\begin{Highlighting}[]
\KeywordTok{print}\NormalTok{(}\StringTok{"The Robust variance of OLS"}\NormalTok{); vcov\_b}
\end{Highlighting}
\end{Shaded}

\begin{verbatim}
## [1] "The Robust variance of OLS"
\end{verbatim}

\begin{verbatim}
##               (Intercept)        female           age          fare
## (Intercept)  1.133289e-03 -2.798532e-04 -2.789675e-05  2.813843e-07
## female      -2.798532e-04  7.903766e-04  3.169092e-06 -2.401923e-06
## age         -2.789675e-05  3.169092e-06  8.857523e-07 -3.650375e-08
## fare         2.813843e-07 -2.401923e-06 -3.650375e-08  4.071639e-08
\end{verbatim}

\begin{Shaded}
\begin{Highlighting}[]
\KeywordTok{print}\NormalTok{(}\StringTok{"The Robust se using White method"}\NormalTok{); rse\_b}
\end{Highlighting}
\end{Shaded}

\begin{verbatim}
## [1] "The Robust se using White method"
\end{verbatim}

\begin{verbatim}
##     (Intercept) factor(female)1             age            fare 
##    0.0336643606    0.0281136372    0.0009411442    0.0002017830
\end{verbatim}

\begin{Shaded}
\begin{Highlighting}[]
\KeywordTok{print}\NormalTok{(}\StringTok{"The Robust t{-}value using White method"}\NormalTok{); }\KeywordTok{coef}\NormalTok{(LPM)}\OperatorTok{/}\NormalTok{rse\_b}
\end{Highlighting}
\end{Shaded}

\begin{verbatim}
## [1] "The Robust t-value using White method"
\end{verbatim}

\begin{verbatim}
##     (Intercept) factor(female)1             age            fare 
##        6.482874       18.229508       -1.884168        7.162302
\end{verbatim}

Using the package \texttt{lmtest} and \texttt{sandwich} is the easiest
way to calculate heteroskedasticity-robust standard errors and
\(t\)-statistics.

\begin{Shaded}
\begin{Highlighting}[]
\KeywordTok{library}\NormalTok{(lmtest) }\CommentTok{\#use function \textasciigrave{}coeftest\textasciigrave{}}
\KeywordTok{library}\NormalTok{(sandwich) }\CommentTok{\#use function \textasciigrave{}vcovHC\textasciigrave{}}
\KeywordTok{coeftest}\NormalTok{(LPM, }\DataTypeTok{vcov =} \KeywordTok{vcovHC}\NormalTok{(LPM, }\DataTypeTok{type =} \StringTok{"HC0"}\NormalTok{))[, }\StringTok{"Std. Error"}\NormalTok{]}
\end{Highlighting}
\end{Shaded}

\begin{verbatim}
##     (Intercept) factor(female)1             age            fare 
##    0.0336643606    0.0281136372    0.0009411442    0.0002017830
\end{verbatim}

\begin{Shaded}
\begin{Highlighting}[]
\KeywordTok{coeftest}\NormalTok{(LPM, }\DataTypeTok{vcov =} \KeywordTok{vcovHC}\NormalTok{(LPM, }\DataTypeTok{type =} \StringTok{"HC0"}\NormalTok{))[, }\StringTok{"t value"}\NormalTok{]}
\end{Highlighting}
\end{Shaded}

\begin{verbatim}
##     (Intercept) factor(female)1             age            fare 
##        6.482874       18.229508       -1.884168        7.162302
\end{verbatim}

Finally, we summarize results of linear probability model in table
\ref{LPM}. We will discuss interpretation of results and goodness-of-fit
of LPM later.

\begin{Shaded}
\begin{Highlighting}[]
\CommentTok{\# t{-}stats}
\NormalTok{t\_b \textless{}{-}}\StringTok{ }\KeywordTok{coef}\NormalTok{(LPM)}\OperatorTok{/}\NormalTok{se\_b }
\NormalTok{rt\_b \textless{}{-}}\StringTok{ }\KeywordTok{coef}\NormalTok{(LPM)}\OperatorTok{/}\NormalTok{rse\_b}
\CommentTok{\# p{-}value Pr( \textgreater{} |t|)}
\NormalTok{p\_b \textless{}{-}}\StringTok{ }\KeywordTok{pt}\NormalTok{(}\KeywordTok{abs}\NormalTok{(t\_b), }\DataTypeTok{df =} \KeywordTok{nrow}\NormalTok{(X)}\OperatorTok{{-}}\KeywordTok{ncol}\NormalTok{(X), }\DataTypeTok{lower =} \OtherTok{FALSE}\NormalTok{)}\OperatorTok{*}\DecValTok{2}
\NormalTok{rp\_b \textless{}{-}}\StringTok{ }\KeywordTok{pt}\NormalTok{(}\KeywordTok{abs}\NormalTok{(rt\_b), }\DataTypeTok{df =} \KeywordTok{nrow}\NormalTok{(X)}\OperatorTok{{-}}\KeywordTok{ncol}\NormalTok{(X), }\DataTypeTok{lower =} \OtherTok{FALSE}\NormalTok{)}\OperatorTok{*}\DecValTok{2}

\KeywordTok{library}\NormalTok{(stargazer)}
\KeywordTok{stargazer}\NormalTok{(}
\NormalTok{  LPM, LPM,}
  \DataTypeTok{se =} \KeywordTok{list}\NormalTok{(se\_b, rse\_b), }\DataTypeTok{t =} \KeywordTok{list}\NormalTok{(t\_b, rt\_b), }\DataTypeTok{p =} \KeywordTok{list}\NormalTok{(p\_b, rp\_b),}
  \DataTypeTok{t.auto =} \OtherTok{FALSE}\NormalTok{, }\DataTypeTok{p.auto =} \OtherTok{FALSE}\NormalTok{,}
  \DataTypeTok{report =} \StringTok{"vcstp"}\NormalTok{, }\DataTypeTok{keep.stat =} \KeywordTok{c}\NormalTok{(}\StringTok{"n"}\NormalTok{),}
  \DataTypeTok{covariate.labels =} \KeywordTok{c}\NormalTok{(}\StringTok{"Female = 1"}\NormalTok{),}
  \DataTypeTok{add.lines =} \KeywordTok{list}\NormalTok{(}
    \KeywordTok{c}\NormalTok{(}\StringTok{"Standard errors"}\NormalTok{, }\StringTok{"Homoskedasticity{-}based"}\NormalTok{, }\StringTok{"Heteroskedasticity{-}robust"}\NormalTok{)),}
  \DataTypeTok{title =} \StringTok{"Results of Linear Probability Model"}\NormalTok{, }\DataTypeTok{label =} \StringTok{"LPM"}\NormalTok{,}
  \DataTypeTok{type =} \StringTok{"latex"}\NormalTok{, }\DataTypeTok{header =} \OtherTok{FALSE}\NormalTok{, }\DataTypeTok{font.size =} \StringTok{"small"}\NormalTok{,}
  \DataTypeTok{omit.table.layout =} \StringTok{"n"}\NormalTok{, }\DataTypeTok{table.placement =} \StringTok{"h"}
\NormalTok{)}
\end{Highlighting}
\end{Shaded}

\begin{table}[h] \centering 
  \caption{Results of Linear Probability Model} 
  \label{LPM} 
\small 
\begin{tabular}{@{\extracolsep{5pt}}lcc} 
\\[-1.8ex]\hline 
\hline \\[-1.8ex] 
 & \multicolumn{2}{c}{\textit{Dependent variable:}} \\ 
\cline{2-3} 
\\[-1.8ex] & \multicolumn{2}{c}{survived} \\ 
\\[-1.8ex] & (1) & (2)\\ 
\hline \\[-1.8ex] 
 Female = 1 & 0.512 & 0.512 \\ 
  & (0.027) & (0.028) \\ 
  & t = 19.184 & t = 18.230 \\ 
  & p = 0.000 & p = 0.000 \\ 
  & & \\ 
 age & $-$0.002 & $-$0.002 \\ 
  & (0.001) & (0.001) \\ 
  & t = $-$1.979 & t = $-$1.884 \\ 
  & p = 0.049 & p = 0.060 \\ 
  & & \\ 
 fare & 0.001 & 0.001 \\ 
  & (0.0002) & (0.0002) \\ 
  & t = 6.149 & t = 7.162 \\ 
  & p = 0.000 & p = 0.000 \\ 
  & & \\ 
 Constant & 0.218 & 0.218 \\ 
  & (0.031) & (0.034) \\ 
  & t = 6.988 & t = 6.483 \\ 
  & p = 0.000 & p = 0.000 \\ 
  & & \\ 
\hline \\[-1.8ex] 
Standard errors & Homoskedasticity-based & Heteroskedasticity-robust \\ 
Observations & 1,045 & 1,045 \\ 
\hline 
\hline \\[-1.8ex] 
\end{tabular} 
\end{table}

\hypertarget{probit-and-logit-model}{%
\subsection{Probit and Logit Model}\label{probit-and-logit-model}}

Unlike LPM, the probit and logit model must be estimated using the ML
method. The probability of observing \(y_i\) is \begin{equation*}
  p_{\beta}(y_i|\mathbf{x}_i)  
  = \mathbb{P}(y_i = 1 | x_i)^{y_i} [1 - \mathbb{P}(y_i = 1 | x_i)]^{1-y_i}
  = G(\mathbf{x}_i \beta)^{y_i} (1 - G(\mathbf{x}_i \beta))^{1-y_i}.
\end{equation*} Taking logalithm yields \begin{equation*}
  \log p_{\beta}(y_i|\mathbf{x}_i) = y_i \log(G(\mathbf{x}_i \beta)) + (1 - y_i)\log(1 - G(\mathbf{x}_i \beta)).
\end{equation*} The log-likelihood is \begin{equation*}
  M_n(\beta) = \sum_{i=1}^n \log p_{\beta}(y_i|\mathbf{x}_i).
\end{equation*}

The MLE \(\hat{\beta}\) holds that the score, which is the first-order
derivatives with respect to \(\beta\), is equal to 0. That is
\(\nabla_{\beta} M_n(\hat{\beta}) = 0\). For both logit and probit
model, the Hessian matrix, \(\nabla^2_{\beta\beta'} M_n(\beta)\), is
always negative definite. This implies that log-likelihood function
based on both models is grobally concave, and ensures that the MLE
maximizes the log-likelihood function. The first-order condition of the
probit model is \begin{equation*}
  \nabla_{\beta} M_n(\hat{\beta}) 
  = \sum_{i = 1}^n \left( y_i - \Phi(\mathbf{x}_i \hat{\beta}) \right) 
  \frac{\phi(\mathbf{x}_i \hat{\beta})}{\Phi(\mathbf{x}_i \hat{\beta})(1 - \phi(\mathbf{x}_i \hat{\beta}))} = 0.
\end{equation*} The first-order condition of the logit model is
\begin{equation*}
  \nabla_{\beta} M_n(\hat{\beta}) 
  = \sum_{i = 1}^n \left( y_i - G(\mathbf{x}_i \hat{\beta}) \right) \mathbf{x}'_i = 0.
\end{equation*} Since it is hard for us to solve this condition
analytically, we obtain estimators using numerical procedure.

The asymptotic distribution of \(\hat{\beta}\) is
\(\hat{\beta} \overset{d}{\to} N(\beta, \Sigma_{\beta})\) where
\begin{equation*}
  \Sigma_{\beta} = - \left( \sum_i E[E[ \nabla^2_{\beta\beta'} \log p_{\beta}(y_i | \mathbf{x}_i) | \mathbf{x}_i ]] \right)^{-1}.
\end{equation*} In practice, we replace
\(E[E[ \nabla^2_{\beta\beta'} \log p_{\beta}(y_i | \mathbf{x}_i) | \mathbf{x}_i ]]\)
by \begin{equation*}
  \frac{1}{n} \sum_i \nabla^2_{\beta\beta'} \log p_{\hat{\beta}}(y_i | \mathbf{x}_i),
\end{equation*} that is, \begin{equation*}
  \hat{\Sigma}_{\beta} = \left( \sum_i \nabla^2_{\beta\beta'} (-\log p_{\hat{\beta}}(y_i | \mathbf{x}_i)) \right)^{-1}.
\end{equation*}

In \texttt{R}, the function \texttt{nlm()} provides the Newton-Raphson
algorithm to minimize the function \footnote{\texttt{optim()} function
  is an another way to minimize the function. Especially, the function
  \texttt{optim(method\ =\ "BFGS")} provides the Quasi-Newton algorithm
  which carries on the spirit of Newton method.}. To run this function,
we need to define the log-likelihood function (\texttt{LnLik})
beforehand. Moreover, since we need to give initial values in augments,
we use coefficients estimated by OLS. Alternatively, we often use
\texttt{glm()} function. Using this function, we do not need to define
the log-likelihood function and initial values. Since estimates of
\texttt{glm()} are approximate to estiamtes of \texttt{nlm()}, we can
use this command safely. In this application, we use \texttt{nlm}
function to minimize the log-likelihood function.

\begin{Shaded}
\begin{Highlighting}[]
\NormalTok{Y \textless{}{-}}\StringTok{ }\NormalTok{dt}\OperatorTok{$}\NormalTok{survived}
\NormalTok{female \textless{}{-}}\StringTok{ }\NormalTok{dt}\OperatorTok{$}\NormalTok{female; age \textless{}{-}}\StringTok{ }\NormalTok{dt}\OperatorTok{$}\NormalTok{age; fare \textless{}{-}}\StringTok{ }\NormalTok{dt}\OperatorTok{$}\NormalTok{fare}

\CommentTok{\# log{-}likelihood}
\NormalTok{LnLik \textless{}{-}}\StringTok{ }\ControlFlowTok{function}\NormalTok{(b, }\DataTypeTok{model =} \KeywordTok{c}\NormalTok{(}\StringTok{"probit"}\NormalTok{, }\StringTok{"logit"}\NormalTok{)) \{}

\NormalTok{  xb \textless{}{-}}\StringTok{ }\NormalTok{b[}\DecValTok{1}\NormalTok{]}\OperatorTok{+}\StringTok{ }\NormalTok{b[}\DecValTok{2}\NormalTok{]}\OperatorTok{*}\NormalTok{female }\OperatorTok{+}\StringTok{ }\NormalTok{b[}\DecValTok{3}\NormalTok{]}\OperatorTok{*}\NormalTok{age }\OperatorTok{+}\StringTok{ }\NormalTok{b[}\DecValTok{4}\NormalTok{]}\OperatorTok{*}\NormalTok{fare}

  \ControlFlowTok{if}\NormalTok{ (model }\OperatorTok{==}\StringTok{ "probit"}\NormalTok{) \{}
\NormalTok{    L \textless{}{-}}\StringTok{ }\KeywordTok{pnorm}\NormalTok{(xb)}
\NormalTok{  \} }\ControlFlowTok{else}\NormalTok{ \{}
\NormalTok{    L \textless{}{-}}\StringTok{ }\DecValTok{1}\OperatorTok{/}\NormalTok{(}\DecValTok{1} \OperatorTok{+}\StringTok{ }\KeywordTok{exp}\NormalTok{(}\OperatorTok{{-}}\NormalTok{xb))}
\NormalTok{  \}}

\NormalTok{  LL\_i \textless{}{-}}\StringTok{ }\NormalTok{Y }\OperatorTok{*}\StringTok{ }\KeywordTok{log}\NormalTok{(L) }\OperatorTok{+}\StringTok{ }\NormalTok{(}\DecValTok{1} \OperatorTok{{-}}\StringTok{ }\NormalTok{Y) }\OperatorTok{*}\StringTok{ }\KeywordTok{log}\NormalTok{(}\DecValTok{1} \OperatorTok{{-}}\StringTok{ }\NormalTok{L)}
\NormalTok{  LL \textless{}{-}}\StringTok{ }\OperatorTok{{-}}\KeywordTok{sum}\NormalTok{(LL\_i)}

  \KeywordTok{return}\NormalTok{(LL)}
\NormalTok{\}}

\CommentTok{\#Newton{-}Raphson}
\NormalTok{init \textless{}{-}}\StringTok{ }\KeywordTok{c}\NormalTok{(}\FloatTok{0.218}\NormalTok{, }\FloatTok{0.512}\NormalTok{, }\FloatTok{{-}0.002}\NormalTok{, }\FloatTok{0.001}\NormalTok{)}
\NormalTok{probit \textless{}{-}}\StringTok{ }\KeywordTok{nlm}\NormalTok{(LnLik, init, }\DataTypeTok{model =} \StringTok{"probit"}\NormalTok{, }\DataTypeTok{hessian =} \OtherTok{TRUE}\NormalTok{)}

\NormalTok{label \textless{}{-}}\StringTok{ }\KeywordTok{c}\NormalTok{(}\StringTok{"(Intercept)"}\NormalTok{, }\StringTok{"factor(female)1"}\NormalTok{, }\StringTok{"age"}\NormalTok{, }\StringTok{"fare"}\NormalTok{)}
\KeywordTok{names}\NormalTok{(probit}\OperatorTok{$}\NormalTok{estimate) \textless{}{-}}\StringTok{ }\NormalTok{label}
\KeywordTok{colnames}\NormalTok{(probit}\OperatorTok{$}\NormalTok{hessian) \textless{}{-}}\StringTok{ }\NormalTok{label; }\KeywordTok{rownames}\NormalTok{(probit}\OperatorTok{$}\NormalTok{hessian) \textless{}{-}}\StringTok{ }\NormalTok{label}

\NormalTok{b\_probit \textless{}{-}}\StringTok{ }\NormalTok{probit}\OperatorTok{$}\NormalTok{estimate}
\NormalTok{vcov\_probit \textless{}{-}}\StringTok{ }\KeywordTok{solve}\NormalTok{(probit}\OperatorTok{$}\NormalTok{hessian); se\_probit \textless{}{-}}\StringTok{ }\KeywordTok{sqrt}\NormalTok{(}\KeywordTok{diag}\NormalTok{(vcov\_probit))}
\NormalTok{LL\_probit \textless{}{-}}\StringTok{ }\OperatorTok{{-}}\NormalTok{probit}\OperatorTok{$}\NormalTok{minimum}

\CommentTok{\#glm function}
\NormalTok{model \textless{}{-}}\StringTok{ }\NormalTok{survived }\OperatorTok{\textasciitilde{}}\StringTok{ }\KeywordTok{factor}\NormalTok{(female) }\OperatorTok{+}\StringTok{ }\NormalTok{age }\OperatorTok{+}\StringTok{ }\NormalTok{fare}
\NormalTok{probit\_glm \textless{}{-}}\StringTok{ }\KeywordTok{glm}\NormalTok{(model, }\DataTypeTok{data =}\NormalTok{ dt, }\DataTypeTok{family =} \KeywordTok{binomial}\NormalTok{(}\StringTok{"probit"}\NormalTok{))}

\CommentTok{\#result}
\KeywordTok{print}\NormalTok{(}\StringTok{"The MLE of probit model using nlm"}\NormalTok{); b\_probit}
\end{Highlighting}
\end{Shaded}

\begin{verbatim}
## [1] "The MLE of probit model using nlm"
\end{verbatim}

\begin{verbatim}
##     (Intercept) factor(female)1             age            fare 
##    -0.813995120     1.435384017    -0.006415761     0.005954843
\end{verbatim}

\begin{Shaded}
\begin{Highlighting}[]
\KeywordTok{print}\NormalTok{(}\StringTok{"The Variance of probit model using nlm"}\NormalTok{); vcov\_probit}
\end{Highlighting}
\end{Shaded}

\begin{verbatim}
## [1] "The Variance of probit model using nlm"
\end{verbatim}

\begin{verbatim}
##                   (Intercept) factor(female)1           age          fare
## (Intercept)      1.149118e-02   -3.569149e-03 -2.654781e-04 -1.375309e-05
## factor(female)1 -3.569149e-03    8.251773e-03  2.000500e-05 -5.991997e-06
## age             -2.654781e-04    2.000500e-05  9.630856e-06 -6.874343e-07
## fare            -1.375309e-05   -5.991997e-06 -6.874343e-07  1.103772e-06
\end{verbatim}

\begin{Shaded}
\begin{Highlighting}[]
\KeywordTok{print}\NormalTok{(}\StringTok{"The se of probit model using nlm"}\NormalTok{); se\_probit}
\end{Highlighting}
\end{Shaded}

\begin{verbatim}
## [1] "The se of probit model using nlm"
\end{verbatim}

\begin{verbatim}
##     (Intercept) factor(female)1             age            fare 
##     0.107196925     0.090839272     0.003103362     0.001050606
\end{verbatim}

\begin{Shaded}
\begin{Highlighting}[]
\KeywordTok{print}\NormalTok{(}\StringTok{"The coefficients of probit using glm"}\NormalTok{); }\KeywordTok{coef}\NormalTok{(probit\_glm)}
\end{Highlighting}
\end{Shaded}

\begin{verbatim}
## [1] "The coefficients of probit using glm"
\end{verbatim}

\begin{verbatim}
##     (Intercept) factor(female)1             age            fare 
##    -0.814075240     1.435384903    -0.006413717     0.005955479
\end{verbatim}

\begin{Shaded}
\begin{Highlighting}[]
\KeywordTok{print}\NormalTok{(}\StringTok{"The se of probit using glm"}\NormalTok{); }\KeywordTok{sqrt}\NormalTok{(}\KeywordTok{diag}\NormalTok{(}\KeywordTok{vcov}\NormalTok{(probit\_glm)))}
\end{Highlighting}
\end{Shaded}

\begin{verbatim}
## [1] "The se of probit using glm"
\end{verbatim}

\begin{verbatim}
##     (Intercept) factor(female)1             age            fare 
##     0.108614928     0.090860818     0.003139413     0.001056285
\end{verbatim}

Using \texttt{LogLik}, we can also estimate logit model by
Newton-Raphson algorithm. To compare result, we also use \texttt{glm()}
function.

\begin{Shaded}
\begin{Highlighting}[]
\CommentTok{\#Newton{-}Raphson}
\NormalTok{logit \textless{}{-}}\StringTok{ }\KeywordTok{nlm}\NormalTok{(LnLik, init, }\DataTypeTok{model =} \StringTok{"logit"}\NormalTok{, }\DataTypeTok{hessian =} \OtherTok{TRUE}\NormalTok{)}

\NormalTok{label \textless{}{-}}\StringTok{ }\KeywordTok{c}\NormalTok{(}\StringTok{"(Intercept)"}\NormalTok{, }\StringTok{"factor(female)1"}\NormalTok{, }\StringTok{"age"}\NormalTok{, }\StringTok{"fare"}\NormalTok{)}
\KeywordTok{names}\NormalTok{(logit}\OperatorTok{$}\NormalTok{estimate) \textless{}{-}}\StringTok{ }\NormalTok{label}
\KeywordTok{colnames}\NormalTok{(logit}\OperatorTok{$}\NormalTok{hessian) \textless{}{-}}\StringTok{ }\NormalTok{label; }\KeywordTok{rownames}\NormalTok{(logit}\OperatorTok{$}\NormalTok{hessian) \textless{}{-}}\StringTok{ }\NormalTok{label}

\NormalTok{b\_logit \textless{}{-}}\StringTok{ }\NormalTok{logit}\OperatorTok{$}\NormalTok{estimate}
\NormalTok{vcov\_logit \textless{}{-}}\StringTok{ }\KeywordTok{solve}\NormalTok{(logit}\OperatorTok{$}\NormalTok{hessian); se\_logit \textless{}{-}}\StringTok{ }\KeywordTok{sqrt}\NormalTok{(}\KeywordTok{diag}\NormalTok{(vcov\_logit))}
\NormalTok{LL\_logit \textless{}{-}}\StringTok{ }\OperatorTok{{-}}\NormalTok{logit}\OperatorTok{$}\NormalTok{minimum}

\CommentTok{\#glm function}
\NormalTok{logit\_glm \textless{}{-}}\StringTok{ }\KeywordTok{glm}\NormalTok{(model, }\DataTypeTok{data =}\NormalTok{ dt, }\DataTypeTok{family =} \KeywordTok{binomial}\NormalTok{(}\StringTok{"logit"}\NormalTok{))}

\CommentTok{\#result}
\KeywordTok{print}\NormalTok{(}\StringTok{"The MLE of logit model"}\NormalTok{); b\_logit}
\end{Highlighting}
\end{Shaded}

\begin{verbatim}
## [1] "The MLE of logit model"
\end{verbatim}

\begin{verbatim}
##     (Intercept) factor(female)1             age            fare 
##     -1.33719278      2.35516448     -0.01105760      0.01002878
\end{verbatim}

\begin{Shaded}
\begin{Highlighting}[]
\KeywordTok{print}\NormalTok{(}\StringTok{"The Variance of logit model"}\NormalTok{); vcov\_logit}
\end{Highlighting}
\end{Shaded}

\begin{verbatim}
## [1] "The Variance of logit model"
\end{verbatim}

\begin{verbatim}
##                   (Intercept) factor(female)1           age          fare
## (Intercept)      0.0351392692   -1.052616e-02 -8.031155e-04 -4.682750e-05
## factor(female)1 -0.0105261593    2.411636e-02  3.401375e-05 -7.818252e-06
## age             -0.0008031155    3.401375e-05  2.939124e-05 -2.170680e-06
## fare            -0.0000468275   -7.818252e-06 -2.170680e-06  3.448283e-06
\end{verbatim}

\begin{Shaded}
\begin{Highlighting}[]
\KeywordTok{print}\NormalTok{(}\StringTok{"The se of logit model"}\NormalTok{); se\_logit}
\end{Highlighting}
\end{Shaded}

\begin{verbatim}
## [1] "The se of logit model"
\end{verbatim}

\begin{verbatim}
##     (Intercept) factor(female)1             age            fare 
##     0.187454712     0.155294438     0.005421369     0.001856955
\end{verbatim}

\begin{Shaded}
\begin{Highlighting}[]
\KeywordTok{print}\NormalTok{(}\StringTok{"The coefficients of logit using glm"}\NormalTok{); }\KeywordTok{coef}\NormalTok{(logit\_glm)}
\end{Highlighting}
\end{Shaded}

\begin{verbatim}
## [1] "The coefficients of logit using glm"
\end{verbatim}

\begin{verbatim}
##     (Intercept) factor(female)1             age            fare 
##     -1.33727469      2.35516632     -0.01105553      0.01002942
\end{verbatim}

\begin{Shaded}
\begin{Highlighting}[]
\KeywordTok{print}\NormalTok{(}\StringTok{"The se of logit using glm"}\NormalTok{); }\KeywordTok{sqrt}\NormalTok{(}\KeywordTok{diag}\NormalTok{(}\KeywordTok{vcov}\NormalTok{(logit\_glm)))}
\end{Highlighting}
\end{Shaded}

\begin{verbatim}
## [1] "The se of logit using glm"
\end{verbatim}

\begin{verbatim}
##     (Intercept) factor(female)1             age            fare 
##     0.187350369     0.155280058     0.005424281     0.001847912
\end{verbatim}

As a result, table \ref{probit_logit} summarizes results of probit model
and logit model. \(t\)-statistics represents \(z\)-value which follows
the standard normal distribution. Standard errors are in parentheses. We
will discuss interpretation of results and goodness-of-fit later.

\begin{Shaded}
\begin{Highlighting}[]
\CommentTok{\# z{-}value}
\NormalTok{z\_probit \textless{}{-}}\StringTok{ }\NormalTok{b\_probit}\OperatorTok{/}\NormalTok{se\_probit}
\NormalTok{z\_logit \textless{}{-}}\StringTok{ }\NormalTok{b\_logit}\OperatorTok{/}\NormalTok{se\_logit}

\CommentTok{\# Pr(\textgreater{}|z|)}
\NormalTok{p\_probit \textless{}{-}}\StringTok{ }\KeywordTok{pnorm}\NormalTok{(}\KeywordTok{abs}\NormalTok{(z\_probit), }\DataTypeTok{lower =} \OtherTok{FALSE}\NormalTok{)}\OperatorTok{*}\DecValTok{2}
\NormalTok{p\_logit \textless{}{-}}\StringTok{ }\KeywordTok{pnorm}\NormalTok{(}\KeywordTok{abs}\NormalTok{(z\_logit), }\DataTypeTok{lower =} \OtherTok{FALSE}\NormalTok{)}\OperatorTok{*}\DecValTok{2}

\KeywordTok{stargazer}\NormalTok{(}
\NormalTok{  probit\_glm, logit\_glm,}
  \DataTypeTok{coef =} \KeywordTok{list}\NormalTok{(b\_probit, b\_logit), }\DataTypeTok{se =} \KeywordTok{list}\NormalTok{(se\_probit, se\_logit),}
  \DataTypeTok{t =} \KeywordTok{list}\NormalTok{(z\_probit, z\_logit), }\DataTypeTok{p =} \KeywordTok{list}\NormalTok{(p\_probit, p\_logit),}
  \DataTypeTok{t.auto =} \OtherTok{FALSE}\NormalTok{, }\DataTypeTok{p.auto =} \OtherTok{FALSE}\NormalTok{,}
  \DataTypeTok{report =} \StringTok{"vcstp"}\NormalTok{, }\DataTypeTok{keep.stat =} \KeywordTok{c}\NormalTok{(}\StringTok{"n"}\NormalTok{),}
  \DataTypeTok{covariate.labels =} \KeywordTok{c}\NormalTok{(}\StringTok{"Female = 1"}\NormalTok{),}
  \DataTypeTok{add.lines =} \KeywordTok{list}\NormalTok{(}
    \KeywordTok{c}\NormalTok{(}\StringTok{"Log{-}Likelihood"}\NormalTok{, }\KeywordTok{round}\NormalTok{(LL\_probit, }\DecValTok{3}\NormalTok{), }\KeywordTok{round}\NormalTok{(LL\_logit, }\DecValTok{3}\NormalTok{))),}
  \DataTypeTok{title =} \StringTok{"Results of Probit and Logit model"}\NormalTok{,}
  \DataTypeTok{label =} \StringTok{"probit\_logit"}\NormalTok{,}
  \DataTypeTok{type =} \StringTok{"latex"}\NormalTok{, }\DataTypeTok{header =} \OtherTok{FALSE}\NormalTok{, }\DataTypeTok{font.size =} \StringTok{"small"}\NormalTok{,}
  \DataTypeTok{table.placement =} \StringTok{"h"}\NormalTok{, }\DataTypeTok{omit.table.layout =} \StringTok{"n"}
\NormalTok{)}
\end{Highlighting}
\end{Shaded}

\begin{table}[h] \centering 
  \caption{Results of Probit and Logit model} 
  \label{probit_logit} 
\small 
\begin{tabular}{@{\extracolsep{5pt}}lcc} 
\\[-1.8ex]\hline 
\hline \\[-1.8ex] 
 & \multicolumn{2}{c}{\textit{Dependent variable:}} \\ 
\cline{2-3} 
\\[-1.8ex] & \multicolumn{2}{c}{survived} \\ 
\\[-1.8ex] & \textit{probit} & \textit{logistic} \\ 
\\[-1.8ex] & (1) & (2)\\ 
\hline \\[-1.8ex] 
 Female = 1 & 1.435 & 2.355 \\ 
  & (0.091) & (0.155) \\ 
  & t = 15.801 & t = 15.166 \\ 
  & p = 0.000 & p = 0.000 \\ 
  & & \\ 
 age & $-$0.006 & $-$0.011 \\ 
  & (0.003) & (0.005) \\ 
  & t = $-$2.067 & t = $-$2.040 \\ 
  & p = 0.039 & p = 0.042 \\ 
  & & \\ 
 fare & 0.006 & 0.010 \\ 
  & (0.001) & (0.002) \\ 
  & t = 5.668 & t = 5.401 \\ 
  & p = 0.000 & p = 0.00000 \\ 
  & & \\ 
 Constant & $-$0.814 & $-$1.337 \\ 
  & (0.107) & (0.187) \\ 
  & t = $-$7.593 & t = $-$7.133 \\ 
  & p = 0.000 & p = 0.000 \\ 
  & & \\ 
\hline \\[-1.8ex] 
Log-Likelihood & -530.404 & -530.947 \\ 
Observations & 1,045 & 1,045 \\ 
\hline 
\hline \\[-1.8ex] 
\end{tabular} 
\end{table}

\hypertarget{interpretaions}{%
\subsection{Interpretaions}\label{interpretaions}}

In the linear probability model, interepretations of coefficients are
straight-forward. The coefficient \(\beta_1\) is the change in survival
probability given a one-unit increase in continuous variable \(x\). In
the case of discrete variable, the coefficient \(\beta_1\) is the
difference in survival probability between two groups. However, when we
use the probit or logit model, it is hard for us to interepret results
because the partial effect is not constant across other covariates. As
an illustration, the partial effect of continuous variable \texttt{age}
is \begin{equation*}
  \partial_{age} \mathbb{P}[survived = 1 | female, age, fare] =
  \begin{cases}
    \beta_2  &\text{if LPM}  \\
    \phi(\mathbf{x}_i \beta) \beta_2  &\text{if Probit}  \\
    \frac{\exp(-\mathbf{x}_i \beta)}{(1 + \exp(-\mathbf{x}_i \beta))^2} \beta_2 &\text{if Logit}
  \end{cases}.
\end{equation*} The partial effect of dummy variable \texttt{female} is
\begin{equation*}
  \begin{split}
  &\mathbb{P}[survived = 1 | female = 1, age, fare] - \mathbb{P}[survived = 1 | female = 0, age, fare] \\
  =& 
  \begin{cases}
    \beta_1 &\text{if LPM}  \\
    \Phi(\beta_0 + \beta_1 + \beta_2 age + \beta_3 fare) - \Phi(\beta_0 + \beta_1 + \beta_2 age + \beta_3 fare)  &\text{if Probit}  \\
    \Lambda(\beta_0 + \beta_1 + \beta_2 age + \beta_3 fare) - \Lambda(\beta_0 + \beta_1 + \beta_2 age + \beta_3 fare)  &\text{if Logit}
  \end{cases}
  \end{split},
\end{equation*} where \(\Lambda(a) = 1/(1 + \exp(-a))\).

The first solution is to compute the partial effect at interesting
values of \(\mathbf{x}_i\). We often use the sample average of
covariates (``average'' person) to plugin in the partial effect formula.
This is sometimes called \emph{marginal effect at means}. However, since
it is unclear what the sample average of dummy variable represents, the
marginal effect at means may be hard to explain.

The second solution is to compute the average value of partial effect
across the population, that is, \begin{equation*}
  \partial_{x_{ij}} \mathbb{P}[y_i = 1 | \mathbf{x}_i] = \beta_j E[g(\mathbf{x}_i \beta)],
\end{equation*} or, in the case of discrete variable, \begin{equation*}
  E[ \mathbb{P}[y_i = 1 | x_{ij} = 1, \mathbf{x}_{i,-k}] - \mathbb{P}[y_i = 1 | x_{ij} = 0, \mathbf{x}_{i,-k}] ].
\end{equation*} This is called \emph{average marginal effect} (AME).
When we use dummy variables as explanatory variables, we should use this
solution.

Standard errors of average marginal effect can be obtained by the Delta
method. Let \(h_{ij}(\hat{\beta})\) be marginal (partial) effect of the
variable \(x_j\) for unit \(i\). Then, AME is
\(h_j(\hat{\beta}) = E[h_{ij}(\hat{\beta})]\). The Delta method implies
that
\(h_j(\hat{\beta}) \overset{d}{\to} N(h_j(\beta), \nabla_{\beta} h_j(\hat{\beta}) V(\beta) (\nabla_{\beta} h_j(\hat{\beta}))')\),
where \(V\) is variance of \(\beta\), and \begin{align*}
  \nabla_{\beta} h_j(\hat{\beta}) =
  \begin{pmatrix}
    \frac{\partial h_j(\hat{\beta})}{\partial \beta_1} & \cdots & \frac{\partial h_j(\hat{\beta})}{\partial \beta_k}
  \end{pmatrix}
\end{align*} When you use the \texttt{nlm} function to obtaine MLE, we
need to calculate standard errors manually. The \texttt{DeltaAME}
function is a function returing average marginal effect and its standard
errors.

\begin{Shaded}
\begin{Highlighting}[]
\NormalTok{DeltaAME \textless{}{-}}\StringTok{ }\ControlFlowTok{function}\NormalTok{(b, X, vcov, }\DataTypeTok{jbin =} \OtherTok{NULL}\NormalTok{, }\DataTypeTok{model =} \KeywordTok{c}\NormalTok{(}\StringTok{"probit"}\NormalTok{, }\StringTok{"logit"}\NormalTok{)) \{}
\NormalTok{  Xb \textless{}{-}}\StringTok{ }\KeywordTok{numeric}\NormalTok{(}\KeywordTok{nrow}\NormalTok{(X))}
  \ControlFlowTok{for}\NormalTok{ (i }\ControlFlowTok{in} \DecValTok{1}\OperatorTok{:}\KeywordTok{length}\NormalTok{(b)) \{}
\NormalTok{     Xb \textless{}{-}}\StringTok{ }\NormalTok{Xb }\OperatorTok{+}\StringTok{ }\NormalTok{b[i] }\OperatorTok{*}\StringTok{ }\NormalTok{X[,i]}
\NormalTok{  \}}

  \ControlFlowTok{if}\NormalTok{ (model }\OperatorTok{==}\StringTok{ "probit"}\NormalTok{) \{}
\NormalTok{    dens \textless{}{-}}\StringTok{ }\KeywordTok{dnorm}\NormalTok{(Xb)}
\NormalTok{    grad \textless{}{-}}\StringTok{ }\OperatorTok{{-}}\NormalTok{Xb }\OperatorTok{*}\StringTok{ }\NormalTok{dens}
\NormalTok{  \} }\ControlFlowTok{else}\NormalTok{ \{}
\NormalTok{    dens \textless{}{-}}\StringTok{ }\KeywordTok{exp}\NormalTok{(}\OperatorTok{{-}}\NormalTok{Xb)}\OperatorTok{/}\NormalTok{(}\DecValTok{1} \OperatorTok{+}\StringTok{ }\KeywordTok{exp}\NormalTok{(}\OperatorTok{{-}}\NormalTok{Xb))}\OperatorTok{\^{}}\DecValTok{2}
\NormalTok{    grad \textless{}{-}}\StringTok{ }\NormalTok{dens }\OperatorTok{*}\StringTok{ }\NormalTok{(}\OperatorTok{{-}}\DecValTok{1}\OperatorTok{+}\DecValTok{2}\OperatorTok{*}\KeywordTok{exp}\NormalTok{(}\OperatorTok{{-}}\NormalTok{Xb)}\OperatorTok{/}\NormalTok{(}\DecValTok{1}\OperatorTok{+}\KeywordTok{exp}\NormalTok{(}\OperatorTok{{-}}\NormalTok{Xb)))}
\NormalTok{  \}}

\NormalTok{  ame \textless{}{-}}\StringTok{ }\KeywordTok{mean}\NormalTok{(dens) }\OperatorTok{*}\StringTok{ }\NormalTok{b}
  \ControlFlowTok{if}\NormalTok{ (}\OperatorTok{!}\KeywordTok{is.null}\NormalTok{(jbin)) \{}
    \ControlFlowTok{for}\NormalTok{ (i }\ControlFlowTok{in}\NormalTok{ jbin) \{}
\NormalTok{      val1 \textless{}{-}}\StringTok{ }\NormalTok{X[,}\OperatorTok{{-}}\NormalTok{i] }\OperatorTok{\%*\%}\StringTok{ }\KeywordTok{matrix}\NormalTok{(b[}\OperatorTok{{-}}\NormalTok{i], }\DataTypeTok{ncol =} \DecValTok{1}\NormalTok{) }\OperatorTok{+}\StringTok{ }\NormalTok{b[i]}
\NormalTok{      val0 \textless{}{-}}\StringTok{ }\NormalTok{X[,}\OperatorTok{{-}}\NormalTok{i] }\OperatorTok{\%*\%}\StringTok{ }\KeywordTok{matrix}\NormalTok{(b[}\OperatorTok{{-}}\NormalTok{i], }\DataTypeTok{ncol =} \DecValTok{1}\NormalTok{)}
      \ControlFlowTok{if}\NormalTok{ (model }\OperatorTok{==}\StringTok{ "probit"}\NormalTok{) \{}
\NormalTok{        amed \textless{}{-}}\StringTok{ }\KeywordTok{mean}\NormalTok{(}\KeywordTok{pnorm}\NormalTok{(val1) }\OperatorTok{{-}}\StringTok{ }\KeywordTok{pnorm}\NormalTok{(val0))}
\NormalTok{      \} }\ControlFlowTok{else}\NormalTok{ \{ }
\NormalTok{        amed \textless{}{-}}\StringTok{ }\KeywordTok{mean}\NormalTok{((}\DecValTok{1}\OperatorTok{/}\NormalTok{(}\DecValTok{1} \OperatorTok{+}\StringTok{ }\KeywordTok{exp}\NormalTok{(}\OperatorTok{{-}}\NormalTok{val1))) }\OperatorTok{{-}}\StringTok{ }\NormalTok{(}\DecValTok{1}\OperatorTok{/}\NormalTok{(}\DecValTok{1} \OperatorTok{+}\StringTok{ }\KeywordTok{exp}\NormalTok{(}\OperatorTok{{-}}\NormalTok{val0))))}
\NormalTok{      \}}
\NormalTok{      ame[i] \textless{}{-}}\StringTok{ }\NormalTok{amed}
\NormalTok{    \}}
\NormalTok{  \}}

\NormalTok{  e \textless{}{-}}\StringTok{ }\OtherTok{NULL}
  \ControlFlowTok{for}\NormalTok{ (i }\ControlFlowTok{in} \DecValTok{1}\OperatorTok{:}\KeywordTok{length}\NormalTok{(b)) \{}
\NormalTok{    e \textless{}{-}}\StringTok{ }\KeywordTok{c}\NormalTok{(e, }\KeywordTok{rep}\NormalTok{(}\KeywordTok{mean}\NormalTok{(X[,i] }\OperatorTok{*}\StringTok{ }\NormalTok{grad), }\KeywordTok{length}\NormalTok{(b)))}
\NormalTok{  \}}

\NormalTok{  Jacob \textless{}{-}}\StringTok{ }\KeywordTok{matrix}\NormalTok{(e, }\DataTypeTok{nrow =} \KeywordTok{length}\NormalTok{(b), }\DataTypeTok{ncol =} \KeywordTok{length}\NormalTok{(b))}

  \ControlFlowTok{for}\NormalTok{ (i }\ControlFlowTok{in} \DecValTok{1}\OperatorTok{:}\KeywordTok{nrow}\NormalTok{(Jacob)) \{}
\NormalTok{    Jacob[i,] \textless{}{-}}\StringTok{ }\NormalTok{b[i] }\OperatorTok{*}\StringTok{ }\NormalTok{Jacob[i,]}
\NormalTok{  \}}
  \KeywordTok{diag}\NormalTok{(Jacob) \textless{}{-}}\StringTok{ }\KeywordTok{diag}\NormalTok{(Jacob) }\OperatorTok{+}\StringTok{ }\KeywordTok{rep}\NormalTok{(}\KeywordTok{mean}\NormalTok{(dens), }\KeywordTok{length}\NormalTok{(b))}

  \ControlFlowTok{if}\NormalTok{ (}\OperatorTok{!}\KeywordTok{is.null}\NormalTok{(jbin)) \{}
    \ControlFlowTok{for}\NormalTok{ (i }\ControlFlowTok{in}\NormalTok{ jbin) \{}
\NormalTok{      val1 \textless{}{-}}\StringTok{ }\NormalTok{X[,}\OperatorTok{{-}}\NormalTok{i] }\OperatorTok{\%*\%}\StringTok{ }\KeywordTok{matrix}\NormalTok{(b[}\OperatorTok{{-}}\NormalTok{i], }\DataTypeTok{ncol =} \DecValTok{1}\NormalTok{) }\OperatorTok{+}\StringTok{ }\NormalTok{b[i]}
\NormalTok{      val0 \textless{}{-}}\StringTok{ }\NormalTok{X[,}\OperatorTok{{-}}\NormalTok{i] }\OperatorTok{\%*\%}\StringTok{ }\KeywordTok{matrix}\NormalTok{(b[}\OperatorTok{{-}}\NormalTok{i], }\DataTypeTok{ncol =} \DecValTok{1}\NormalTok{)}
\NormalTok{      de \textless{}{-}}\StringTok{ }\OtherTok{NULL}
      \ControlFlowTok{if}\NormalTok{ (model }\OperatorTok{==}\StringTok{ "probit"}\NormalTok{) \{}
        \ControlFlowTok{for}\NormalTok{ (j }\ControlFlowTok{in} \DecValTok{1}\OperatorTok{:}\KeywordTok{length}\NormalTok{(b)) \{}
           \ControlFlowTok{if}\NormalTok{ (j }\OperatorTok{!=}\StringTok{ }\NormalTok{i) \{}
\NormalTok{            dep \textless{}{-}}\StringTok{ }\NormalTok{X[,j] }\OperatorTok{*}\StringTok{ }\NormalTok{(}\KeywordTok{dnorm}\NormalTok{(val1) }\OperatorTok{{-}}\StringTok{ }\KeywordTok{dnorm}\NormalTok{(val0))}
\NormalTok{            de \textless{}{-}}\StringTok{ }\KeywordTok{c}\NormalTok{(de, }\KeywordTok{mean}\NormalTok{(dep))}
\NormalTok{           \} }\ControlFlowTok{else}\NormalTok{ \{}
\NormalTok{            dep \textless{}{-}}\StringTok{ }\KeywordTok{dnorm}\NormalTok{(val1)}
\NormalTok{            de \textless{}{-}}\StringTok{ }\KeywordTok{c}\NormalTok{(de, }\KeywordTok{mean}\NormalTok{(dep))}
\NormalTok{           \}}
\NormalTok{        \}        }
\NormalTok{      \} }\ControlFlowTok{else}\NormalTok{ \{}
        \ControlFlowTok{for}\NormalTok{ (j }\ControlFlowTok{in} \DecValTok{1}\OperatorTok{:}\KeywordTok{length}\NormalTok{(b)) \{}
           \ControlFlowTok{if}\NormalTok{ (j }\OperatorTok{!=}\StringTok{ }\NormalTok{i) \{}
\NormalTok{            dep \textless{}{-}}\StringTok{ }\NormalTok{X[,j] }\OperatorTok{*}\StringTok{ }
\StringTok{              }\NormalTok{((}\KeywordTok{exp}\NormalTok{(}\OperatorTok{{-}}\NormalTok{val1)}\OperatorTok{/}\NormalTok{(}\DecValTok{1} \OperatorTok{+}\StringTok{ }\KeywordTok{exp}\NormalTok{(}\OperatorTok{{-}}\NormalTok{val1))}\OperatorTok{\^{}}\DecValTok{2}\NormalTok{) }\OperatorTok{{-}}\StringTok{ }\NormalTok{(}\KeywordTok{exp}\NormalTok{(}\OperatorTok{{-}}\NormalTok{val0)}\OperatorTok{/}\NormalTok{(}\DecValTok{1} \OperatorTok{+}\StringTok{ }\KeywordTok{exp}\NormalTok{(}\OperatorTok{{-}}\NormalTok{val0))}\OperatorTok{\^{}}\DecValTok{2}\NormalTok{))}
\NormalTok{            de \textless{}{-}}\StringTok{ }\KeywordTok{c}\NormalTok{(de, }\KeywordTok{mean}\NormalTok{(dep))}
\NormalTok{           \} }\ControlFlowTok{else}\NormalTok{ \{}
\NormalTok{            dep \textless{}{-}}\StringTok{ }\KeywordTok{exp}\NormalTok{(}\OperatorTok{{-}}\NormalTok{val1)}\OperatorTok{/}\NormalTok{(}\DecValTok{1} \OperatorTok{+}\StringTok{ }\KeywordTok{exp}\NormalTok{(}\OperatorTok{{-}}\NormalTok{val1))}\OperatorTok{\^{}}\DecValTok{2}
\NormalTok{            de \textless{}{-}}\StringTok{ }\KeywordTok{c}\NormalTok{(de, }\KeywordTok{mean}\NormalTok{(dep))}
\NormalTok{           \}}
\NormalTok{        \} }
\NormalTok{      \}}
\NormalTok{      Jacob[i,] \textless{}{-}}\StringTok{ }\NormalTok{de}
\NormalTok{    \}}
\NormalTok{  \}}

\NormalTok{  label \textless{}{-}}\StringTok{ }\KeywordTok{names}\NormalTok{(b)}
  \KeywordTok{colnames}\NormalTok{(Jacob) \textless{}{-}}\StringTok{ }\NormalTok{label; }\KeywordTok{rownames}\NormalTok{(Jacob) \textless{}{-}}\StringTok{ }\NormalTok{label}

\NormalTok{  vcov\_ame \textless{}{-}}\StringTok{ }\NormalTok{Jacob }\OperatorTok{\%*\%}\StringTok{ }\NormalTok{vcov }\OperatorTok{\%*\%}\StringTok{ }\KeywordTok{t}\NormalTok{(Jacob)}
\NormalTok{  se\_ame \textless{}{-}}\StringTok{ }\KeywordTok{sqrt}\NormalTok{(}\KeywordTok{diag}\NormalTok{(vcov\_ame))}
\NormalTok{  z\_ame \textless{}{-}}\StringTok{ }\NormalTok{ame}\OperatorTok{/}\NormalTok{se\_ame}
\NormalTok{  p\_ame \textless{}{-}}\StringTok{ }\KeywordTok{pnorm}\NormalTok{(}\KeywordTok{abs}\NormalTok{(z\_ame), }\DataTypeTok{lower =} \OtherTok{FALSE}\NormalTok{)}\OperatorTok{*}\DecValTok{2}

  \KeywordTok{return}\NormalTok{(}\KeywordTok{list}\NormalTok{(}\DataTypeTok{AME =}\NormalTok{ ame[}\OperatorTok{{-}}\DecValTok{1}\NormalTok{], }\DataTypeTok{SE =}\NormalTok{ se\_ame[}\OperatorTok{{-}}\DecValTok{1}\NormalTok{], }\DataTypeTok{zval =}\NormalTok{ z\_ame[}\OperatorTok{{-}}\DecValTok{1}\NormalTok{], }\DataTypeTok{pval =}\NormalTok{ p\_ame[}\OperatorTok{{-}}\DecValTok{1}\NormalTok{]))}
\NormalTok{\}}

\NormalTok{X \textless{}{-}}\StringTok{ }\KeywordTok{as.matrix}\NormalTok{(dt[,}\KeywordTok{c}\NormalTok{(}\StringTok{"(Intercept)"}\NormalTok{, }\StringTok{"female"}\NormalTok{, }\StringTok{"age"}\NormalTok{, }\StringTok{"fare"}\NormalTok{)])}
\NormalTok{ame\_probit \textless{}{-}}\StringTok{ }\KeywordTok{DeltaAME}\NormalTok{(b\_probit, X, vcov\_probit, }\DataTypeTok{jbin =} \DecValTok{2}\NormalTok{, }\DataTypeTok{model =} \StringTok{"probit"}\NormalTok{)}
\NormalTok{ame\_logit \textless{}{-}}\StringTok{ }\KeywordTok{DeltaAME}\NormalTok{(b\_logit, X, vcov\_logit, }\DataTypeTok{jbin =} \DecValTok{2}\NormalTok{, }\DataTypeTok{model =} \StringTok{"logit"}\NormalTok{)}

\KeywordTok{print}\NormalTok{(}\StringTok{"AME of probit estimates"}\NormalTok{); ame\_probit}\OperatorTok{$}\NormalTok{AME}
\end{Highlighting}
\end{Shaded}

\begin{verbatim}
## [1] "AME of probit estimates"
\end{verbatim}

\begin{verbatim}
## factor(female)1             age            fare 
##     0.508541457    -0.001824620     0.001693537
\end{verbatim}

\begin{Shaded}
\begin{Highlighting}[]
\KeywordTok{print}\NormalTok{(}\StringTok{"AME of logit estimates"}\NormalTok{); ame\_logit}\OperatorTok{$}\NormalTok{AME }
\end{Highlighting}
\end{Shaded}

\begin{verbatim}
## [1] "AME of logit estimates"
\end{verbatim}

\begin{verbatim}
## factor(female)1             age            fare 
##     0.507384641    -0.001823282     0.001653639
\end{verbatim}

\begin{Shaded}
\begin{Highlighting}[]
\KeywordTok{print}\NormalTok{(}\StringTok{"SE of AME of probit estimates"}\NormalTok{); ame\_probit}\OperatorTok{$}\NormalTok{SE }
\end{Highlighting}
\end{Shaded}

\begin{verbatim}
## [1] "SE of AME of probit estimates"
\end{verbatim}

\begin{verbatim}
## factor(female)1             age            fare 
##    0.0285474135    0.0008786651    0.0002874017
\end{verbatim}

\begin{Shaded}
\begin{Highlighting}[]
\KeywordTok{print}\NormalTok{(}\StringTok{"SE of AME of logit estimates"}\NormalTok{); ame\_logit}\OperatorTok{$}\NormalTok{SE}
\end{Highlighting}
\end{Shaded}

\begin{verbatim}
## [1] "SE of AME of logit estimates"
\end{verbatim}

\begin{verbatim}
## factor(female)1             age            fare 
##    0.0287277842    0.0008897546    0.0002948759
\end{verbatim}

When we use the \texttt{glm} function, we can use the function
\texttt{margins} in the library \texttt{margins} to obtain the average
marginal effect.

\begin{Shaded}
\begin{Highlighting}[]
\KeywordTok{library}\NormalTok{(margins)}
\KeywordTok{summary}\NormalTok{(}\KeywordTok{margins}\NormalTok{(probit\_glm))}
\end{Highlighting}
\end{Shaded}

\begin{verbatim}
##   factor     AME     SE       z      p   lower   upper
##      age -0.0018 0.0009 -2.0520 0.0402 -0.0036 -0.0001
##     fare  0.0017 0.0003  5.8398 0.0000  0.0011  0.0023
##  female1  0.5085 0.0286 17.7611 0.0000  0.4524  0.5647
\end{verbatim}

\begin{Shaded}
\begin{Highlighting}[]
\KeywordTok{summary}\NormalTok{(}\KeywordTok{margins}\NormalTok{(logit\_glm))}
\end{Highlighting}
\end{Shaded}

\begin{verbatim}
##   factor     AME     SE       z      p   lower   upper
##      age -0.0018 0.0009 -2.0480 0.0406 -0.0036 -0.0001
##     fare  0.0017 0.0003  5.6338 0.0000  0.0011  0.0022
##  female1  0.5074 0.0287 17.6652 0.0000  0.4511  0.5637
\end{verbatim}

Table \ref{titanic} shows results of linear probability model, probit
model, and logit model. In the probit and logit model, coefficients
report average marginal effects, \(t\)-statistics report
\(z\)-statistics which follows the standard normal distribution.

All specifications shows that the survival probability of female is
about 50\% point higher than of male, which is statistically
significant. Moreover, the survival probability is decreasing in age,
which implies children are more likely to survive. However, the size of
coefficient is small. Overall, crews obeyed the code of ``women and
children first'', but the survival probability of children is not
largely different from of adult.

\hypertarget{model-fitness}{%
\subsection{Model Fitness}\label{model-fitness}}

There are two measurements of goodness-of-fit. First, the \emph{percent
correctly predicted} reports the percentage of unit whose predicted
\(y_i\) matches the actual \(y_i\). The predicted \(y_i\) takes one if
\(G(\mathbf{x}_i \hat{\beta}) > 0.5\), and takes zero if
\(G(\mathbf{x}_i \hat{\beta}) \le 0.5\).

\begin{Shaded}
\begin{Highlighting}[]
\NormalTok{Y \textless{}{-}}\StringTok{ }\NormalTok{dt}\OperatorTok{$}\NormalTok{survived}
\NormalTok{X \textless{}{-}}\StringTok{ }\KeywordTok{as.matrix}\NormalTok{(dt[,}\KeywordTok{c}\NormalTok{(}\StringTok{"(Intercept)"}\NormalTok{, }\StringTok{"female"}\NormalTok{, }\StringTok{"age"}\NormalTok{, }\StringTok{"fare"}\NormalTok{)])}

\NormalTok{Xb\_lpm \textless{}{-}}\StringTok{ }\NormalTok{X }\OperatorTok{\%*\%}\StringTok{ }\KeywordTok{matrix}\NormalTok{(}\KeywordTok{coef}\NormalTok{(LPM), }\DataTypeTok{ncol =} \DecValTok{1}\NormalTok{)}
\NormalTok{Xb\_probit \textless{}{-}}\StringTok{ }\NormalTok{X }\OperatorTok{\%*\%}\StringTok{ }\KeywordTok{matrix}\NormalTok{(b\_probit, }\DataTypeTok{ncol =} \DecValTok{1}\NormalTok{)}
\NormalTok{Xb\_logit \textless{}{-}}\StringTok{ }\NormalTok{X }\OperatorTok{\%*\%}\StringTok{ }\KeywordTok{matrix}\NormalTok{(b\_logit, }\DataTypeTok{ncol =} \DecValTok{1}\NormalTok{)}

\NormalTok{hatY\_lpm \textless{}{-}}\StringTok{ }\KeywordTok{ifelse}\NormalTok{(Xb\_lpm }\OperatorTok{\textgreater{}}\StringTok{ }\FloatTok{0.5}\NormalTok{, }\DecValTok{1}\NormalTok{, }\DecValTok{0}\NormalTok{)}
\NormalTok{hatY\_probit \textless{}{-}}\StringTok{ }\KeywordTok{ifelse}\NormalTok{(}\KeywordTok{pnorm}\NormalTok{(Xb\_probit) }\OperatorTok{\textgreater{}}\StringTok{ }\FloatTok{0.5}\NormalTok{, }\DecValTok{1}\NormalTok{, }\DecValTok{0}\NormalTok{)}
\NormalTok{hatY\_logit \textless{}{-}}\StringTok{ }\KeywordTok{ifelse}\NormalTok{(}\DecValTok{1}\OperatorTok{/}\NormalTok{(}\DecValTok{1} \OperatorTok{+}\StringTok{ }\KeywordTok{exp}\NormalTok{(}\OperatorTok{{-}}\NormalTok{Xb\_logit)) }\OperatorTok{\textgreater{}}\StringTok{ }\FloatTok{0.5}\NormalTok{, }\DecValTok{1}\NormalTok{, }\DecValTok{0}\NormalTok{)}

\NormalTok{pcp\_lpm \textless{}{-}}\StringTok{ }\KeywordTok{round}\NormalTok{(}\KeywordTok{sum}\NormalTok{(Y }\OperatorTok{==}\StringTok{ }\NormalTok{hatY\_lpm)}\OperatorTok{/}\KeywordTok{nrow}\NormalTok{(X), }\DecValTok{4}\NormalTok{)}
\NormalTok{pcp\_probit \textless{}{-}}\StringTok{ }\KeywordTok{round}\NormalTok{(}\KeywordTok{sum}\NormalTok{(Y }\OperatorTok{==}\StringTok{ }\NormalTok{hatY\_probit)}\OperatorTok{/}\KeywordTok{nrow}\NormalTok{(X), }\DecValTok{4}\NormalTok{)}
\NormalTok{pcp\_logit \textless{}{-}}\StringTok{ }\KeywordTok{round}\NormalTok{(}\KeywordTok{sum}\NormalTok{(Y }\OperatorTok{==}\StringTok{ }\NormalTok{hatY\_logit)}\OperatorTok{/}\KeywordTok{nrow}\NormalTok{(X), }\DecValTok{4}\NormalTok{)}
\end{Highlighting}
\end{Shaded}

Second measurement is the \emph{pseudo R-squared}. The pseudo R-squared
is obtained by \(1 - \sum_i \hat{u}_i^2/ \sum_i y_i^2\), where
\(\hat{u}_i = y_i - G(\mathbf{x}_i \hat{\beta})\).

\begin{Shaded}
\begin{Highlighting}[]
\NormalTok{Y2 \textless{}{-}}\StringTok{ }\NormalTok{Y}\OperatorTok{\^{}}\DecValTok{2}

\NormalTok{hatu\_lpm \textless{}{-}}\StringTok{ }\NormalTok{(Y }\OperatorTok{{-}}\StringTok{ }\NormalTok{Xb\_lpm)}\OperatorTok{\^{}}\DecValTok{2}
\NormalTok{hatu\_probit \textless{}{-}}\StringTok{ }\NormalTok{(Y }\OperatorTok{{-}}\StringTok{ }\KeywordTok{pnorm}\NormalTok{(Xb\_probit))}\OperatorTok{\^{}}\DecValTok{2}
\NormalTok{hatu\_logit \textless{}{-}}\StringTok{ }\NormalTok{(Y }\OperatorTok{{-}}\StringTok{ }\DecValTok{1}\OperatorTok{/}\NormalTok{(}\DecValTok{1} \OperatorTok{+}\StringTok{ }\KeywordTok{exp}\NormalTok{(}\OperatorTok{{-}}\NormalTok{Xb\_logit)))}\OperatorTok{\^{}}\DecValTok{2}

\NormalTok{pr2\_lpm \textless{}{-}}\StringTok{ }\KeywordTok{round}\NormalTok{(}\DecValTok{1} \OperatorTok{{-}}\StringTok{ }\KeywordTok{sum}\NormalTok{(hatu\_lpm)}\OperatorTok{/}\KeywordTok{sum}\NormalTok{(Y2), }\DecValTok{4}\NormalTok{)}
\NormalTok{pr2\_probit \textless{}{-}}\StringTok{ }\KeywordTok{round}\NormalTok{(}\DecValTok{1} \OperatorTok{{-}}\StringTok{ }\KeywordTok{sum}\NormalTok{(hatu\_probit)}\OperatorTok{/}\KeywordTok{sum}\NormalTok{(Y2), }\DecValTok{4}\NormalTok{)}
\NormalTok{pr2\_logit \textless{}{-}}\StringTok{ }\KeywordTok{round}\NormalTok{(}\DecValTok{1} \OperatorTok{{-}}\StringTok{ }\KeywordTok{sum}\NormalTok{(hatu\_logit)}\OperatorTok{/}\KeywordTok{sum}\NormalTok{(Y2), }\DecValTok{4}\NormalTok{)}
\end{Highlighting}
\end{Shaded}

Table \ref{titanic} summarizes two measurements of model fitness. There
is little difference among LPM, probit model, and logit model.

\begin{Shaded}
\begin{Highlighting}[]
\KeywordTok{stargazer}\NormalTok{(}
\NormalTok{  LPM, probit\_glm, logit\_glm,}
  \DataTypeTok{coef =} \KeywordTok{list}\NormalTok{(}\KeywordTok{coef}\NormalTok{(LPM), ame\_probit}\OperatorTok{$}\NormalTok{AME, ame\_logit}\OperatorTok{$}\NormalTok{AME),}
  \DataTypeTok{se =} \KeywordTok{list}\NormalTok{(rse\_b, ame\_probit}\OperatorTok{$}\NormalTok{SE, ame\_logit}\OperatorTok{$}\NormalTok{SE),}
  \DataTypeTok{t =} \KeywordTok{list}\NormalTok{(rt\_b, ame\_probit}\OperatorTok{$}\NormalTok{zval, ame\_logit}\OperatorTok{$}\NormalTok{zval),}
  \DataTypeTok{p =} \KeywordTok{list}\NormalTok{(rp\_b, ame\_probit}\OperatorTok{$}\NormalTok{pval, ame\_logit}\OperatorTok{$}\NormalTok{pval),}
  \DataTypeTok{t.auto =} \OtherTok{FALSE}\NormalTok{, }\DataTypeTok{p.auto =} \OtherTok{FALSE}\NormalTok{,}
  \DataTypeTok{omit =} \KeywordTok{c}\NormalTok{(}\StringTok{"Constant"}\NormalTok{), }\DataTypeTok{covariate.labels =} \KeywordTok{c}\NormalTok{(}\StringTok{"Female = 1"}\NormalTok{),}
  \DataTypeTok{report =} \StringTok{"vcstp"}\NormalTok{, }\DataTypeTok{keep.stat =} \KeywordTok{c}\NormalTok{(}\StringTok{"n"}\NormalTok{),}
  \DataTypeTok{add.lines =} \KeywordTok{list}\NormalTok{(}
    \KeywordTok{c}\NormalTok{(}\StringTok{"Percent correctly predicted"}\NormalTok{, pcp\_lpm, pcp\_probit, pcp\_logit),}
    \KeywordTok{c}\NormalTok{(}\StringTok{"Pseudo R{-}squared"}\NormalTok{, pr2\_lpm, pr2\_probit, pr2\_logit)}
\NormalTok{  ),}
  \DataTypeTok{omit.table.layout =} \StringTok{"n"}\NormalTok{, }\DataTypeTok{table.placement =} \StringTok{"t"}\NormalTok{,}
  \DataTypeTok{title =} \StringTok{"Titanic Survivors: LPM, Probit (AME), and Logit (AME)"}\NormalTok{,}
  \DataTypeTok{label =} \StringTok{"titanic"}\NormalTok{,}
  \DataTypeTok{type =} \StringTok{"latex"}\NormalTok{, }\DataTypeTok{header =} \OtherTok{FALSE}
\NormalTok{)}
\end{Highlighting}
\end{Shaded}

\begin{table}[t] \centering 
  \caption{Titanic Survivors: LPM, Probit (AME), and Logit (AME)} 
  \label{titanic} 
\begin{tabular}{@{\extracolsep{5pt}}lccc} 
\\[-1.8ex]\hline 
\hline \\[-1.8ex] 
 & \multicolumn{3}{c}{\textit{Dependent variable:}} \\ 
\cline{2-4} 
\\[-1.8ex] & \multicolumn{3}{c}{survived} \\ 
\\[-1.8ex] & \textit{OLS} & \textit{probit} & \textit{logistic} \\ 
\\[-1.8ex] & (1) & (2) & (3)\\ 
\hline \\[-1.8ex] 
 Female = 1 & 0.512 & 0.509 & 0.507 \\ 
  & (0.028) & (0.029) & (0.029) \\ 
  & t = 18.230 & t = 17.814 & t = 17.662 \\ 
  & p = 0.000 & p = 0.000 & p = 0.000 \\ 
  & & & \\ 
 age & $-$0.002 & $-$0.002 & $-$0.002 \\ 
  & (0.001) & (0.001) & (0.001) \\ 
  & t = $-$1.884 & t = $-$2.077 & t = $-$2.049 \\ 
  & p = 0.060 & p = 0.038 & p = 0.041 \\ 
  & & & \\ 
 fare & 0.001 & 0.002 & 0.002 \\ 
  & (0.0002) & (0.0003) & (0.0003) \\ 
  & t = 7.162 & t = 5.893 & t = 5.608 \\ 
  & p = 0.000 & p = 0.000 & p = 0.00000 \\ 
  & & & \\ 
\hline \\[-1.8ex] 
Percent correctly predicted & 0.7799 & 0.7742 & 0.7742 \\ 
Pseudo R-squared & 0.5946 & 0.5945 & 0.594 \\ 
Observations & 1,045 & 1,045 & 1,045 \\ 
\hline 
\hline \\[-1.8ex] 
\end{tabular} 
\end{table}

\hypertarget{empirical-application-of-ordered-probit-and-logit-model-housing-as-status-goods}{%
\section{Empirical Application of Ordered Probit and Logit Model:
Housing as Status
Goods}\label{empirical-application-of-ordered-probit-and-logit-model-housing-as-status-goods}}

\textbf{Breif Background}. Social image may affect consumption behavior.
Specifically, a desire to signal high income or wealth may cause
consumers to purchase status goods. In this application, we explore
whether living in an upper floor serves as a status goods.

\noindent \textbf{Data}. We use the housing data originally coming from
the American Housing Survey conducted in 2013 \footnote{\url{https://www.census.gov/programs-surveys/ahs.html}.
  This is a repeated cross-section survey. We use the data at one time.}.
We use the following variable

\begin{itemize}
\tightlist
\item
  \texttt{Level}: ordered value of a story of respondent's living (1:Low
  - 4:High)
\item
  \texttt{Levelnum}: variable we recode the response \texttt{Level} as
  25, 50, 75, 100. This represents the extent of floor height.
\item
  \texttt{lnPrice}: logged price of housing (proxy for quality of house)
\item
  \texttt{Top25}: a dummy variable taking one if household income is in
  the top 25 percentile in sample.
\end{itemize}

\begin{Shaded}
\begin{Highlighting}[]
\NormalTok{house \textless{}{-}}\StringTok{ }\KeywordTok{read.csv}\NormalTok{(}\DataTypeTok{file =} \StringTok{"./data/housing.csv"}\NormalTok{, }\DataTypeTok{header =} \OtherTok{TRUE}\NormalTok{,  }\DataTypeTok{sep =} \StringTok{","}\NormalTok{)}
\NormalTok{house \textless{}{-}}\StringTok{ }\NormalTok{house[,}\KeywordTok{c}\NormalTok{(}\StringTok{"Level"}\NormalTok{, }\StringTok{"lnPrice"}\NormalTok{, }\StringTok{"Top25"}\NormalTok{)]}
\NormalTok{house}\OperatorTok{$}\NormalTok{Levelnum \textless{}{-}}\StringTok{ }\KeywordTok{ifelse}\NormalTok{(}
\NormalTok{  house}\OperatorTok{$}\NormalTok{Level }\OperatorTok{==}\StringTok{ }\DecValTok{1}\NormalTok{, }\DecValTok{25}\NormalTok{, }
  \KeywordTok{ifelse}\NormalTok{(house}\OperatorTok{$}\NormalTok{Level }\OperatorTok{==}\StringTok{ }\DecValTok{2}\NormalTok{, }\DecValTok{50}\NormalTok{, }
  \KeywordTok{ifelse}\NormalTok{(house}\OperatorTok{$}\NormalTok{Level }\OperatorTok{==}\StringTok{ }\DecValTok{3}\NormalTok{, }\DecValTok{75}\NormalTok{, }\DecValTok{100}\NormalTok{)))}
\KeywordTok{head}\NormalTok{(house)}
\end{Highlighting}
\end{Shaded}

\begin{verbatim}
##   Level  lnPrice Top25 Levelnum
## 1     3 11.51294     0       75
## 2     4 11.51294     1      100
## 3     3 11.60824     0       75
## 4     3 11.69526     0       75
## 5     3 12.57764     0       75
## 6     3 12.64433     0       75
\end{verbatim}

\noindent \textbf{Model}. The outcome variable is \texttt{Level} taking
\(\{1, 2, 3, 4\}\). Consider the following regression equation of a
latent variable: \begin{equation*}
  y_i^* = \mathbf{x}_i \beta + u_i,
\end{equation*} where \(\mathbf{x}_i = (lnPrice, Top25)\) and \(u_i\) is
an error term. The relationship between the latent variable \(y_i^*\)
and the observed outcome variable is \begin{equation*}
  Level =
  \begin{cases}
    1 &\text{if}\quad -\infty < y_i^* \le a_1  \\
    2 &\text{if}\quad a_1 < y_i^* \le a_2 \\
    3 &\text{if}\quad a_2 < y_i^* \le a_3 \\
    4 &\text{if}\quad a_3 < y_i^* < +\infty
  \end{cases}.
\end{equation*}

Consider the probability of realization of \(y_i\), that is,
\begin{equation*}
  \begin{split}
  \mathbb{P}(y_i = k | \mathbf{x}_i) 
  &= \mathbb{P}(a_{k-1} - \mathbf{x}_i \beta < u_i \le a_k - \mathbf{x}_i \beta | \mathbf{x}_i)  \\
  &= G(a_k - \mathbf{x}_i \beta) - G(a_{k-1} - \mathbf{x}_i \beta),
  \end{split}
\end{equation*} where \(a_{4} = +\infty\) and \(a_0 = -\infty\). Then,
the likelihood function is defined by \begin{equation*}
  p((y_i|\mathbf{x}_i), i = 1, \ldots, n; \beta, a_1, \ldots, a_3)
  = \prod_{i=1}^n \prod_{k=1}^4 (G(a_k - \mathbf{x}_i \beta) - G(a_{k-1} - \mathbf{x}_i \beta))^{I_{ik}}.
\end{equation*} where \(I_{ik}\) is a indicator variable taking 1 if
\(y_i = k\). Finally, the log-likelihood function is \begin{equation*}
  M(\beta, a_1, a_2, a_3) = \sum_{i=1}^n \sum_{k=1}^4 I_{ik} \log(G(a_k - \mathbf{x}_i \beta) - G(a_{k-1} - \mathbf{x}_i \beta)).
\end{equation*} Usually, \(G(a)\) assumes the standard normal
distribution, \(\Phi(a)\), or the logistic distribution,
\(1/(1 + \exp(-a))\).

In \texttt{R}, the library (package) \texttt{MASS} provides the
\texttt{polr} function which estimates the ordered probit and logit
model. Although we can use the \texttt{nlm} function when we define the
log-likelihood function, we do not report this method. To compare
results, we use the variable \texttt{Levelnum} as outcome variable, and
apply the linear regression model.

\begin{Shaded}
\begin{Highlighting}[]
\KeywordTok{library}\NormalTok{(MASS)}
\KeywordTok{library}\NormalTok{(tidyverse) }\CommentTok{\#use case\_when()}

\NormalTok{ols \textless{}{-}}\StringTok{ }\KeywordTok{lm}\NormalTok{(Levelnum }\OperatorTok{\textasciitilde{}}\StringTok{ }\NormalTok{lnPrice }\OperatorTok{+}\StringTok{ }\NormalTok{Top25, }\DataTypeTok{data =}\NormalTok{ house)}

\NormalTok{model \textless{}{-}}\StringTok{ }\KeywordTok{factor}\NormalTok{(Level) }\OperatorTok{\textasciitilde{}}\StringTok{ }\NormalTok{lnPrice }\OperatorTok{+}\StringTok{ }\NormalTok{Top25}
\NormalTok{oprobit \textless{}{-}}\StringTok{ }\KeywordTok{polr}\NormalTok{(model, }\DataTypeTok{data =}\NormalTok{ house, }\DataTypeTok{method =} \StringTok{"probit"}\NormalTok{)}
\NormalTok{ologit \textless{}{-}}\StringTok{ }\KeywordTok{polr}\NormalTok{(model, }\DataTypeTok{data =}\NormalTok{ house, }\DataTypeTok{method =} \StringTok{"logistic"}\NormalTok{)}

\NormalTok{a\_oprobit \textless{}{-}}\StringTok{ }\KeywordTok{round}\NormalTok{(oprobit}\OperatorTok{$}\NormalTok{zeta, }\DecValTok{3}\NormalTok{)}
\NormalTok{a\_ologit \textless{}{-}}\StringTok{ }\KeywordTok{round}\NormalTok{(ologit}\OperatorTok{$}\NormalTok{zeta, }\DecValTok{3}\NormalTok{)}

\NormalTok{xb\_oprobit \textless{}{-}}\StringTok{ }\NormalTok{oprobit}\OperatorTok{$}\NormalTok{lp }
\NormalTok{xb\_ologit \textless{}{-}}\StringTok{ }\NormalTok{ologit}\OperatorTok{$}\NormalTok{lp}

\NormalTok{hatY\_oprobit \textless{}{-}}\StringTok{ }\KeywordTok{case\_when}\NormalTok{(}
\NormalTok{  xb\_oprobit }\OperatorTok{\textless{}=}\StringTok{ }\NormalTok{oprobit}\OperatorTok{$}\NormalTok{zeta[}\DecValTok{1}\NormalTok{] }\OperatorTok{\textasciitilde{}}\StringTok{ }\DecValTok{1}\NormalTok{,}
\NormalTok{  xb\_oprobit }\OperatorTok{\textless{}=}\StringTok{ }\NormalTok{oprobit}\OperatorTok{$}\NormalTok{zeta[}\DecValTok{2}\NormalTok{] }\OperatorTok{\textasciitilde{}}\StringTok{ }\DecValTok{2}\NormalTok{,}
\NormalTok{  xb\_oprobit }\OperatorTok{\textless{}=}\StringTok{ }\NormalTok{oprobit}\OperatorTok{$}\NormalTok{zeta[}\DecValTok{3}\NormalTok{] }\OperatorTok{\textasciitilde{}}\StringTok{ }\DecValTok{3}\NormalTok{,}
  \OtherTok{TRUE} \OperatorTok{\textasciitilde{}}\StringTok{ }\DecValTok{4}
\NormalTok{)}
\NormalTok{hatY\_ologit \textless{}{-}}\StringTok{ }\KeywordTok{case\_when}\NormalTok{(}
\NormalTok{  xb\_ologit }\OperatorTok{\textless{}=}\StringTok{ }\NormalTok{ologit}\OperatorTok{$}\NormalTok{zeta[}\DecValTok{1}\NormalTok{] }\OperatorTok{\textasciitilde{}}\StringTok{ }\DecValTok{1}\NormalTok{,}
\NormalTok{  xb\_ologit }\OperatorTok{\textless{}=}\StringTok{ }\NormalTok{ologit}\OperatorTok{$}\NormalTok{zeta[}\DecValTok{2}\NormalTok{] }\OperatorTok{\textasciitilde{}}\StringTok{ }\DecValTok{2}\NormalTok{,}
\NormalTok{  xb\_ologit }\OperatorTok{\textless{}=}\StringTok{ }\NormalTok{ologit}\OperatorTok{$}\NormalTok{zeta[}\DecValTok{3}\NormalTok{] }\OperatorTok{\textasciitilde{}}\StringTok{ }\DecValTok{3}\NormalTok{,}
  \OtherTok{TRUE} \OperatorTok{\textasciitilde{}}\StringTok{ }\DecValTok{4}
\NormalTok{)}

\NormalTok{pred\_oprobit \textless{}{-}}\StringTok{ }\KeywordTok{round}\NormalTok{(}\KeywordTok{sum}\NormalTok{(house}\OperatorTok{$}\NormalTok{Level }\OperatorTok{==}\StringTok{ }\NormalTok{hatY\_oprobit)}\OperatorTok{/}\KeywordTok{nrow}\NormalTok{(house), }\DecValTok{3}\NormalTok{)}
\NormalTok{pred\_ologit \textless{}{-}}\StringTok{ }\KeywordTok{round}\NormalTok{(}\KeywordTok{sum}\NormalTok{(house}\OperatorTok{$}\NormalTok{Level }\OperatorTok{==}\StringTok{ }\NormalTok{hatY\_ologit)}\OperatorTok{/}\KeywordTok{nrow}\NormalTok{(house), }\DecValTok{3}\NormalTok{)}
\end{Highlighting}
\end{Shaded}

\hypertarget{interepretations}{%
\subsection{Interepretations}\label{interepretations}}

Table \ref{housing} shows results. OLS model shows that respondents
whose household income is in the top 25 percentile live in 3.7\% higher
floor than other respondents. This implies that high earners want to
live in higher floor, which may serve as a status goods. The ordered
probit and logit model are in line with this result. To evaluate two
models quantitatively, consider the following equation.
\begin{equation*}
  E[Levelnum | \mathbf{x}_i] = 
  25\mathbb{P}[level = 1| \mathbf{x}_i] + 
  50\mathbb{P}[level = 2| \mathbf{x}_i] + 
  75\mathbb{P}[level = 3| \mathbf{x}_i] + 
  100\mathbb{P}[level = 4| \mathbf{x}_i].
\end{equation*} We compute this equation with \(Top25 = 1\) and
\(Top25 = 0\) at mean value of \(lnPrice\) and take difference.

\begin{Shaded}
\begin{Highlighting}[]
\NormalTok{quantef \textless{}{-}}\StringTok{ }\ControlFlowTok{function}\NormalTok{(model) \{}
\NormalTok{  b \textless{}{-}}\StringTok{ }\KeywordTok{coef}\NormalTok{(model)}
\NormalTok{  val1 \textless{}{-}}\StringTok{ }\KeywordTok{mean}\NormalTok{(house}\OperatorTok{$}\NormalTok{lnPrice)}\OperatorTok{*}\NormalTok{b[}\DecValTok{1}\NormalTok{] }\OperatorTok{+}\StringTok{ }\NormalTok{b[}\DecValTok{2}\NormalTok{]}
\NormalTok{  val0 \textless{}{-}}\StringTok{ }\KeywordTok{mean}\NormalTok{(house}\OperatorTok{$}\NormalTok{lnPrice)}\OperatorTok{*}\NormalTok{b[}\DecValTok{1}\NormalTok{]}

\NormalTok{  prob \textless{}{-}}\StringTok{ }\KeywordTok{matrix}\NormalTok{(}\KeywordTok{c}\NormalTok{(}\KeywordTok{rep}\NormalTok{(val1, }\DecValTok{3}\NormalTok{), }\KeywordTok{rep}\NormalTok{(val0, }\DecValTok{3}\NormalTok{)), }\DataTypeTok{ncol =} \DecValTok{2}\NormalTok{, }\DataTypeTok{nrow =} \DecValTok{3}\NormalTok{)}
  \ControlFlowTok{for}\NormalTok{ (i }\ControlFlowTok{in} \DecValTok{1}\OperatorTok{:}\DecValTok{3}\NormalTok{) \{}
    \ControlFlowTok{for}\NormalTok{ (j }\ControlFlowTok{in} \DecValTok{1}\OperatorTok{:}\DecValTok{2}\NormalTok{) \{}
\NormalTok{      prob[i,j] \textless{}{-}}\StringTok{ }\KeywordTok{pnorm}\NormalTok{(model}\OperatorTok{$}\NormalTok{zeta[i] }\OperatorTok{{-}}\StringTok{ }\NormalTok{prob[i,j])}
\NormalTok{    \}}
\NormalTok{  \}}
\NormalTok{  Ey1 \textless{}{-}}\StringTok{ }\DecValTok{25}\OperatorTok{*}\NormalTok{prob[}\DecValTok{1}\NormalTok{,}\DecValTok{1}\NormalTok{] }\OperatorTok{+}\StringTok{ }\DecValTok{50}\OperatorTok{*}\NormalTok{(prob[}\DecValTok{2}\NormalTok{,}\DecValTok{1}\NormalTok{]}\OperatorTok{{-}}\NormalTok{prob[}\DecValTok{1}\NormalTok{,}\DecValTok{1}\NormalTok{]) }\OperatorTok{+}\StringTok{ }
\StringTok{    }\DecValTok{75}\OperatorTok{*}\NormalTok{(prob[}\DecValTok{3}\NormalTok{,}\DecValTok{1}\NormalTok{]}\OperatorTok{{-}}\NormalTok{prob[}\DecValTok{2}\NormalTok{,}\DecValTok{1}\NormalTok{]) }\OperatorTok{+}\StringTok{ }\DecValTok{100}\OperatorTok{*}\NormalTok{(}\DecValTok{1}\OperatorTok{{-}}\NormalTok{prob[}\DecValTok{3}\NormalTok{,}\DecValTok{1}\NormalTok{])}
\NormalTok{  Ey0 \textless{}{-}}\StringTok{ }\DecValTok{25}\OperatorTok{*}\NormalTok{prob[}\DecValTok{1}\NormalTok{,}\DecValTok{2}\NormalTok{] }\OperatorTok{+}\StringTok{ }\DecValTok{50}\OperatorTok{*}\NormalTok{(prob[}\DecValTok{2}\NormalTok{,}\DecValTok{2}\NormalTok{]}\OperatorTok{{-}}\NormalTok{prob[}\DecValTok{1}\NormalTok{,}\DecValTok{2}\NormalTok{]) }\OperatorTok{+}\StringTok{ }
\StringTok{    }\DecValTok{75}\OperatorTok{*}\NormalTok{(prob[}\DecValTok{3}\NormalTok{,}\DecValTok{2}\NormalTok{]}\OperatorTok{{-}}\NormalTok{prob[}\DecValTok{2}\NormalTok{,}\DecValTok{2}\NormalTok{]) }\OperatorTok{+}\StringTok{ }\DecValTok{100}\OperatorTok{*}\NormalTok{(}\DecValTok{1}\OperatorTok{{-}}\NormalTok{prob[}\DecValTok{3}\NormalTok{,}\DecValTok{2}\NormalTok{])}
  
  \KeywordTok{return}\NormalTok{(Ey1 }\OperatorTok{{-}}\StringTok{ }\NormalTok{Ey0)}
\NormalTok{\} }

\NormalTok{ef\_oprobit \textless{}{-}}\StringTok{ }\KeywordTok{round}\NormalTok{(}\KeywordTok{quantef}\NormalTok{(oprobit), }\DecValTok{3}\NormalTok{)}
\NormalTok{ef\_ologit \textless{}{-}}\StringTok{ }\KeywordTok{round}\NormalTok{(}\KeywordTok{quantef}\NormalTok{(ologit), }\DecValTok{3}\NormalTok{)}
\end{Highlighting}
\end{Shaded}

As a result, we obtain similar values to OLSE. In the ordered probit
model, earners in the top 25 percentile live in 4.2\% higher floor than
others. In the ordered logit model, earners in the top 25 percentile
live in 5.9\% higher floor than others. Note that, in this application,
model fitness seems to be bad because the percent correctly predicted is
low (16.7\%).

\begin{Shaded}
\begin{Highlighting}[]
\KeywordTok{stargazer}\NormalTok{(}
\NormalTok{  ols, oprobit, ologit,}
  \DataTypeTok{report =} \StringTok{"vcstp"}\NormalTok{, }\DataTypeTok{keep.stat =} \KeywordTok{c}\NormalTok{(}\StringTok{"n"}\NormalTok{),}
  \DataTypeTok{omit =} \KeywordTok{c}\NormalTok{(}\StringTok{"Constant"}\NormalTok{),}
  \DataTypeTok{add.lines =} \KeywordTok{list}\NormalTok{(}
    \KeywordTok{c}\NormalTok{(}\StringTok{"Cutoff value at 1|2"}\NormalTok{, }\StringTok{""}\NormalTok{, a\_oprobit[}\DecValTok{1}\NormalTok{], a\_ologit[}\DecValTok{1}\NormalTok{]),}
    \KeywordTok{c}\NormalTok{(}\StringTok{"Cutoff value at 2|3"}\NormalTok{, }\StringTok{""}\NormalTok{, a\_oprobit[}\DecValTok{2}\NormalTok{], a\_ologit[}\DecValTok{2}\NormalTok{]),}
    \KeywordTok{c}\NormalTok{(}\StringTok{"Cutoff value at 3|4"}\NormalTok{, }\StringTok{""}\NormalTok{, a\_oprobit[}\DecValTok{3}\NormalTok{], a\_ologit[}\DecValTok{3}\NormalTok{]),}
    \KeywordTok{c}\NormalTok{(}\StringTok{"Quantitative Effect of Top25"}\NormalTok{, }\StringTok{""}\NormalTok{, ef\_oprobit, ef\_ologit),}
    \KeywordTok{c}\NormalTok{(}\StringTok{"Percent correctly predicted"}\NormalTok{, }\StringTok{""}\NormalTok{, pred\_oprobit, pred\_ologit)}
\NormalTok{  ),}
  \DataTypeTok{omit.table.layout =} \StringTok{"n"}\NormalTok{, }\DataTypeTok{table.placement =} \StringTok{"t"}\NormalTok{,}
  \DataTypeTok{title =} \StringTok{"Floor Level of House: Ordered Probit and Logit Model"}\NormalTok{,}
  \DataTypeTok{label =} \StringTok{"housing"}\NormalTok{,}
  \DataTypeTok{type =} \StringTok{"latex"}\NormalTok{, }\DataTypeTok{header =} \OtherTok{FALSE}
\NormalTok{)}
\end{Highlighting}
\end{Shaded}

\begin{table}[t] \centering 
  \caption{Floor Level of House: Ordered Probit and Logit Model} 
  \label{housing} 
\begin{tabular}{@{\extracolsep{5pt}}lccc} 
\\[-1.8ex]\hline 
\hline \\[-1.8ex] 
 & \multicolumn{3}{c}{\textit{Dependent variable:}} \\ 
\cline{2-4} 
\\[-1.8ex] & Levelnum & \multicolumn{2}{c}{Level} \\ 
\\[-1.8ex] & \textit{OLS} & \textit{ordered} & \textit{ordered} \\ 
 & \textit{} & \textit{probit} & \textit{logistic} \\ 
\\[-1.8ex] & (1) & (2) & (3)\\ 
\hline \\[-1.8ex] 
 lnPrice & 0.348 & 0.012 & 0.019 \\ 
  & (0.430) & (0.016) & (0.026) \\ 
  & t = 0.810 & t = 0.777 & t = 0.745 \\ 
  & p = 0.418 & p = 0.438 & p = 0.457 \\ 
  & & & \\ 
 Top25 & 3.714 & 0.156 & 0.239 \\ 
  & (1.723) & (0.064) & (0.106) \\ 
  & t = 2.156 & t = 2.426 & t = 2.259 \\ 
  & p = 0.032 & p = 0.016 & p = 0.024 \\ 
  & & & \\ 
\hline \\[-1.8ex] 
Cutoff value at 1|2 &  & -0.149 & -0.25 \\ 
Cutoff value at 2|3 &  & 0.246 & 0.384 \\ 
Cutoff value at 3|4 &  & 0.97 & 1.574 \\ 
Quantitative Effect of Top25 &  & 4.17 & 5.488 \\ 
Percent correctly predicted &  & 0.167 & 0.167 \\ 
Observations & 1,612 & 1,612 & 1,612 \\ 
\hline 
\hline \\[-1.8ex] 
\end{tabular} 
\end{table}

\hypertarget{empirical-application-of-multinomial-model-gender-discremination-in-job-position}{%
\section{Empirical Application of Multinomial Model: Gender
Discremination in Job
Position}\label{empirical-application-of-multinomial-model-gender-discremination-in-job-position}}

\textbf{Brief Background}. Recently, many developed countries move
toward women's social advancement, for example, an increase of number of
board member. In this application, we explore whether the U.S. bank
hindered the entrance of female into the workhorse.

\noindent \textbf{Data}. We use a built-in dataset called
\texttt{BankWages} in the library \texttt{AER}. This dataset contains
choice of three job position: \texttt{custodial}, \texttt{admin} and
\texttt{manage}. The rank of position is
\texttt{custodial\ \textless{}\ admin\ \textless{}\ manage}. Other
variables are \texttt{education}, \texttt{gender}, and
\texttt{minority}. We use former two variables as explanatory variables.

\begin{Shaded}
\begin{Highlighting}[]
\KeywordTok{library}\NormalTok{(AER)}
\KeywordTok{data}\NormalTok{(BankWages)}
\NormalTok{dt \textless{}{-}}\StringTok{ }\NormalTok{BankWages}
\NormalTok{dt}\OperatorTok{$}\NormalTok{job \textless{}{-}}\StringTok{ }\KeywordTok{as.character}\NormalTok{(dt}\OperatorTok{$}\NormalTok{job)}
\NormalTok{dt}\OperatorTok{$}\NormalTok{job \textless{}{-}}\StringTok{ }\KeywordTok{factor}\NormalTok{(dt}\OperatorTok{$}\NormalTok{job, }\DataTypeTok{levels =} \KeywordTok{c}\NormalTok{(}\StringTok{"admin"}\NormalTok{, }\StringTok{"custodial"}\NormalTok{, }\StringTok{"manage"}\NormalTok{))}
\KeywordTok{head}\NormalTok{(BankWages, }\DecValTok{5}\NormalTok{)}
\end{Highlighting}
\end{Shaded}

\begin{verbatim}
##      job education gender minority
## 1 manage        15   male       no
## 2  admin        16   male       no
## 3  admin        12 female       no
## 4  admin         8 female       no
## 5  admin        15   male       no
\end{verbatim}

\noindent \textbf{Model}. The outcome variable \(y_i\) takes three
values \(\{0, 1, 2\}\). Then, the multinomial logit model has the
following response probabilities \begin{equation*}
  P_{ij} = \mathbb{P}(y_i = j | \mathbf{x}_i) =
  \begin{cases}
    \frac{\exp(\mathbf{x}_i \beta_j)}{1 + \sum_{k=1}^2 \exp(\mathbf{x}_i \beta_k)} &\text{if}\quad j = 1, 2  \\
    \frac{1}{1 + \sum_{k=1}^2 \exp(\mathbf{x}_i \beta_k)}  &\text{if}\quad j = 0
  \end{cases}.
\end{equation*} The log-likelihood function is \begin{equation*}
  M_n(\beta_1, \beta_2) = \sum_{i=1}^n \sum_{j=0}^3 d_{ij} \log (P_{ij}),
\end{equation*} where \(d_{ij}\) is a dummy variable taking 1 if
\(y_i = j\).

In \texttt{R}, some packages provide the multinomial logit model. In
this application, we use the \texttt{multinom} function in the library
\texttt{nnet}.

\begin{Shaded}
\begin{Highlighting}[]
\KeywordTok{library}\NormalTok{(nnet)}
\NormalTok{est\_mlogit \textless{}{-}}\StringTok{ }\KeywordTok{multinom}\NormalTok{(job }\OperatorTok{\textasciitilde{}}\StringTok{ }\NormalTok{education }\OperatorTok{+}\StringTok{ }\NormalTok{gender, }\DataTypeTok{data =}\NormalTok{ dt)}

\CommentTok{\# observations and percent correctly predicted}
\NormalTok{pred \textless{}{-}}\StringTok{ }\NormalTok{est\_mlogit}\OperatorTok{$}\NormalTok{fitted.value}
\NormalTok{pred \textless{}{-}}\StringTok{ }\KeywordTok{colnames}\NormalTok{(pred)[}\KeywordTok{apply}\NormalTok{(pred, }\DecValTok{1}\NormalTok{, which.max)]}
\NormalTok{n \textless{}{-}}\StringTok{ }\KeywordTok{length}\NormalTok{(pred)}
\NormalTok{pcp \textless{}{-}}\StringTok{ }\KeywordTok{round}\NormalTok{(}\KeywordTok{sum}\NormalTok{(pred }\OperatorTok{==}\StringTok{ }\NormalTok{dt}\OperatorTok{$}\NormalTok{job)}\OperatorTok{/}\NormalTok{n, }\DecValTok{3}\NormalTok{)}

\CommentTok{\# Log{-}likelihood and pseudo R{-}sq}
\NormalTok{loglik1 \textless{}{-}}\StringTok{ }\KeywordTok{as.numeric}\NormalTok{(nnet}\OperatorTok{:::}\KeywordTok{logLik.multinom}\NormalTok{(est\_mlogit))}
\NormalTok{est\_mlogit0 \textless{}{-}}\StringTok{ }\KeywordTok{multinom}\NormalTok{(job }\OperatorTok{\textasciitilde{}}\StringTok{ }\DecValTok{1}\NormalTok{, }\DataTypeTok{data =}\NormalTok{ dt)}
\NormalTok{loglik0 \textless{}{-}}\StringTok{ }\KeywordTok{as.numeric}\NormalTok{(nnet}\OperatorTok{:::}\KeywordTok{logLik.multinom}\NormalTok{(est\_mlogit0))}
\NormalTok{pr2 \textless{}{-}}\StringTok{ }\KeywordTok{round}\NormalTok{(}\DecValTok{1} \OperatorTok{{-}}\StringTok{ }\NormalTok{loglik1}\OperatorTok{/}\NormalTok{loglik0, }\DecValTok{3}\NormalTok{)}
\end{Highlighting}
\end{Shaded}

\hypertarget{interpretations}{%
\subsection{Interpretations}\label{interpretations}}

Table \ref{job} summarizes the result of multinomial logit model. The
coefficient represents the change of \(\log(P_{ij}/P_{i0})\) in
corresponding covariate. For example, eduction decreases the log-odds
between \texttt{custodial} and \texttt{admin},
\(\log(P_{i, custodial}/P_{i, admin})\) by -0.562. This implies that
those who received higher education are more likely to obtain the
position \texttt{admin}. Highly-educated workers are also more likely to
obtain the position \texttt{manage}. Moreover, a female dummy decrease
the log-odds between \texttt{manage} and \texttt{admin} by -0.748, which
implies that females are less likely to obtain higher position
\texttt{manage}. From this result, we conclude that the U.S. bank
disencouraged females to assign higher job position.

Finally, we should check the model fitness. The predicted position is
the outcome with the highest estimated probability. The multinomial
logit model correctly predicts many cases (correction rate: 85.2\%).

\begin{Shaded}
\begin{Highlighting}[]
\KeywordTok{stargazer}\NormalTok{(}
\NormalTok{  est\_mlogit,}
  \DataTypeTok{covariate.labels =} \KeywordTok{c}\NormalTok{(}\StringTok{"Education"}\NormalTok{, }\StringTok{"Female = 1"}\NormalTok{),}
  \DataTypeTok{report =} \StringTok{"vcstp"}\NormalTok{, }\DataTypeTok{omit.stat =} \KeywordTok{c}\NormalTok{(}\StringTok{"aic"}\NormalTok{),}
  \DataTypeTok{add.lines =} \KeywordTok{list}\NormalTok{(}
    \KeywordTok{c}\NormalTok{(}\StringTok{"Observations"}\NormalTok{, n, }\StringTok{""}\NormalTok{),}
    \KeywordTok{c}\NormalTok{(}\StringTok{"Percent correctly predicted"}\NormalTok{, pcp, }\StringTok{""}\NormalTok{),}
    \KeywordTok{c}\NormalTok{(}\StringTok{"Log{-}likelihood"}\NormalTok{, }\KeywordTok{round}\NormalTok{(loglik1, }\DecValTok{3}\NormalTok{), }\StringTok{""}\NormalTok{),}
    \KeywordTok{c}\NormalTok{(}\StringTok{"Pseudo R{-}sq"}\NormalTok{, pr2, }\StringTok{""}\NormalTok{)}
\NormalTok{  ),}
  \DataTypeTok{omit.table.layout =} \StringTok{"n"}\NormalTok{, }\DataTypeTok{table.placement =} \StringTok{"t"}\NormalTok{,}
  \DataTypeTok{title =} \StringTok{"Multinomial Logit Model of Job Position"}\NormalTok{,}
  \DataTypeTok{label =} \StringTok{"job"}\NormalTok{,}
  \DataTypeTok{type =} \StringTok{"latex"}\NormalTok{, }\DataTypeTok{header =} \OtherTok{FALSE}  
\NormalTok{)}
\end{Highlighting}
\end{Shaded}

\begin{table}[t] \centering 
  \caption{Multinomial Logit Model of Job Position} 
  \label{job} 
\begin{tabular}{@{\extracolsep{5pt}}lcc} 
\\[-1.8ex]\hline 
\hline \\[-1.8ex] 
 & \multicolumn{2}{c}{\textit{Dependent variable:}} \\ 
\cline{2-3} 
\\[-1.8ex] & custodial & manage \\ 
\\[-1.8ex] & (1) & (2)\\ 
\hline \\[-1.8ex] 
 Education & $-$0.562 & 1.661 \\ 
  & (0.098) & (0.247) \\ 
  & t = $-$5.721 & t = 6.715 \\ 
  & p = 0.000 & p = 0.000 \\ 
  & & \\ 
 Female = 1 & $-$10.976 & $-$0.748 \\ 
  & (27.808) & (0.429) \\ 
  & t = $-$0.395 & t = $-$1.743 \\ 
  & p = 0.694 & p = 0.082 \\ 
  & & \\ 
 Constant & 5.030 & $-$26.730 \\ 
  & (1.130) & (3.874) \\ 
  & t = 4.450 & t = $-$6.899 \\ 
  & p = 0.00001 & p = 0.000 \\ 
  & & \\ 
\hline \\[-1.8ex] 
Observations & 474 &  \\ 
Percent correctly predicted & 0.852 &  \\ 
Log-likelihood & -144.928 &  \\ 
Pseudo R-sq & 0.546 &  \\ 
\hline 
\hline \\[-1.8ex] 
\end{tabular} 
\end{table}

\end{document}
