% Options for packages loaded elsewhere
\PassOptionsToPackage{unicode}{hyperref}
\PassOptionsToPackage{hyphens}{url}
%
\documentclass[
  12pt,
]{article}
\usepackage{lmodern}
\usepackage{amssymb,amsmath}
\usepackage{ifxetex,ifluatex}
\ifnum 0\ifxetex 1\fi\ifluatex 1\fi=0 % if pdftex
  \usepackage[T1]{fontenc}
  \usepackage[utf8]{inputenc}
  \usepackage{textcomp} % provide euro and other symbols
\else % if luatex or xetex
  \usepackage{unicode-math}
  \defaultfontfeatures{Scale=MatchLowercase}
  \defaultfontfeatures[\rmfamily]{Ligatures=TeX,Scale=1}
\fi
% Use upquote if available, for straight quotes in verbatim environments
\IfFileExists{upquote.sty}{\usepackage{upquote}}{}
\IfFileExists{microtype.sty}{% use microtype if available
  \usepackage[]{microtype}
  \UseMicrotypeSet[protrusion]{basicmath} % disable protrusion for tt fonts
}{}
\makeatletter
\@ifundefined{KOMAClassName}{% if non-KOMA class
  \IfFileExists{parskip.sty}{%
    \usepackage{parskip}
  }{% else
    \setlength{\parindent}{0pt}
    \setlength{\parskip}{6pt plus 2pt minus 1pt}}
}{% if KOMA class
  \KOMAoptions{parskip=half}}
\makeatother
\usepackage{xcolor}
\IfFileExists{xurl.sty}{\usepackage{xurl}}{} % add URL line breaks if available
\IfFileExists{bookmark.sty}{\usepackage{bookmark}}{\usepackage{hyperref}}
\hypersetup{
  pdftitle={Econometrics II TA Session \#3},
  pdfauthor={Hiroki Kato},
  hidelinks,
  pdfcreator={LaTeX via pandoc}}
\urlstyle{same} % disable monospaced font for URLs
\usepackage[margin=1in]{geometry}
\usepackage{color}
\usepackage{fancyvrb}
\newcommand{\VerbBar}{|}
\newcommand{\VERB}{\Verb[commandchars=\\\{\}]}
\DefineVerbatimEnvironment{Highlighting}{Verbatim}{commandchars=\\\{\}}
% Add ',fontsize=\small' for more characters per line
\usepackage{framed}
\definecolor{shadecolor}{RGB}{248,248,248}
\newenvironment{Shaded}{\begin{snugshade}}{\end{snugshade}}
\newcommand{\AlertTok}[1]{\textcolor[rgb]{0.94,0.16,0.16}{#1}}
\newcommand{\AnnotationTok}[1]{\textcolor[rgb]{0.56,0.35,0.01}{\textbf{\textit{#1}}}}
\newcommand{\AttributeTok}[1]{\textcolor[rgb]{0.77,0.63,0.00}{#1}}
\newcommand{\BaseNTok}[1]{\textcolor[rgb]{0.00,0.00,0.81}{#1}}
\newcommand{\BuiltInTok}[1]{#1}
\newcommand{\CharTok}[1]{\textcolor[rgb]{0.31,0.60,0.02}{#1}}
\newcommand{\CommentTok}[1]{\textcolor[rgb]{0.56,0.35,0.01}{\textit{#1}}}
\newcommand{\CommentVarTok}[1]{\textcolor[rgb]{0.56,0.35,0.01}{\textbf{\textit{#1}}}}
\newcommand{\ConstantTok}[1]{\textcolor[rgb]{0.00,0.00,0.00}{#1}}
\newcommand{\ControlFlowTok}[1]{\textcolor[rgb]{0.13,0.29,0.53}{\textbf{#1}}}
\newcommand{\DataTypeTok}[1]{\textcolor[rgb]{0.13,0.29,0.53}{#1}}
\newcommand{\DecValTok}[1]{\textcolor[rgb]{0.00,0.00,0.81}{#1}}
\newcommand{\DocumentationTok}[1]{\textcolor[rgb]{0.56,0.35,0.01}{\textbf{\textit{#1}}}}
\newcommand{\ErrorTok}[1]{\textcolor[rgb]{0.64,0.00,0.00}{\textbf{#1}}}
\newcommand{\ExtensionTok}[1]{#1}
\newcommand{\FloatTok}[1]{\textcolor[rgb]{0.00,0.00,0.81}{#1}}
\newcommand{\FunctionTok}[1]{\textcolor[rgb]{0.00,0.00,0.00}{#1}}
\newcommand{\ImportTok}[1]{#1}
\newcommand{\InformationTok}[1]{\textcolor[rgb]{0.56,0.35,0.01}{\textbf{\textit{#1}}}}
\newcommand{\KeywordTok}[1]{\textcolor[rgb]{0.13,0.29,0.53}{\textbf{#1}}}
\newcommand{\NormalTok}[1]{#1}
\newcommand{\OperatorTok}[1]{\textcolor[rgb]{0.81,0.36,0.00}{\textbf{#1}}}
\newcommand{\OtherTok}[1]{\textcolor[rgb]{0.56,0.35,0.01}{#1}}
\newcommand{\PreprocessorTok}[1]{\textcolor[rgb]{0.56,0.35,0.01}{\textit{#1}}}
\newcommand{\RegionMarkerTok}[1]{#1}
\newcommand{\SpecialCharTok}[1]{\textcolor[rgb]{0.00,0.00,0.00}{#1}}
\newcommand{\SpecialStringTok}[1]{\textcolor[rgb]{0.31,0.60,0.02}{#1}}
\newcommand{\StringTok}[1]{\textcolor[rgb]{0.31,0.60,0.02}{#1}}
\newcommand{\VariableTok}[1]{\textcolor[rgb]{0.00,0.00,0.00}{#1}}
\newcommand{\VerbatimStringTok}[1]{\textcolor[rgb]{0.31,0.60,0.02}{#1}}
\newcommand{\WarningTok}[1]{\textcolor[rgb]{0.56,0.35,0.01}{\textbf{\textit{#1}}}}
\usepackage{graphicx}
\makeatletter
\def\maxwidth{\ifdim\Gin@nat@width>\linewidth\linewidth\else\Gin@nat@width\fi}
\def\maxheight{\ifdim\Gin@nat@height>\textheight\textheight\else\Gin@nat@height\fi}
\makeatother
% Scale images if necessary, so that they will not overflow the page
% margins by default, and it is still possible to overwrite the defaults
% using explicit options in \includegraphics[width, height, ...]{}
\setkeys{Gin}{width=\maxwidth,height=\maxheight,keepaspectratio}
% Set default figure placement to htbp
\makeatletter
\def\fps@figure{htbp}
\makeatother
\setlength{\emergencystretch}{3em} % prevent overfull lines
\providecommand{\tightlist}{%
  \setlength{\itemsep}{0pt}\setlength{\parskip}{0pt}}
\setcounter{secnumdepth}{5}
\usepackage{zxjatype}
\setCJKmainfont[BoldFont = IPAゴシック]{IPA明朝}
\setCJKsansfont{IPAゴシック}
\setCJKmonofont{IPAゴシック}
\parindent = 1em
\newcommand{\argmax}{\mathop{\rm arg~max}\limits}
\newcommand{\argmin}{\mathop{\rm arg~min}\limits}
\DeclareMathOperator*{\plim}{plim}
\usepackage{xcolor}
\ifluatex
  \usepackage{selnolig}  % disable illegal ligatures
\fi

\title{Econometrics II TA Session \#3}
\author{Hiroki Kato}
\date{}

\begin{document}
\maketitle

\hypertarget{empirical-application-of-binary-model-racial-discrimination-in-court}{%
\section{Empirical Application of Binary Model: Racial Discrimination in
Court}\label{empirical-application-of-binary-model-racial-discrimination-in-court}}

\textbf{Brief Background}. Recently, in the U.S., anti-racism activities
called ``Black Lives Matter'' are getting hot. These activities stems
from the death of George Floyd, who was killed by a white police officer
on May 25, 2020. The empirical application of binary model investigates
whether the judgement of death penalty is based on race of defendant and
race of victim.

\noindent \textbf{Data}. The package \texttt{catdata} contains many
built-in dataets which include categorical variables. We use the
built-in dataset \texttt{deathpenalty} which is about the death-penalty
judgement of defendants in cases of multiple murders in Florida between
1976 and 1987.

\begin{Shaded}
\begin{Highlighting}[]
\NormalTok{dt \textless{}{-}}\StringTok{ }\KeywordTok{read.csv}\NormalTok{(}
  \DataTypeTok{file =} \StringTok{"./data/titanic.csv"}\NormalTok{, }
  \DataTypeTok{header =} \OtherTok{TRUE}\NormalTok{,  }\DataTypeTok{sep =} \StringTok{","}\NormalTok{, }\DataTypeTok{row.names =} \OtherTok{NULL}\NormalTok{,  }\DataTypeTok{stringsAsFactors =} \OtherTok{FALSE}\NormalTok{)}
\NormalTok{dt \textless{}{-}}\StringTok{ }\NormalTok{dt[,}\KeywordTok{c}\NormalTok{(}\StringTok{"survived"}\NormalTok{, }\StringTok{"age"}\NormalTok{, }\StringTok{"fare"}\NormalTok{, }\StringTok{"sex"}\NormalTok{)]}
\KeywordTok{head}\NormalTok{(dt)}
\end{Highlighting}
\end{Shaded}

\begin{verbatim}
##   survived   age     fare    sex
## 1        1 29.00 211.3375 female
## 2        1  0.92 151.5500   male
## 3        0  2.00 151.5500 female
## 4        0 30.00 151.5500   male
## 5        0 25.00 151.5500 female
## 6        1 48.00  26.5500   male
\end{verbatim}

This dataset contains three dummy variables

\begin{enumerate}
\def\labelenumi{\arabic{enumi}.}
\tightlist
\item
  \texttt{DeathPenalty} is a dummy variable taking 1 if the judgement is
  death penalty.
\item
  \texttt{VictimRace} is a dummy variable taking 1 if the race of the
  victim is white.
\item
  \texttt{DefendantRace} is a dummy variable taking 1 if the race of the
  defendant is white.
\end{enumerate}

This dataset aggregates observations with repect to
\texttt{DeathPenalty}, \texttt{VictimRace} and \texttt{DefendantRace}.
The variable \texttt{Freq} represents the number of observations. Since
it is inconvinient for us to use the original data for estimation, we
disaggregate this dataset. For example, we make 37 rows whose elements
are \texttt{DeathPenalty\ =\ 0}, \texttt{VictimRace\ =\ 1}, and
\texttt{DefendantRace\ =\ 0} because there are 37 obaservations, i.e.,
\texttt{Freq\ =\ 37}.

\begin{Shaded}
\begin{Highlighting}[]
\NormalTok{dt \textless{}{-}}\StringTok{ }\KeywordTok{subset}\NormalTok{(dt, }\OperatorTok{!}\KeywordTok{is.na}\NormalTok{(survived)}\OperatorTok{\&!}\KeywordTok{is.na}\NormalTok{(age)}\OperatorTok{\&!}\KeywordTok{is.na}\NormalTok{(fare)}\OperatorTok{\&!}\KeywordTok{is.na}\NormalTok{(sex))}
\NormalTok{dt}\OperatorTok{$}\NormalTok{female \textless{}{-}}\StringTok{ }\KeywordTok{ifelse}\NormalTok{(dt}\OperatorTok{$}\NormalTok{sex }\OperatorTok{==}\StringTok{ "female"}\NormalTok{, }\DecValTok{1}\NormalTok{, }\DecValTok{0}\NormalTok{)}
\end{Highlighting}
\end{Shaded}

\noindent \textbf{Model}. In a binary model, a dependent (outcome)
variable \(y_i\) takes only two values, i.e., \(y_i \in \{0, 1\}\). A
binary variable is sometimes called a \emph{dummy} variable. In this
application, the outcome variable is \texttt{DeathPenalty} taking 1 if
the judgement is death penalty. We make three explanatory variables.

\begin{enumerate}
\def\labelenumi{\arabic{enumi}.}
\tightlist
\item
  \texttt{WB} is a dummy variable taking 1 if the race of the victim and
  the defendant is white and black, respectively.
\item
  \texttt{BW} is a dummy variable taking 1 if the race of the victim and
  the defendant is black and white, respectively.
\item
  \texttt{WW} is a dummy variable taking 1 if the race of both the
  victim and the defendant is black.
\end{enumerate}

The regression function is \begin{equation*}
  \begin{split}
    &\mathbb{E}[DeathPenalty | WB, BW, WW] \\
    =& \mathbb{P}[DeathPenalty = 1 | WB, BW, WW]
    = G(\beta_0 + \beta_1 WB + \beta_2 BW + \beta3 WW).
  \end{split}
\end{equation*} The function \(G(\cdot)\) is arbitrary function. In
practice, we often use following three specifications:

\begin{itemize}
\tightlist
\item
  Linear probability model (LPM):
  \(G(\mathbf{x}_i \beta) = \mathbf{x}_i \beta\).
\item
  Probit model: \(G(\mathbf{x}_i \beta) = \Phi(\mathbf{x}_i \beta)\)
  where \(\Phi(\cdot)\) is the standard Gaussian cumulative function.
\item
  Logit model:
  \(G(\mathbf{x}_i \beta) = 1/(1 + \exp(-\mathbf{x}_i \beta))\).
\end{itemize}

\hypertarget{linear-probability-model}{%
\subsection{Linear Probability Model}\label{linear-probability-model}}

The linear probability model is \begin{equation*}
  \mathbb{P}[DeathPenalty = 1 | WB, BW, WW]
  = \beta_0 + \beta_1 WB + \beta_2 BW + \beta3 WW
\end{equation*} This model can be estimated using the OLS method. In
\texttt{R}, we can use the OLS method, running \texttt{lm()} function.

\begin{Shaded}
\begin{Highlighting}[]
\NormalTok{model \textless{}{-}}\StringTok{ }\NormalTok{survived }\OperatorTok{\textasciitilde{}}\StringTok{ }\NormalTok{female }\OperatorTok{+}\StringTok{ }\NormalTok{age }\OperatorTok{+}\StringTok{ }\NormalTok{fare}
\NormalTok{LPM \textless{}{-}}\StringTok{ }\KeywordTok{lm}\NormalTok{(model, }\DataTypeTok{data =}\NormalTok{ dt)}
\end{Highlighting}
\end{Shaded}

However, \texttt{lm()} function does not deal with heteroskedasticity
problem. To resolve it, we need to claculate heteroskedasticity-robust
standard errors using the White method. \begin{equation*}
  \hat{V}(\hat{\beta}) =
  \left( \frac{1}{n} \sum_i \mathbf{x}'_i \mathbf{x}_i  \right)^{-1}
  \left( \frac{1}{n} \sum_i \hat{u}_i^2 \mathbf{x}'_i \mathbf{x}_i \right)
  \left( \frac{1}{n} \sum_i \mathbf{x}'_i \mathbf{x}_i \right)^{-1}
\end{equation*}

\begin{Shaded}
\begin{Highlighting}[]
\CommentTok{\# heteroskedasticity{-}robust standard errors}
\NormalTok{dt}\OperatorTok{$}\StringTok{"(Intercept)"}\NormalTok{ \textless{}{-}}\StringTok{ }\DecValTok{1}
\NormalTok{X \textless{}{-}}\StringTok{ }\KeywordTok{as.matrix}\NormalTok{(dt[,}\KeywordTok{c}\NormalTok{(}\StringTok{"(Intercept)"}\NormalTok{, }\StringTok{"female"}\NormalTok{, }\StringTok{"age"}\NormalTok{, }\StringTok{"fare"}\NormalTok{)])}
\NormalTok{u \textless{}{-}}\StringTok{ }\KeywordTok{diag}\NormalTok{(LPM}\OperatorTok{$}\NormalTok{residuals}\OperatorTok{\^{}}\DecValTok{2}\NormalTok{)}

\NormalTok{XX \textless{}{-}}\StringTok{ }\KeywordTok{t}\NormalTok{(X) }\OperatorTok{\%*\%}\StringTok{ }\NormalTok{X}
\NormalTok{avgXX \textless{}{-}}\StringTok{ }\NormalTok{XX }\OperatorTok{*}\StringTok{ }\KeywordTok{nrow}\NormalTok{(X)}\OperatorTok{\^{}}\NormalTok{\{}\OperatorTok{{-}}\DecValTok{1}\NormalTok{\}}
\NormalTok{inv\_avgXX \textless{}{-}}\StringTok{ }\KeywordTok{solve}\NormalTok{(avgXX)}

\NormalTok{uXX \textless{}{-}}\StringTok{ }\KeywordTok{t}\NormalTok{(X) }\OperatorTok{\%*\%}\StringTok{ }\NormalTok{u }\OperatorTok{\%*\%}\StringTok{ }\NormalTok{X}
\NormalTok{avguXX \textless{}{-}}\StringTok{ }\NormalTok{uXX }\OperatorTok{*}\StringTok{ }\KeywordTok{nrow}\NormalTok{(X)}\OperatorTok{\^{}}\NormalTok{\{}\OperatorTok{{-}}\DecValTok{1}\NormalTok{\} }

\NormalTok{vcov\_b \textless{}{-}}\StringTok{ }\NormalTok{(inv\_avgXX }\OperatorTok{\%*\%}\StringTok{ }\NormalTok{avguXX }\OperatorTok{\%*\%}\StringTok{ }\NormalTok{inv\_avgXX) }\OperatorTok{*}\StringTok{ }\KeywordTok{nrow}\NormalTok{(X)}\OperatorTok{\^{}}\NormalTok{\{}\OperatorTok{{-}}\DecValTok{1}\NormalTok{\}}
\NormalTok{rse\_b \textless{}{-}}\StringTok{ }\KeywordTok{sqrt}\NormalTok{(}\KeywordTok{diag}\NormalTok{(vcov\_b))}

\CommentTok{\# homoskedasticity{-}based standard errors}
\NormalTok{se\_b \textless{}{-}}\StringTok{ }\KeywordTok{sqrt}\NormalTok{(}\KeywordTok{diag}\NormalTok{(}\KeywordTok{vcov}\NormalTok{(LPM)))}

\KeywordTok{print}\NormalTok{(}\StringTok{"The Variance of OLS"}\NormalTok{); }\KeywordTok{vcov}\NormalTok{(LPM)}
\end{Highlighting}
\end{Shaded}

\begin{verbatim}
## [1] "The Variance of OLS"
\end{verbatim}

\begin{verbatim}
##               (Intercept)        female           age          fare
## (Intercept)  9.754357e-04 -2.891381e-04 -2.333963e-05 -3.329763e-07
## female      -2.891381e-04  7.136865e-04  2.373259e-06 -1.272800e-06
## age         -2.333963e-05  2.373259e-06  8.026024e-07 -4.090649e-08
## fare        -3.329763e-07 -1.272800e-06 -4.090649e-08  5.524412e-08
\end{verbatim}

\begin{Shaded}
\begin{Highlighting}[]
\KeywordTok{print}\NormalTok{(}\StringTok{"The Robust variance of OLS"}\NormalTok{); vcov\_b}
\end{Highlighting}
\end{Shaded}

\begin{verbatim}
## [1] "The Robust variance of OLS"
\end{verbatim}

\begin{verbatim}
##               (Intercept)        female           age          fare
## (Intercept)  1.133289e-03 -2.798532e-04 -2.789675e-05  2.813843e-07
## female      -2.798532e-04  7.903766e-04  3.169092e-06 -2.401923e-06
## age         -2.789675e-05  3.169092e-06  8.857523e-07 -3.650375e-08
## fare         2.813843e-07 -2.401923e-06 -3.650375e-08  4.071639e-08
\end{verbatim}

\begin{Shaded}
\begin{Highlighting}[]
\KeywordTok{print}\NormalTok{(}\StringTok{"The Robust se using White method"}\NormalTok{); rse\_b}
\end{Highlighting}
\end{Shaded}

\begin{verbatim}
## [1] "The Robust se using White method"
\end{verbatim}

\begin{verbatim}
##  (Intercept)       female          age         fare 
## 0.0336643606 0.0281136372 0.0009411442 0.0002017830
\end{verbatim}

\begin{Shaded}
\begin{Highlighting}[]
\KeywordTok{print}\NormalTok{(}\StringTok{"The Robust t{-}value using White method"}\NormalTok{); }\KeywordTok{coef}\NormalTok{(LPM)}\OperatorTok{/}\NormalTok{rse\_b}
\end{Highlighting}
\end{Shaded}

\begin{verbatim}
## [1] "The Robust t-value using White method"
\end{verbatim}

\begin{verbatim}
## (Intercept)      female         age        fare 
##    6.482874   18.229508   -1.884168    7.162302
\end{verbatim}

Using the package \texttt{lmtest} and \texttt{sandwich} is the most
easiest way to calculate heteroskedasticity-robust standard errors and
\(t\)-statistics.

\begin{Shaded}
\begin{Highlighting}[]
\KeywordTok{library}\NormalTok{(lmtest) }\CommentTok{\#use function \textasciigrave{}coeftest\textasciigrave{}}
\KeywordTok{library}\NormalTok{(sandwich) }\CommentTok{\#use function \textasciigrave{}vcovHC\textasciigrave{}}
\KeywordTok{coeftest}\NormalTok{(LPM, }\DataTypeTok{vcov =} \KeywordTok{vcovHC}\NormalTok{(LPM, }\DataTypeTok{type =} \StringTok{"HC0"}\NormalTok{))[, }\StringTok{"Std. Error"}\NormalTok{]}
\end{Highlighting}
\end{Shaded}

\begin{verbatim}
##  (Intercept)       female          age         fare 
## 0.0336643606 0.0281136372 0.0009411442 0.0002017830
\end{verbatim}

\begin{Shaded}
\begin{Highlighting}[]
\KeywordTok{coeftest}\NormalTok{(LPM, }\DataTypeTok{vcov =} \KeywordTok{vcovHC}\NormalTok{(LPM, }\DataTypeTok{type =} \StringTok{"HC0"}\NormalTok{))[, }\StringTok{"t value"}\NormalTok{]}
\end{Highlighting}
\end{Shaded}

\begin{verbatim}
## (Intercept)      female         age        fare 
##    6.482874   18.229508   -1.884168    7.162302
\end{verbatim}

Finally, we obtain follwing results of linear probability model. We will
discuss interpretation of results and goodness-of-fit of LPM later.

\begin{Shaded}
\begin{Highlighting}[]
\CommentTok{\# t{-}stats}
\NormalTok{t\_b \textless{}{-}}\StringTok{ }\KeywordTok{coef}\NormalTok{(LPM)}\OperatorTok{/}\NormalTok{se\_b }
\NormalTok{rt\_b \textless{}{-}}\StringTok{ }\KeywordTok{coef}\NormalTok{(LPM)}\OperatorTok{/}\NormalTok{rse\_b}
\CommentTok{\# p{-}value Pr( \textgreater{} |t|)}
\NormalTok{p\_b \textless{}{-}}\StringTok{ }\KeywordTok{pt}\NormalTok{(}\KeywordTok{abs}\NormalTok{(t\_b), }\DataTypeTok{df =} \KeywordTok{nrow}\NormalTok{(X)}\OperatorTok{{-}}\KeywordTok{ncol}\NormalTok{(X), }\DataTypeTok{lower =} \OtherTok{FALSE}\NormalTok{)}\OperatorTok{*}\DecValTok{2}
\NormalTok{rp\_b \textless{}{-}}\StringTok{ }\KeywordTok{pt}\NormalTok{(}\KeywordTok{abs}\NormalTok{(rt\_b), }\DataTypeTok{df =} \KeywordTok{nrow}\NormalTok{(X)}\OperatorTok{{-}}\KeywordTok{ncol}\NormalTok{(X), }\DataTypeTok{lower =} \OtherTok{FALSE}\NormalTok{)}\OperatorTok{*}\DecValTok{2}

\KeywordTok{library}\NormalTok{(stargazer)}
\KeywordTok{stargazer}\NormalTok{(}
\NormalTok{  LPM, LPM,}
  \DataTypeTok{se =} \KeywordTok{list}\NormalTok{(se\_b, rse\_b), }\DataTypeTok{t =} \KeywordTok{list}\NormalTok{(t\_b, rt\_b), }\DataTypeTok{p =} \KeywordTok{list}\NormalTok{(p\_b, rp\_b),}
  \DataTypeTok{t.auto =} \OtherTok{FALSE}\NormalTok{, }\DataTypeTok{p.auto =} \OtherTok{FALSE}\NormalTok{,}
  \DataTypeTok{report =} \StringTok{"vcstp"}\NormalTok{, }\DataTypeTok{keep.stat =} \KeywordTok{c}\NormalTok{(}\StringTok{"n"}\NormalTok{),}
  \DataTypeTok{add.lines =} \KeywordTok{list}\NormalTok{(}
    \KeywordTok{c}\NormalTok{(}\StringTok{"Standard errors"}\NormalTok{, }\StringTok{"Homoskedasticity{-}based"}\NormalTok{, }\StringTok{"Heteroskedasticity{-}robust"}\NormalTok{)),}
  \DataTypeTok{title =} \StringTok{"Results of Linear Probability Model"}\NormalTok{,}
  \DataTypeTok{type =} \StringTok{"latex"}\NormalTok{, }\DataTypeTok{header =} \OtherTok{FALSE}\NormalTok{, }\DataTypeTok{font.size =} \StringTok{"small"}\NormalTok{,}
  \DataTypeTok{omit.table.layout =} \StringTok{"n"}
\NormalTok{)}
\end{Highlighting}
\end{Shaded}

\begin{table}[!htbp] \centering 
  \caption{Results of Linear Probability Model} 
  \label{} 
\small 
\begin{tabular}{@{\extracolsep{5pt}}lcc} 
\\[-1.8ex]\hline 
\hline \\[-1.8ex] 
 & \multicolumn{2}{c}{\textit{Dependent variable:}} \\ 
\cline{2-3} 
\\[-1.8ex] & \multicolumn{2}{c}{survived} \\ 
\\[-1.8ex] & (1) & (2)\\ 
\hline \\[-1.8ex] 
 female & 0.512 & 0.512 \\ 
  & (0.027) & (0.028) \\ 
  & t = 19.184 & t = 18.230 \\ 
  & p = 0.000 & p = 0.000 \\ 
  & & \\ 
 age & $-$0.002 & $-$0.002 \\ 
  & (0.001) & (0.001) \\ 
  & t = $-$1.979 & t = $-$1.884 \\ 
  & p = 0.049 & p = 0.060 \\ 
  & & \\ 
 fare & 0.001 & 0.001 \\ 
  & (0.0002) & (0.0002) \\ 
  & t = 6.149 & t = 7.162 \\ 
  & p = 0.000 & p = 0.000 \\ 
  & & \\ 
 Constant & 0.218 & 0.218 \\ 
  & (0.031) & (0.034) \\ 
  & t = 6.988 & t = 6.483 \\ 
  & p = 0.000 & p = 0.000 \\ 
  & & \\ 
\hline \\[-1.8ex] 
Standard errors & Homoskedasticity-based & Heteroskedasticity-robust \\ 
Observations & 1,045 & 1,045 \\ 
\hline 
\hline \\[-1.8ex] 
\end{tabular} 
\end{table}

\end{document}
