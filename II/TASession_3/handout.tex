% Options for packages loaded elsewhere
\PassOptionsToPackage{unicode}{hyperref}
\PassOptionsToPackage{hyphens}{url}
%
\documentclass[
  12pt,
]{article}
\usepackage{lmodern}
\usepackage{amssymb,amsmath}
\usepackage{ifxetex,ifluatex}
\ifnum 0\ifxetex 1\fi\ifluatex 1\fi=0 % if pdftex
  \usepackage[T1]{fontenc}
  \usepackage[utf8]{inputenc}
  \usepackage{textcomp} % provide euro and other symbols
\else % if luatex or xetex
  \usepackage{unicode-math}
  \defaultfontfeatures{Scale=MatchLowercase}
  \defaultfontfeatures[\rmfamily]{Ligatures=TeX,Scale=1}
\fi
% Use upquote if available, for straight quotes in verbatim environments
\IfFileExists{upquote.sty}{\usepackage{upquote}}{}
\IfFileExists{microtype.sty}{% use microtype if available
  \usepackage[]{microtype}
  \UseMicrotypeSet[protrusion]{basicmath} % disable protrusion for tt fonts
}{}
\makeatletter
\@ifundefined{KOMAClassName}{% if non-KOMA class
  \IfFileExists{parskip.sty}{%
    \usepackage{parskip}
  }{% else
    \setlength{\parindent}{0pt}
    \setlength{\parskip}{6pt plus 2pt minus 1pt}}
}{% if KOMA class
  \KOMAoptions{parskip=half}}
\makeatother
\usepackage{xcolor}
\IfFileExists{xurl.sty}{\usepackage{xurl}}{} % add URL line breaks if available
\IfFileExists{bookmark.sty}{\usepackage{bookmark}}{\usepackage{hyperref}}
\hypersetup{
  pdftitle={Econometrics II TA Session \#3},
  pdfauthor={Hiroki Kato},
  hidelinks,
  pdfcreator={LaTeX via pandoc}}
\urlstyle{same} % disable monospaced font for URLs
\usepackage[margin=1in]{geometry}
\usepackage{color}
\usepackage{fancyvrb}
\newcommand{\VerbBar}{|}
\newcommand{\VERB}{\Verb[commandchars=\\\{\}]}
\DefineVerbatimEnvironment{Highlighting}{Verbatim}{commandchars=\\\{\}}
% Add ',fontsize=\small' for more characters per line
\usepackage{framed}
\definecolor{shadecolor}{RGB}{248,248,248}
\newenvironment{Shaded}{\begin{snugshade}}{\end{snugshade}}
\newcommand{\AlertTok}[1]{\textcolor[rgb]{0.94,0.16,0.16}{#1}}
\newcommand{\AnnotationTok}[1]{\textcolor[rgb]{0.56,0.35,0.01}{\textbf{\textit{#1}}}}
\newcommand{\AttributeTok}[1]{\textcolor[rgb]{0.77,0.63,0.00}{#1}}
\newcommand{\BaseNTok}[1]{\textcolor[rgb]{0.00,0.00,0.81}{#1}}
\newcommand{\BuiltInTok}[1]{#1}
\newcommand{\CharTok}[1]{\textcolor[rgb]{0.31,0.60,0.02}{#1}}
\newcommand{\CommentTok}[1]{\textcolor[rgb]{0.56,0.35,0.01}{\textit{#1}}}
\newcommand{\CommentVarTok}[1]{\textcolor[rgb]{0.56,0.35,0.01}{\textbf{\textit{#1}}}}
\newcommand{\ConstantTok}[1]{\textcolor[rgb]{0.00,0.00,0.00}{#1}}
\newcommand{\ControlFlowTok}[1]{\textcolor[rgb]{0.13,0.29,0.53}{\textbf{#1}}}
\newcommand{\DataTypeTok}[1]{\textcolor[rgb]{0.13,0.29,0.53}{#1}}
\newcommand{\DecValTok}[1]{\textcolor[rgb]{0.00,0.00,0.81}{#1}}
\newcommand{\DocumentationTok}[1]{\textcolor[rgb]{0.56,0.35,0.01}{\textbf{\textit{#1}}}}
\newcommand{\ErrorTok}[1]{\textcolor[rgb]{0.64,0.00,0.00}{\textbf{#1}}}
\newcommand{\ExtensionTok}[1]{#1}
\newcommand{\FloatTok}[1]{\textcolor[rgb]{0.00,0.00,0.81}{#1}}
\newcommand{\FunctionTok}[1]{\textcolor[rgb]{0.00,0.00,0.00}{#1}}
\newcommand{\ImportTok}[1]{#1}
\newcommand{\InformationTok}[1]{\textcolor[rgb]{0.56,0.35,0.01}{\textbf{\textit{#1}}}}
\newcommand{\KeywordTok}[1]{\textcolor[rgb]{0.13,0.29,0.53}{\textbf{#1}}}
\newcommand{\NormalTok}[1]{#1}
\newcommand{\OperatorTok}[1]{\textcolor[rgb]{0.81,0.36,0.00}{\textbf{#1}}}
\newcommand{\OtherTok}[1]{\textcolor[rgb]{0.56,0.35,0.01}{#1}}
\newcommand{\PreprocessorTok}[1]{\textcolor[rgb]{0.56,0.35,0.01}{\textit{#1}}}
\newcommand{\RegionMarkerTok}[1]{#1}
\newcommand{\SpecialCharTok}[1]{\textcolor[rgb]{0.00,0.00,0.00}{#1}}
\newcommand{\SpecialStringTok}[1]{\textcolor[rgb]{0.31,0.60,0.02}{#1}}
\newcommand{\StringTok}[1]{\textcolor[rgb]{0.31,0.60,0.02}{#1}}
\newcommand{\VariableTok}[1]{\textcolor[rgb]{0.00,0.00,0.00}{#1}}
\newcommand{\VerbatimStringTok}[1]{\textcolor[rgb]{0.31,0.60,0.02}{#1}}
\newcommand{\WarningTok}[1]{\textcolor[rgb]{0.56,0.35,0.01}{\textbf{\textit{#1}}}}
\usepackage{graphicx}
\makeatletter
\def\maxwidth{\ifdim\Gin@nat@width>\linewidth\linewidth\else\Gin@nat@width\fi}
\def\maxheight{\ifdim\Gin@nat@height>\textheight\textheight\else\Gin@nat@height\fi}
\makeatother
% Scale images if necessary, so that they will not overflow the page
% margins by default, and it is still possible to overwrite the defaults
% using explicit options in \includegraphics[width, height, ...]{}
\setkeys{Gin}{width=\maxwidth,height=\maxheight,keepaspectratio}
% Set default figure placement to htbp
\makeatletter
\def\fps@figure{htbp}
\makeatother
\setlength{\emergencystretch}{3em} % prevent overfull lines
\providecommand{\tightlist}{%
  \setlength{\itemsep}{0pt}\setlength{\parskip}{0pt}}
\setcounter{secnumdepth}{5}
\usepackage{zxjatype}
\setCJKmainfont[BoldFont = IPAゴシック]{IPA明朝}
\setCJKsansfont{IPAゴシック}
\setCJKmonofont{IPAゴシック}
\parindent = 1em
\newcommand{\argmax}{\mathop{\rm arg~max}\limits}
\newcommand{\argmin}{\mathop{\rm arg~min}\limits}
\DeclareMathOperator*{\plim}{plim}
\usepackage{xcolor}
\ifluatex
  \usepackage{selnolig}  % disable illegal ligatures
\fi

\title{Econometrics II TA Session \#3}
\author{Hiroki Kato}
\date{}

\begin{document}
\maketitle

\hypertarget{empirical-application-of-binary-model-racial-discrimination-in-court}{%
\section{Empirical Application of Binary Model: Racial Discrimination in
Court}\label{empirical-application-of-binary-model-racial-discrimination-in-court}}

\textbf{Brief Background}. Recently, in the U.S., anti-racism activities
called ``Black Lives Matter'' are getting hot. These activities stems
from the death of George Floyd, who was killed by a white police officer
on May 25, 2020. The empirical application of binary model investigates
whether the judgement of death penalty is based on race of defendant and
race of victim.

\noindent \textbf{Data}. The package \texttt{catdata} contains many
built-in dataets which include categorical variables. We use the
built-in dataset \texttt{deathpenalty} which is about the death-penalty
judgement of defendants in cases of multiple murders in Florida between
1976 and 1987.

\begin{Shaded}
\begin{Highlighting}[]
\CommentTok{\# If there is no package called \textquotesingle{}catdata\textquotesingle{}, run \textquotesingle{}install.packages("catdata")\textquotesingle{}}
\CommentTok{\# After that run following codes}
\KeywordTok{library}\NormalTok{(catdata)}
\KeywordTok{data}\NormalTok{(deathpenalty)}
\NormalTok{deathpenalty}
\end{Highlighting}
\end{Shaded}

\begin{verbatim}
##   DeathPenalty VictimRace DefendantRace Freq
## 1            0          0             0  139
## 2            1          0             0    4
## 3            0          1             0   37
## 4            1          1             0   11
## 5            0          0             1   16
## 6            1          0             1    0
## 7            0          1             1  414
## 8            1          1             1   53
\end{verbatim}

Since this datasets is aggregated with repect to \texttt{DeathPenalty},
\texttt{VictimRace} and \texttt{DefendantRace}, we disaggregate it. For
example, we make 37 rows whose elements are \texttt{DeathPenalty\ =\ 0},
\texttt{VictimRace\ =\ 1}, and \texttt{DefendantRace\ =\ 0} because
there are 37 obaservations, i.e., \texttt{Freq\ =\ 37}.

\begin{Shaded}
\begin{Highlighting}[]
\NormalTok{dt \textless{}{-}}\StringTok{ }\NormalTok{deathpenalty}
\NormalTok{dt \textless{}{-}}\StringTok{ }\NormalTok{dt[}\KeywordTok{rep}\NormalTok{(}\KeywordTok{seq\_len}\NormalTok{(}\KeywordTok{nrow}\NormalTok{(dt)), dt[,}\StringTok{"Freq"}\NormalTok{]), }\DecValTok{{-}4}\NormalTok{]}
\end{Highlighting}
\end{Shaded}

\noindent \textbf{Model}. In a binary model, a dependent (outcome)
variable \(y_i\) takes only two values, i.e., \(y_i \in \{0, 1\}\). A
binary variable is sometimes called a \emph{dummy} variable. In this
application, the outcome variable is \texttt{DeathPenalty} taking 1 if
the judgement is death penalty. There are two explanatory variables.
First, \texttt{VictimRace} is a dummy variable taking 1 if the race of
the victim is white. Second, \texttt{DefendantRace} is a dummy variable
taking 1 if the race of the defendant is white. The regression function
is \begin{equation}
  \begin{split}
    &\mathbb{E}[DeathP | Vrace, Drace] \\
    =& \mathbb{P}[DeathP = 1 | Vrace, Drace]
    = G(\beta_0 + \beta_1 Vrace + \beta_2 Drace).
  \end{split}
\end{equation} The function \(G(\cdot)\) is arbitrary function. In
practice, we often use following three specifications:

\begin{itemize}
\tightlist
\item
  Linear probability model (LPM):
  \(G(\mathbf{x}_i \beta) = \mathbf{x}_i \beta\).
\item
  Probit model: \(G(\mathbf{x}_i \beta) = \Phi(\mathbf{x}_i \beta)\)
  where \(\Phi(\cdot)\) is the standard Gaussian cumulative function.
\item
  Logit model:
  \(G(\mathbf{x}_i \beta) = 1/(1 + \exp(-\mathbf{x}_i \beta))\).
\end{itemize}

\end{document}
