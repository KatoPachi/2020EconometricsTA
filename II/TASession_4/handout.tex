% Options for packages loaded elsewhere
\PassOptionsToPackage{unicode}{hyperref}
\PassOptionsToPackage{hyphens}{url}
%
\documentclass[
  12pt,
]{article}
\usepackage{lmodern}
\usepackage{amssymb,amsmath}
\usepackage{ifxetex,ifluatex}
\ifnum 0\ifxetex 1\fi\ifluatex 1\fi=0 % if pdftex
  \usepackage[T1]{fontenc}
  \usepackage[utf8]{inputenc}
  \usepackage{textcomp} % provide euro and other symbols
\else % if luatex or xetex
  \usepackage{unicode-math}
  \defaultfontfeatures{Scale=MatchLowercase}
  \defaultfontfeatures[\rmfamily]{Ligatures=TeX,Scale=1}
\fi
% Use upquote if available, for straight quotes in verbatim environments
\IfFileExists{upquote.sty}{\usepackage{upquote}}{}
\IfFileExists{microtype.sty}{% use microtype if available
  \usepackage[]{microtype}
  \UseMicrotypeSet[protrusion]{basicmath} % disable protrusion for tt fonts
}{}
\makeatletter
\@ifundefined{KOMAClassName}{% if non-KOMA class
  \IfFileExists{parskip.sty}{%
    \usepackage{parskip}
  }{% else
    \setlength{\parindent}{0pt}
    \setlength{\parskip}{6pt plus 2pt minus 1pt}}
}{% if KOMA class
  \KOMAoptions{parskip=half}}
\makeatother
\usepackage{xcolor}
\IfFileExists{xurl.sty}{\usepackage{xurl}}{} % add URL line breaks if available
\IfFileExists{bookmark.sty}{\usepackage{bookmark}}{\usepackage{hyperref}}
\hypersetup{
  pdftitle={Econometrics II TA Session \#4},
  pdfauthor={Hiroki Kato},
  hidelinks,
  pdfcreator={LaTeX via pandoc}}
\urlstyle{same} % disable monospaced font for URLs
\usepackage[margin=1in]{geometry}
\usepackage{color}
\usepackage{fancyvrb}
\newcommand{\VerbBar}{|}
\newcommand{\VERB}{\Verb[commandchars=\\\{\}]}
\DefineVerbatimEnvironment{Highlighting}{Verbatim}{commandchars=\\\{\}}
% Add ',fontsize=\small' for more characters per line
\usepackage{framed}
\definecolor{shadecolor}{RGB}{248,248,248}
\newenvironment{Shaded}{\begin{snugshade}}{\end{snugshade}}
\newcommand{\AlertTok}[1]{\textcolor[rgb]{0.94,0.16,0.16}{#1}}
\newcommand{\AnnotationTok}[1]{\textcolor[rgb]{0.56,0.35,0.01}{\textbf{\textit{#1}}}}
\newcommand{\AttributeTok}[1]{\textcolor[rgb]{0.77,0.63,0.00}{#1}}
\newcommand{\BaseNTok}[1]{\textcolor[rgb]{0.00,0.00,0.81}{#1}}
\newcommand{\BuiltInTok}[1]{#1}
\newcommand{\CharTok}[1]{\textcolor[rgb]{0.31,0.60,0.02}{#1}}
\newcommand{\CommentTok}[1]{\textcolor[rgb]{0.56,0.35,0.01}{\textit{#1}}}
\newcommand{\CommentVarTok}[1]{\textcolor[rgb]{0.56,0.35,0.01}{\textbf{\textit{#1}}}}
\newcommand{\ConstantTok}[1]{\textcolor[rgb]{0.00,0.00,0.00}{#1}}
\newcommand{\ControlFlowTok}[1]{\textcolor[rgb]{0.13,0.29,0.53}{\textbf{#1}}}
\newcommand{\DataTypeTok}[1]{\textcolor[rgb]{0.13,0.29,0.53}{#1}}
\newcommand{\DecValTok}[1]{\textcolor[rgb]{0.00,0.00,0.81}{#1}}
\newcommand{\DocumentationTok}[1]{\textcolor[rgb]{0.56,0.35,0.01}{\textbf{\textit{#1}}}}
\newcommand{\ErrorTok}[1]{\textcolor[rgb]{0.64,0.00,0.00}{\textbf{#1}}}
\newcommand{\ExtensionTok}[1]{#1}
\newcommand{\FloatTok}[1]{\textcolor[rgb]{0.00,0.00,0.81}{#1}}
\newcommand{\FunctionTok}[1]{\textcolor[rgb]{0.00,0.00,0.00}{#1}}
\newcommand{\ImportTok}[1]{#1}
\newcommand{\InformationTok}[1]{\textcolor[rgb]{0.56,0.35,0.01}{\textbf{\textit{#1}}}}
\newcommand{\KeywordTok}[1]{\textcolor[rgb]{0.13,0.29,0.53}{\textbf{#1}}}
\newcommand{\NormalTok}[1]{#1}
\newcommand{\OperatorTok}[1]{\textcolor[rgb]{0.81,0.36,0.00}{\textbf{#1}}}
\newcommand{\OtherTok}[1]{\textcolor[rgb]{0.56,0.35,0.01}{#1}}
\newcommand{\PreprocessorTok}[1]{\textcolor[rgb]{0.56,0.35,0.01}{\textit{#1}}}
\newcommand{\RegionMarkerTok}[1]{#1}
\newcommand{\SpecialCharTok}[1]{\textcolor[rgb]{0.00,0.00,0.00}{#1}}
\newcommand{\SpecialStringTok}[1]{\textcolor[rgb]{0.31,0.60,0.02}{#1}}
\newcommand{\StringTok}[1]{\textcolor[rgb]{0.31,0.60,0.02}{#1}}
\newcommand{\VariableTok}[1]{\textcolor[rgb]{0.00,0.00,0.00}{#1}}
\newcommand{\VerbatimStringTok}[1]{\textcolor[rgb]{0.31,0.60,0.02}{#1}}
\newcommand{\WarningTok}[1]{\textcolor[rgb]{0.56,0.35,0.01}{\textbf{\textit{#1}}}}
\usepackage{graphicx}
\makeatletter
\def\maxwidth{\ifdim\Gin@nat@width>\linewidth\linewidth\else\Gin@nat@width\fi}
\def\maxheight{\ifdim\Gin@nat@height>\textheight\textheight\else\Gin@nat@height\fi}
\makeatother
% Scale images if necessary, so that they will not overflow the page
% margins by default, and it is still possible to overwrite the defaults
% using explicit options in \includegraphics[width, height, ...]{}
\setkeys{Gin}{width=\maxwidth,height=\maxheight,keepaspectratio}
% Set default figure placement to htbp
\makeatletter
\def\fps@figure{htbp}
\makeatother
\setlength{\emergencystretch}{3em} % prevent overfull lines
\providecommand{\tightlist}{%
  \setlength{\itemsep}{0pt}\setlength{\parskip}{0pt}}
\setcounter{secnumdepth}{5}
\usepackage{zxjatype}
\setCJKmainfont[BoldFont = IPAゴシック]{IPA明朝}
\setCJKsansfont{IPAゴシック}
\setCJKmonofont{IPAゴシック}
\parindent = 1em
\newcommand{\argmax}{\mathop{\rm arg~max}\limits}
\newcommand{\argmin}{\mathop{\rm arg~min}\limits}
\DeclareMathOperator*{\plim}{plim}
\usepackage{xcolor}
\ifluatex
  \usepackage{selnolig}  % disable illegal ligatures
\fi

\title{Econometrics II TA Session \#4}
\author{Hiroki Kato}
\date{}

\begin{document}
\maketitle

\hypertarget{empirical-application-of-ordered-probit-and-logit-model-housing-as-status-goods}{%
\section{Empirical Application of Ordered Probit and Logit Model:
Housing as Status
Goods}\label{empirical-application-of-ordered-probit-and-logit-model-housing-as-status-goods}}

\textbf{Breif Background}. Social image may affect consumption behavior.
Specifically, a desire to signal high income or wealth may cause
consumers to purchase status goods. In this application, we explore
whether living in an upper floor serves as a status goods.

\noindent \textbf{Data}. We use the housing data originally coming from
the American Housing Survey conducted in 2013 \footnote{\url{https://www.census.gov/programs-surveys/ahs.html}.
  This is a repeated cross-section survey. We use the data at one time.}.
We use the following variable

\begin{itemize}
\tightlist
\item
  \texttt{Level}: ordered value of a story of respondent's living (1:Low
  - 4:High)
\item
  \texttt{Levelnum}: variable we recode the response \texttt{Level} as
  25, 50, 75, 100. This represents the extent of floor height.
\item
  \texttt{lnPrice}: logged price of housing (proxy for quality of house)
\item
  \texttt{Top25}: a dummy variable taking one if household income is in
  the top 25 percentile in sample.
\end{itemize}

\begin{Shaded}
\begin{Highlighting}[]
\NormalTok{house \textless{}{-}}\StringTok{ }\KeywordTok{read.csv}\NormalTok{(}\DataTypeTok{file =} \StringTok{"./data/housing.csv"}\NormalTok{, }\DataTypeTok{header =} \OtherTok{TRUE}\NormalTok{,  }\DataTypeTok{sep =} \StringTok{","}\NormalTok{)}
\NormalTok{house \textless{}{-}}\StringTok{ }\NormalTok{house[,}\KeywordTok{c}\NormalTok{(}\StringTok{"Level"}\NormalTok{, }\StringTok{"lnPrice"}\NormalTok{, }\StringTok{"Top25"}\NormalTok{)]}
\NormalTok{house}\OperatorTok{$}\NormalTok{Levelnum \textless{}{-}}\StringTok{ }\KeywordTok{ifelse}\NormalTok{(}
\NormalTok{  house}\OperatorTok{$}\NormalTok{Level }\OperatorTok{==}\StringTok{ }\DecValTok{1}\NormalTok{, }\DecValTok{25}\NormalTok{, }
  \KeywordTok{ifelse}\NormalTok{(house}\OperatorTok{$}\NormalTok{Level }\OperatorTok{==}\StringTok{ }\DecValTok{2}\NormalTok{, }\DecValTok{50}\NormalTok{, }
  \KeywordTok{ifelse}\NormalTok{(house}\OperatorTok{$}\NormalTok{Level }\OperatorTok{==}\StringTok{ }\DecValTok{3}\NormalTok{, }\DecValTok{75}\NormalTok{, }\DecValTok{100}\NormalTok{)))}
\KeywordTok{head}\NormalTok{(house)}
\end{Highlighting}
\end{Shaded}

\begin{verbatim}
##   Level  lnPrice Top25 Levelnum
## 1     3 11.51294     0       75
## 2     4 11.51294     1      100
## 3     3 11.60824     0       75
## 4     3 11.69526     0       75
## 5     3 12.57764     0       75
## 6     3 12.64433     0       75
\end{verbatim}

\noindent \textbf{Model}. The outcome variable is \texttt{Level} taking
\(\{1, 2, 3, 4\}\). Consider the following regression equation of a
latent variable: \begin{equation*}
  y_i^* = \mathbf{x}_i \beta + u_i,
\end{equation*} where \(\mathbf{x}_i = (lnPrice, Top25)\) and \(u_i\) is
an error term. The relationship between the latent variable \(y_i^*\)
and the observed outcome variable is \begin{equation*}
  Level =
  \begin{cases}
    1 &\text{if}\quad -\infty < y_i^* \le a_1  \\
    2 &\text{if}\quad a_1 < y_i^* \le a_2 \\
    3 &\text{if}\quad a_2 < y_i^* \le a_3 \\
    4 &\text{if}\quad a_3 < y_i^* < +\infty
  \end{cases}.
\end{equation*}

Consider the probability of realization of \(y_i\), that is,
\begin{equation*}
  \begin{split}
  \mathbb{P}(y_i = k | \mathbf{x}_i) 
  &= \mathbb{P}(a_{k-1} - \mathbf{x}_i \beta < u_i \le a_k - \mathbf{x}_i \beta | \mathbf{x}_i)  \\
  &= G(a_k - \mathbf{x}_i \beta) - G(a_{k-1} - \mathbf{x}_i \beta),
  \end{split}
\end{equation*} where \(a_{4} = +\infty\) and \(a_0 = -\infty\). Then,
the likelihood function is defined by \begin{equation*}
  p((y_i|\mathbf{x}_i), i = 1, \ldots, n; \beta, a_1, \ldots, a_3)
  = \prod_{i=1}^n \prod_{k=1}^4 (G(a_k - \mathbf{x}_i \beta) - G(a_{k-1} - \mathbf{x}_i \beta))^{I_{ik}}.
\end{equation*} where \(I_{ik}\) is a indicator variable taking 1 if
\(y_i = k\). Finally, the log-likelihood function is \begin{equation*}
  M(\beta, a_1, a_2, a_3) = \sum_{i=1}^n \sum_{k=1}^4 I_{ik} \log(G(a_k - \mathbf{x}_i \beta) - G(a_{k-1} - \mathbf{x}_i \beta)).
\end{equation*} Usually, \(G(a)\) assumes the standard normal
distribution, \(\Phi(a)\), or the logistic distribution,
\(1/(1 + \exp(-a))\).

In \texttt{R}, the library (package) \texttt{MASS} provides the
\texttt{polr} function which estimates the ordered probit and logit
model. Although we can use the \texttt{nlm} function when we define the
log-likelihood function, we do not report this method. To compare
results, we use the variable \texttt{Levelnum} as outcome variable, and
apply the linear regression model.

\begin{Shaded}
\begin{Highlighting}[]
\KeywordTok{library}\NormalTok{(MASS)}
\KeywordTok{library}\NormalTok{(tidyverse) }\CommentTok{\#use case\_when()}

\NormalTok{ols \textless{}{-}}\StringTok{ }\KeywordTok{lm}\NormalTok{(Levelnum }\OperatorTok{\textasciitilde{}}\StringTok{ }\NormalTok{lnPrice }\OperatorTok{+}\StringTok{ }\NormalTok{Top25, }\DataTypeTok{data =}\NormalTok{ house)}

\NormalTok{model \textless{}{-}}\StringTok{ }\KeywordTok{factor}\NormalTok{(Level) }\OperatorTok{\textasciitilde{}}\StringTok{ }\NormalTok{lnPrice }\OperatorTok{+}\StringTok{ }\NormalTok{Top25}
\NormalTok{oprobit \textless{}{-}}\StringTok{ }\KeywordTok{polr}\NormalTok{(model, }\DataTypeTok{data =}\NormalTok{ house, }\DataTypeTok{method =} \StringTok{"probit"}\NormalTok{)}
\NormalTok{ologit \textless{}{-}}\StringTok{ }\KeywordTok{polr}\NormalTok{(model, }\DataTypeTok{data =}\NormalTok{ house, }\DataTypeTok{method =} \StringTok{"logistic"}\NormalTok{)}

\NormalTok{a\_oprobit \textless{}{-}}\StringTok{ }\KeywordTok{round}\NormalTok{(oprobit}\OperatorTok{$}\NormalTok{zeta, }\DecValTok{3}\NormalTok{)}
\NormalTok{a\_ologit \textless{}{-}}\StringTok{ }\KeywordTok{round}\NormalTok{(ologit}\OperatorTok{$}\NormalTok{zeta, }\DecValTok{3}\NormalTok{)}

\NormalTok{xb\_oprobit \textless{}{-}}\StringTok{ }\NormalTok{oprobit}\OperatorTok{$}\NormalTok{lp }
\NormalTok{xb\_ologit \textless{}{-}}\StringTok{ }\NormalTok{ologit}\OperatorTok{$}\NormalTok{lp}

\NormalTok{hatY\_oprobit \textless{}{-}}\StringTok{ }\KeywordTok{case\_when}\NormalTok{(}
\NormalTok{  xb\_oprobit }\OperatorTok{\textless{}=}\StringTok{ }\NormalTok{oprobit}\OperatorTok{$}\NormalTok{zeta[}\DecValTok{1}\NormalTok{] }\OperatorTok{\textasciitilde{}}\StringTok{ }\DecValTok{1}\NormalTok{,}
\NormalTok{  xb\_oprobit }\OperatorTok{\textless{}=}\StringTok{ }\NormalTok{oprobit}\OperatorTok{$}\NormalTok{zeta[}\DecValTok{2}\NormalTok{] }\OperatorTok{\textasciitilde{}}\StringTok{ }\DecValTok{2}\NormalTok{,}
\NormalTok{  xb\_oprobit }\OperatorTok{\textless{}=}\StringTok{ }\NormalTok{oprobit}\OperatorTok{$}\NormalTok{zeta[}\DecValTok{3}\NormalTok{] }\OperatorTok{\textasciitilde{}}\StringTok{ }\DecValTok{3}\NormalTok{,}
  \OtherTok{TRUE} \OperatorTok{\textasciitilde{}}\StringTok{ }\DecValTok{4}
\NormalTok{)}
\NormalTok{hatY\_ologit \textless{}{-}}\StringTok{ }\KeywordTok{case\_when}\NormalTok{(}
\NormalTok{  xb\_ologit }\OperatorTok{\textless{}=}\StringTok{ }\NormalTok{ologit}\OperatorTok{$}\NormalTok{zeta[}\DecValTok{1}\NormalTok{] }\OperatorTok{\textasciitilde{}}\StringTok{ }\DecValTok{1}\NormalTok{,}
\NormalTok{  xb\_ologit }\OperatorTok{\textless{}=}\StringTok{ }\NormalTok{ologit}\OperatorTok{$}\NormalTok{zeta[}\DecValTok{2}\NormalTok{] }\OperatorTok{\textasciitilde{}}\StringTok{ }\DecValTok{2}\NormalTok{,}
\NormalTok{  xb\_ologit }\OperatorTok{\textless{}=}\StringTok{ }\NormalTok{ologit}\OperatorTok{$}\NormalTok{zeta[}\DecValTok{3}\NormalTok{] }\OperatorTok{\textasciitilde{}}\StringTok{ }\DecValTok{3}\NormalTok{,}
  \OtherTok{TRUE} \OperatorTok{\textasciitilde{}}\StringTok{ }\DecValTok{4}
\NormalTok{)}

\NormalTok{pred\_oprobit \textless{}{-}}\StringTok{ }\KeywordTok{round}\NormalTok{(}\KeywordTok{sum}\NormalTok{(house}\OperatorTok{$}\NormalTok{Level }\OperatorTok{==}\StringTok{ }\NormalTok{hatY\_oprobit)}\OperatorTok{/}\KeywordTok{nrow}\NormalTok{(house), }\DecValTok{3}\NormalTok{)}
\NormalTok{pred\_ologit \textless{}{-}}\StringTok{ }\KeywordTok{round}\NormalTok{(}\KeywordTok{sum}\NormalTok{(house}\OperatorTok{$}\NormalTok{Level }\OperatorTok{==}\StringTok{ }\NormalTok{hatY\_ologit)}\OperatorTok{/}\KeywordTok{nrow}\NormalTok{(house), }\DecValTok{3}\NormalTok{)}
\end{Highlighting}
\end{Shaded}

\hypertarget{interepretations}{%
\subsection{Interepretations}\label{interepretations}}

Table \ref{housing} shows results. OLS model shows that respondents
whose household income is in the top 25 percentile live in 3.7\% higher
floor than other respondents. This implies that high earners want to
live in higher floor, which may serve as a status goods. The ordered
probit and logit model are in line with this result. To evaluate two
models quantitatively, consider the following equation.
\begin{equation*}
  E[Levelnum | \mathbf{x}_i] = 
  25\mathbb{P}[level = 1| \mathbf{x}_i] + 
  50\mathbb{P}[level = 2| \mathbf{x}_i] + 
  75\mathbb{P}[level = 3| \mathbf{x}_i] + 
  100\mathbb{P}[level = 4| \mathbf{x}_i].
\end{equation*} We compute this equation with \(Top25 = 1\) and
\(Top25 = 0\) at mean value of \(lnPrice\) and take difference.

\begin{Shaded}
\begin{Highlighting}[]
\NormalTok{quantef \textless{}{-}}\StringTok{ }\ControlFlowTok{function}\NormalTok{(model) \{}
\NormalTok{  b \textless{}{-}}\StringTok{ }\KeywordTok{coef}\NormalTok{(model)}
\NormalTok{  val1 \textless{}{-}}\StringTok{ }\KeywordTok{mean}\NormalTok{(house}\OperatorTok{$}\NormalTok{lnPrice)}\OperatorTok{*}\NormalTok{b[}\DecValTok{1}\NormalTok{] }\OperatorTok{+}\StringTok{ }\NormalTok{b[}\DecValTok{2}\NormalTok{]}
\NormalTok{  val0 \textless{}{-}}\StringTok{ }\KeywordTok{mean}\NormalTok{(house}\OperatorTok{$}\NormalTok{lnPrice)}\OperatorTok{*}\NormalTok{b[}\DecValTok{1}\NormalTok{]}

\NormalTok{  prob \textless{}{-}}\StringTok{ }\KeywordTok{matrix}\NormalTok{(}\KeywordTok{c}\NormalTok{(}\KeywordTok{rep}\NormalTok{(val1, }\DecValTok{3}\NormalTok{), }\KeywordTok{rep}\NormalTok{(val0, }\DecValTok{3}\NormalTok{)), }\DataTypeTok{ncol =} \DecValTok{2}\NormalTok{, }\DataTypeTok{nrow =} \DecValTok{3}\NormalTok{)}
  \ControlFlowTok{for}\NormalTok{ (i }\ControlFlowTok{in} \DecValTok{1}\OperatorTok{:}\DecValTok{3}\NormalTok{) \{}
    \ControlFlowTok{for}\NormalTok{ (j }\ControlFlowTok{in} \DecValTok{1}\OperatorTok{:}\DecValTok{2}\NormalTok{) \{}
\NormalTok{      prob[i,j] \textless{}{-}}\StringTok{ }\KeywordTok{pnorm}\NormalTok{(model}\OperatorTok{$}\NormalTok{zeta[i] }\OperatorTok{{-}}\StringTok{ }\NormalTok{prob[i,j])}
\NormalTok{    \}}
\NormalTok{  \}}
\NormalTok{  Ey1 \textless{}{-}}\StringTok{ }\DecValTok{25}\OperatorTok{*}\NormalTok{prob[}\DecValTok{1}\NormalTok{,}\DecValTok{1}\NormalTok{] }\OperatorTok{+}\StringTok{ }\DecValTok{50}\OperatorTok{*}\NormalTok{(prob[}\DecValTok{2}\NormalTok{,}\DecValTok{1}\NormalTok{]}\OperatorTok{{-}}\NormalTok{prob[}\DecValTok{1}\NormalTok{,}\DecValTok{1}\NormalTok{]) }\OperatorTok{+}\StringTok{ }
\StringTok{    }\DecValTok{75}\OperatorTok{*}\NormalTok{(prob[}\DecValTok{3}\NormalTok{,}\DecValTok{1}\NormalTok{]}\OperatorTok{{-}}\NormalTok{prob[}\DecValTok{2}\NormalTok{,}\DecValTok{1}\NormalTok{]) }\OperatorTok{+}\StringTok{ }\DecValTok{100}\OperatorTok{*}\NormalTok{(}\DecValTok{1}\OperatorTok{{-}}\NormalTok{prob[}\DecValTok{3}\NormalTok{,}\DecValTok{1}\NormalTok{])}
\NormalTok{  Ey0 \textless{}{-}}\StringTok{ }\DecValTok{25}\OperatorTok{*}\NormalTok{prob[}\DecValTok{1}\NormalTok{,}\DecValTok{2}\NormalTok{] }\OperatorTok{+}\StringTok{ }\DecValTok{50}\OperatorTok{*}\NormalTok{(prob[}\DecValTok{2}\NormalTok{,}\DecValTok{2}\NormalTok{]}\OperatorTok{{-}}\NormalTok{prob[}\DecValTok{1}\NormalTok{,}\DecValTok{2}\NormalTok{]) }\OperatorTok{+}\StringTok{ }
\StringTok{    }\DecValTok{75}\OperatorTok{*}\NormalTok{(prob[}\DecValTok{3}\NormalTok{,}\DecValTok{2}\NormalTok{]}\OperatorTok{{-}}\NormalTok{prob[}\DecValTok{2}\NormalTok{,}\DecValTok{2}\NormalTok{]) }\OperatorTok{+}\StringTok{ }\DecValTok{100}\OperatorTok{*}\NormalTok{(}\DecValTok{1}\OperatorTok{{-}}\NormalTok{prob[}\DecValTok{3}\NormalTok{,}\DecValTok{2}\NormalTok{])}
  
  \KeywordTok{return}\NormalTok{(Ey1 }\OperatorTok{{-}}\StringTok{ }\NormalTok{Ey0)}
\NormalTok{\} }

\NormalTok{ef\_oprobit \textless{}{-}}\StringTok{ }\KeywordTok{round}\NormalTok{(}\KeywordTok{quantef}\NormalTok{(oprobit), }\DecValTok{3}\NormalTok{)}
\NormalTok{ef\_ologit \textless{}{-}}\StringTok{ }\KeywordTok{round}\NormalTok{(}\KeywordTok{quantef}\NormalTok{(ologit), }\DecValTok{3}\NormalTok{)}
\end{Highlighting}
\end{Shaded}

As a result, we obtain similar values to OLSE. In the ordered probit
model, earners in the top 25 percentile live in 4.2\% higher floor than
others. In the ordered logit model, earners in the top 25 percentile
live in 5.9\% higher floor than others. Note that, in this application,
model fitness seems to be bad because the percent correctly predicted is
low (16.7\%).

\begin{Shaded}
\begin{Highlighting}[]
\KeywordTok{library}\NormalTok{(stargazer)}
\KeywordTok{stargazer}\NormalTok{(}
\NormalTok{  ols, oprobit, ologit,}
  \DataTypeTok{report =} \StringTok{"vcstp"}\NormalTok{, }\DataTypeTok{keep.stat =} \KeywordTok{c}\NormalTok{(}\StringTok{"n"}\NormalTok{),}
  \DataTypeTok{omit =} \KeywordTok{c}\NormalTok{(}\StringTok{"Constant"}\NormalTok{),}
  \DataTypeTok{add.lines =} \KeywordTok{list}\NormalTok{(}
    \KeywordTok{c}\NormalTok{(}\StringTok{"Cutoff value at 1|2"}\NormalTok{, }\StringTok{""}\NormalTok{, a\_oprobit[}\DecValTok{1}\NormalTok{], a\_ologit[}\DecValTok{1}\NormalTok{]),}
    \KeywordTok{c}\NormalTok{(}\StringTok{"Cutoff value at 2|3"}\NormalTok{, }\StringTok{""}\NormalTok{, a\_oprobit[}\DecValTok{2}\NormalTok{], a\_ologit[}\DecValTok{2}\NormalTok{]),}
    \KeywordTok{c}\NormalTok{(}\StringTok{"Cutoff value at 3|4"}\NormalTok{, }\StringTok{""}\NormalTok{, a\_oprobit[}\DecValTok{3}\NormalTok{], a\_ologit[}\DecValTok{3}\NormalTok{]),}
    \KeywordTok{c}\NormalTok{(}\StringTok{"Quantitative Effect of Top25"}\NormalTok{, }\StringTok{""}\NormalTok{, ef\_oprobit, ef\_ologit),}
    \KeywordTok{c}\NormalTok{(}\StringTok{"Percent correctly predicted"}\NormalTok{, }\StringTok{""}\NormalTok{, pred\_oprobit, pred\_ologit)}
\NormalTok{  ),}
  \DataTypeTok{omit.table.layout =} \StringTok{"n"}\NormalTok{, }\DataTypeTok{table.placement =} \StringTok{"t"}\NormalTok{,}
  \DataTypeTok{title =} \StringTok{"Floor Level of House: Ordered Probit and Logit Model"}\NormalTok{,}
  \DataTypeTok{label =} \StringTok{"housing"}\NormalTok{,}
  \DataTypeTok{type =} \StringTok{"latex"}\NormalTok{, }\DataTypeTok{header =} \OtherTok{FALSE}
\NormalTok{)}
\end{Highlighting}
\end{Shaded}

\begin{table}[t] \centering 
  \caption{Floor Level of House: Ordered Probit and Logit Model} 
  \label{housing} 
\begin{tabular}{@{\extracolsep{5pt}}lccc} 
\\[-1.8ex]\hline 
\hline \\[-1.8ex] 
 & \multicolumn{3}{c}{\textit{Dependent variable:}} \\ 
\cline{2-4} 
\\[-1.8ex] & Levelnum & \multicolumn{2}{c}{Level} \\ 
\\[-1.8ex] & \textit{OLS} & \textit{ordered} & \textit{ordered} \\ 
 & \textit{} & \textit{probit} & \textit{logistic} \\ 
\\[-1.8ex] & (1) & (2) & (3)\\ 
\hline \\[-1.8ex] 
 lnPrice & 0.348 & 0.012 & 0.019 \\ 
  & (0.430) & (0.016) & (0.026) \\ 
  & t = 0.810 & t = 0.777 & t = 0.745 \\ 
  & p = 0.418 & p = 0.438 & p = 0.457 \\ 
  & & & \\ 
 Top25 & 3.714 & 0.156 & 0.239 \\ 
  & (1.723) & (0.064) & (0.106) \\ 
  & t = 2.156 & t = 2.426 & t = 2.259 \\ 
  & p = 0.032 & p = 0.016 & p = 0.024 \\ 
  & & & \\ 
\hline \\[-1.8ex] 
Cutoff value at 1|2 &  & -0.149 & -0.25 \\ 
Cutoff value at 2|3 &  & 0.246 & 0.384 \\ 
Cutoff value at 3|4 &  & 0.97 & 1.574 \\ 
Quantitative Effect of Top25 &  & 4.17 & 5.488 \\ 
Percent correctly predicted &  & 0.167 & 0.167 \\ 
Observations & 1,612 & 1,612 & 1,612 \\ 
\hline 
\hline \\[-1.8ex] 
\end{tabular} 
\end{table}

\hypertarget{empirical-application-of-multinomial-model-gender-discremination-in-job-position}{%
\section{Empirical Application of Multinomial Model: Gender
Discremination in Job
Position}\label{empirical-application-of-multinomial-model-gender-discremination-in-job-position}}

\textbf{Brief Background}. Recently, many developed countries move
toward women's social advancement, for example, an increase of number of
board member. In this application, we explore whether the U.S. bank
hindered the entrance of female into the workhorse.

\noindent \textbf{Data}. We use a built-in dataset called
\texttt{BankWages} in the library \texttt{AER}. This dataset contains
choice of three job position: \texttt{custodial}, \texttt{admin} and
\texttt{manage}. The rank of position is
\texttt{custodial\ \textless{}\ admin\ \textless{}\ manage}. Other
variables are \texttt{education}, \texttt{gender}, and
\texttt{minority}. We use former two variables as explanatory variables.

\begin{Shaded}
\begin{Highlighting}[]
\KeywordTok{library}\NormalTok{(AER)}
\KeywordTok{data}\NormalTok{(BankWages)}
\NormalTok{dt \textless{}{-}}\StringTok{ }\NormalTok{BankWages}
\NormalTok{dt}\OperatorTok{$}\NormalTok{job \textless{}{-}}\StringTok{ }\KeywordTok{as.character}\NormalTok{(dt}\OperatorTok{$}\NormalTok{job)}
\NormalTok{dt}\OperatorTok{$}\NormalTok{job \textless{}{-}}\StringTok{ }\KeywordTok{factor}\NormalTok{(dt}\OperatorTok{$}\NormalTok{job, }\DataTypeTok{levels =} \KeywordTok{c}\NormalTok{(}\StringTok{"admin"}\NormalTok{, }\StringTok{"custodial"}\NormalTok{, }\StringTok{"manage"}\NormalTok{))}
\KeywordTok{head}\NormalTok{(BankWages, }\DecValTok{5}\NormalTok{)}
\end{Highlighting}
\end{Shaded}

\begin{verbatim}
##      job education gender minority
## 1 manage        15   male       no
## 2  admin        16   male       no
## 3  admin        12 female       no
## 4  admin         8 female       no
## 5  admin        15   male       no
\end{verbatim}

\noindent \textbf{Model}. The outcome variable \(y_i\) takes three
values \(\{0, 1, 2\}\). Then, the multinomial logit model has the
following response probabilities \begin{equation*}
  P_{ij} = \mathbb{P}(y_i = j | \mathbf{x}_i) =
  \begin{cases}
    \frac{\exp(\mathbf{x}_i \beta_j)}{1 + \sum_{k=1}^2 \exp(\mathbf{x}_i \beta_k)} &\text{if}\quad j = 1, 2  \\
    \frac{1}{1 + \sum_{k=1}^2 \exp(\mathbf{x}_i \beta_k)}  &\text{if}\quad j = 0
  \end{cases}.
\end{equation*} The log-likelihood function is \begin{equation*}
  M_n(\beta_1, \beta_2) = \sum_{i=1}^n \sum_{j=0}^3 d_{ij} \log (P_{ij}),
\end{equation*} where \(d_{ij}\) is a dummy variable taking 1 if
\(y_i = j\).

In \texttt{R}, some packages provide the multinomial logit model. In
this application, we use the \texttt{multinom} function in the library
\texttt{nnet}.

\begin{Shaded}
\begin{Highlighting}[]
\KeywordTok{library}\NormalTok{(nnet)}
\NormalTok{est\_mlogit \textless{}{-}}\StringTok{ }\KeywordTok{multinom}\NormalTok{(job }\OperatorTok{\textasciitilde{}}\StringTok{ }\NormalTok{education }\OperatorTok{+}\StringTok{ }\NormalTok{gender, }\DataTypeTok{data =}\NormalTok{ dt)}

\CommentTok{\# observations and percent correctly predicted}
\NormalTok{pred \textless{}{-}}\StringTok{ }\NormalTok{est\_mlogit}\OperatorTok{$}\NormalTok{fitted.value}
\NormalTok{pred \textless{}{-}}\StringTok{ }\KeywordTok{colnames}\NormalTok{(pred)[}\KeywordTok{apply}\NormalTok{(pred, }\DecValTok{1}\NormalTok{, which.max)]}
\NormalTok{n \textless{}{-}}\StringTok{ }\KeywordTok{length}\NormalTok{(pred)}
\NormalTok{pcp \textless{}{-}}\StringTok{ }\KeywordTok{round}\NormalTok{(}\KeywordTok{sum}\NormalTok{(pred }\OperatorTok{==}\StringTok{ }\NormalTok{dt}\OperatorTok{$}\NormalTok{job)}\OperatorTok{/}\NormalTok{n, }\DecValTok{3}\NormalTok{)}

\CommentTok{\# Log{-}likelihood and pseudo R{-}sq}
\NormalTok{loglik1 \textless{}{-}}\StringTok{ }\KeywordTok{as.numeric}\NormalTok{(nnet}\OperatorTok{:::}\KeywordTok{logLik.multinom}\NormalTok{(est\_mlogit))}
\NormalTok{est\_mlogit0 \textless{}{-}}\StringTok{ }\KeywordTok{multinom}\NormalTok{(job }\OperatorTok{\textasciitilde{}}\StringTok{ }\DecValTok{1}\NormalTok{, }\DataTypeTok{data =}\NormalTok{ dt)}
\NormalTok{loglik0 \textless{}{-}}\StringTok{ }\KeywordTok{as.numeric}\NormalTok{(nnet}\OperatorTok{:::}\KeywordTok{logLik.multinom}\NormalTok{(est\_mlogit0))}
\NormalTok{pr2 \textless{}{-}}\StringTok{ }\KeywordTok{round}\NormalTok{(}\DecValTok{1} \OperatorTok{{-}}\StringTok{ }\NormalTok{loglik1}\OperatorTok{/}\NormalTok{loglik0, }\DecValTok{3}\NormalTok{)}
\end{Highlighting}
\end{Shaded}

\hypertarget{interpretations}{%
\subsection{Interpretations}\label{interpretations}}

Table \ref{job} summarizes the result of multinomial logit model. The
coefficient represents the change of \(\log(P_{ij}/P_{i0})\) in
corresponding covariate. For example, eduction decreases the log-odds
between \texttt{custodial} and \texttt{admin},
\(\log(P_{i, custodial}/P_{i, admin})\) by -0.562. This implies that
those who received higher education are more likely to obtain the
position \texttt{admin}. Highly-educated workers are also more likely to
obtain the position \texttt{manage}. Moreover, a female dummy decrease
the log-odds between \texttt{manage} and \texttt{admin} by -0.748, which
implies that females are less likely to obtain higher position
\texttt{manage}. From this result, we conclude that the U.S. bank
disencouraged females to assign higher job position.

Finally, we should check the model fitness. The predicted position is
the outcome with the highest estimated probability. The multinomial
logit model correctly predicts many cases (correction rate: 85.2\%).

\begin{Shaded}
\begin{Highlighting}[]
\KeywordTok{stargazer}\NormalTok{(}
\NormalTok{  est\_mlogit,}
  \DataTypeTok{covariate.labels =} \KeywordTok{c}\NormalTok{(}\StringTok{"Education"}\NormalTok{, }\StringTok{"Female = 1"}\NormalTok{),}
  \DataTypeTok{report =} \StringTok{"vcstp"}\NormalTok{, }\DataTypeTok{omit.stat =} \KeywordTok{c}\NormalTok{(}\StringTok{"aic"}\NormalTok{),}
  \DataTypeTok{add.lines =} \KeywordTok{list}\NormalTok{(}
    \KeywordTok{c}\NormalTok{(}\StringTok{"Observations"}\NormalTok{, n, }\StringTok{""}\NormalTok{),}
    \KeywordTok{c}\NormalTok{(}\StringTok{"Percent correctly predicted"}\NormalTok{, pcp, }\StringTok{""}\NormalTok{),}
    \KeywordTok{c}\NormalTok{(}\StringTok{"Log{-}likelihood"}\NormalTok{, }\KeywordTok{round}\NormalTok{(loglik1, }\DecValTok{3}\NormalTok{), }\StringTok{""}\NormalTok{),}
    \KeywordTok{c}\NormalTok{(}\StringTok{"Pseudo R{-}sq"}\NormalTok{, pr2, }\StringTok{""}\NormalTok{)}
\NormalTok{  ),}
  \DataTypeTok{omit.table.layout =} \StringTok{"n"}\NormalTok{, }\DataTypeTok{table.placement =} \StringTok{"t"}\NormalTok{,}
  \DataTypeTok{title =} \StringTok{"Multinomial Logit Model of Job Position"}\NormalTok{,}
  \DataTypeTok{label =} \StringTok{"job"}\NormalTok{,}
  \DataTypeTok{type =} \StringTok{"latex"}\NormalTok{, }\DataTypeTok{header =} \OtherTok{FALSE}  
\NormalTok{)}
\end{Highlighting}
\end{Shaded}

\begin{table}[t] \centering 
  \caption{Multinomial Logit Model of Job Position} 
  \label{job} 
\begin{tabular}{@{\extracolsep{5pt}}lcc} 
\\[-1.8ex]\hline 
\hline \\[-1.8ex] 
 & \multicolumn{2}{c}{\textit{Dependent variable:}} \\ 
\cline{2-3} 
\\[-1.8ex] & custodial & manage \\ 
\\[-1.8ex] & (1) & (2)\\ 
\hline \\[-1.8ex] 
 Education & $-$0.562 & 1.661 \\ 
  & (0.098) & (0.247) \\ 
  & t = $-$5.721 & t = 6.715 \\ 
  & p = 0.000 & p = 0.000 \\ 
  & & \\ 
 Female = 1 & $-$10.976 & $-$0.748 \\ 
  & (27.808) & (0.429) \\ 
  & t = $-$0.395 & t = $-$1.743 \\ 
  & p = 0.694 & p = 0.082 \\ 
  & & \\ 
 Constant & 5.030 & $-$26.730 \\ 
  & (1.130) & (3.874) \\ 
  & t = 4.450 & t = $-$6.899 \\ 
  & p = 0.00001 & p = 0.000 \\ 
  & & \\ 
\hline \\[-1.8ex] 
Observations & 474 &  \\ 
Percent correctly predicted & 0.852 &  \\ 
Log-likelihood & -144.928 &  \\ 
Pseudo R-sq & 0.546 &  \\ 
\hline 
\hline \\[-1.8ex] 
\end{tabular} 
\end{table}

\end{document}
