% Options for packages loaded elsewhere
\PassOptionsToPackage{unicode}{hyperref}
\PassOptionsToPackage{hyphens}{url}
%
\documentclass[
  12pt,
]{article}
\usepackage{lmodern}
\usepackage{amssymb,amsmath}
\usepackage{ifxetex,ifluatex}
\ifnum 0\ifxetex 1\fi\ifluatex 1\fi=0 % if pdftex
  \usepackage[T1]{fontenc}
  \usepackage[utf8]{inputenc}
  \usepackage{textcomp} % provide euro and other symbols
\else % if luatex or xetex
  \usepackage{unicode-math}
  \defaultfontfeatures{Scale=MatchLowercase}
  \defaultfontfeatures[\rmfamily]{Ligatures=TeX,Scale=1}
\fi
% Use upquote if available, for straight quotes in verbatim environments
\IfFileExists{upquote.sty}{\usepackage{upquote}}{}
\IfFileExists{microtype.sty}{% use microtype if available
  \usepackage[]{microtype}
  \UseMicrotypeSet[protrusion]{basicmath} % disable protrusion for tt fonts
}{}
\makeatletter
\@ifundefined{KOMAClassName}{% if non-KOMA class
  \IfFileExists{parskip.sty}{%
    \usepackage{parskip}
  }{% else
    \setlength{\parindent}{0pt}
    \setlength{\parskip}{6pt plus 2pt minus 1pt}}
}{% if KOMA class
  \KOMAoptions{parskip=half}}
\makeatother
\usepackage{xcolor}
\IfFileExists{xurl.sty}{\usepackage{xurl}}{} % add URL line breaks if available
\IfFileExists{bookmark.sty}{\usepackage{bookmark}}{\usepackage{hyperref}}
\hypersetup{
  pdftitle={Econometrics II TA Session \#4},
  pdfauthor={Hiroki Kato},
  hidelinks,
  pdfcreator={LaTeX via pandoc}}
\urlstyle{same} % disable monospaced font for URLs
\usepackage[margin=1in]{geometry}
\usepackage{color}
\usepackage{fancyvrb}
\newcommand{\VerbBar}{|}
\newcommand{\VERB}{\Verb[commandchars=\\\{\}]}
\DefineVerbatimEnvironment{Highlighting}{Verbatim}{commandchars=\\\{\}}
% Add ',fontsize=\small' for more characters per line
\usepackage{framed}
\definecolor{shadecolor}{RGB}{248,248,248}
\newenvironment{Shaded}{\begin{snugshade}}{\end{snugshade}}
\newcommand{\AlertTok}[1]{\textcolor[rgb]{0.94,0.16,0.16}{#1}}
\newcommand{\AnnotationTok}[1]{\textcolor[rgb]{0.56,0.35,0.01}{\textbf{\textit{#1}}}}
\newcommand{\AttributeTok}[1]{\textcolor[rgb]{0.77,0.63,0.00}{#1}}
\newcommand{\BaseNTok}[1]{\textcolor[rgb]{0.00,0.00,0.81}{#1}}
\newcommand{\BuiltInTok}[1]{#1}
\newcommand{\CharTok}[1]{\textcolor[rgb]{0.31,0.60,0.02}{#1}}
\newcommand{\CommentTok}[1]{\textcolor[rgb]{0.56,0.35,0.01}{\textit{#1}}}
\newcommand{\CommentVarTok}[1]{\textcolor[rgb]{0.56,0.35,0.01}{\textbf{\textit{#1}}}}
\newcommand{\ConstantTok}[1]{\textcolor[rgb]{0.00,0.00,0.00}{#1}}
\newcommand{\ControlFlowTok}[1]{\textcolor[rgb]{0.13,0.29,0.53}{\textbf{#1}}}
\newcommand{\DataTypeTok}[1]{\textcolor[rgb]{0.13,0.29,0.53}{#1}}
\newcommand{\DecValTok}[1]{\textcolor[rgb]{0.00,0.00,0.81}{#1}}
\newcommand{\DocumentationTok}[1]{\textcolor[rgb]{0.56,0.35,0.01}{\textbf{\textit{#1}}}}
\newcommand{\ErrorTok}[1]{\textcolor[rgb]{0.64,0.00,0.00}{\textbf{#1}}}
\newcommand{\ExtensionTok}[1]{#1}
\newcommand{\FloatTok}[1]{\textcolor[rgb]{0.00,0.00,0.81}{#1}}
\newcommand{\FunctionTok}[1]{\textcolor[rgb]{0.00,0.00,0.00}{#1}}
\newcommand{\ImportTok}[1]{#1}
\newcommand{\InformationTok}[1]{\textcolor[rgb]{0.56,0.35,0.01}{\textbf{\textit{#1}}}}
\newcommand{\KeywordTok}[1]{\textcolor[rgb]{0.13,0.29,0.53}{\textbf{#1}}}
\newcommand{\NormalTok}[1]{#1}
\newcommand{\OperatorTok}[1]{\textcolor[rgb]{0.81,0.36,0.00}{\textbf{#1}}}
\newcommand{\OtherTok}[1]{\textcolor[rgb]{0.56,0.35,0.01}{#1}}
\newcommand{\PreprocessorTok}[1]{\textcolor[rgb]{0.56,0.35,0.01}{\textit{#1}}}
\newcommand{\RegionMarkerTok}[1]{#1}
\newcommand{\SpecialCharTok}[1]{\textcolor[rgb]{0.00,0.00,0.00}{#1}}
\newcommand{\SpecialStringTok}[1]{\textcolor[rgb]{0.31,0.60,0.02}{#1}}
\newcommand{\StringTok}[1]{\textcolor[rgb]{0.31,0.60,0.02}{#1}}
\newcommand{\VariableTok}[1]{\textcolor[rgb]{0.00,0.00,0.00}{#1}}
\newcommand{\VerbatimStringTok}[1]{\textcolor[rgb]{0.31,0.60,0.02}{#1}}
\newcommand{\WarningTok}[1]{\textcolor[rgb]{0.56,0.35,0.01}{\textbf{\textit{#1}}}}
\usepackage{graphicx}
\makeatletter
\def\maxwidth{\ifdim\Gin@nat@width>\linewidth\linewidth\else\Gin@nat@width\fi}
\def\maxheight{\ifdim\Gin@nat@height>\textheight\textheight\else\Gin@nat@height\fi}
\makeatother
% Scale images if necessary, so that they will not overflow the page
% margins by default, and it is still possible to overwrite the defaults
% using explicit options in \includegraphics[width, height, ...]{}
\setkeys{Gin}{width=\maxwidth,height=\maxheight,keepaspectratio}
% Set default figure placement to htbp
\makeatletter
\def\fps@figure{htbp}
\makeatother
\setlength{\emergencystretch}{3em} % prevent overfull lines
\providecommand{\tightlist}{%
  \setlength{\itemsep}{0pt}\setlength{\parskip}{0pt}}
\setcounter{secnumdepth}{5}
\usepackage{zxjatype}
\setCJKmainfont[BoldFont = IPAゴシック]{IPA明朝}
\setCJKsansfont{IPAゴシック}
\setCJKmonofont{IPAゴシック}
\parindent = 1em
\newcommand{\argmax}{\mathop{\rm arg~max}\limits}
\newcommand{\argmin}{\mathop{\rm arg~min}\limits}
\DeclareMathOperator*{\plim}{plim}
\usepackage{xcolor}
\ifluatex
  \usepackage{selnolig}  % disable illegal ligatures
\fi

\title{Econometrics II TA Session \#4}
\author{Hiroki Kato}
\date{}

\begin{document}
\maketitle

\hypertarget{empirical-application-of-ordered-probit-and-logit-model-housing-as-status-goods}{%
\section{Empirical Application of Ordered Probit and Logit Model:
Housing as Status
Goods}\label{empirical-application-of-ordered-probit-and-logit-model-housing-as-status-goods}}

\hypertarget{background-and-data}{%
\subsection{Background and Data}\label{background-and-data}}

A desire to signal high income or wealth may cause consumers to purchase
status goods such as luxury cars. In this application, we explore
whether housing serves as status goods, using the case of apartment
building. We investigate the relationship between living in a high floor
and income, controlling the quality of housing. Our hypothesis is that
high-earners are more likely to live on the upper floor.

We use the housing data originally coming from the American Housing
Survey conducted in 2013 \footnote{\url{https://www.census.gov/programs-surveys/ahs.html}.
  This is a repeated cross-section survey. We use the data at one time.}.
This dataset (hereafter \texttt{housing}) contains the following
variables:

\begin{itemize}
\tightlist
\item
  \texttt{Level}: ordered value of floor where respondent lives (1:Low -
  4:High)
\item
  \texttt{lnPrice}: logged price of housing (proxy for quality of house)
\item
  \texttt{Top25}: a dummy variable taking one if household income is in
  the top 25 percentile in sample.
\end{itemize}

We split data into two subsets: the \emph{training} data and the
\emph{test} data. The training data, which is used for estimation and
model fitness, is randoly drawn from the original data. The sample size
of this subset is two thirds of total observations of the original one
(\(N = 1,074\)). The test data, which is used for model prediction,
consists of observations which the training data does not include
(\(N = 538\)).

\begin{Shaded}
\begin{Highlighting}[]
\NormalTok{dt \textless{}{-}}\StringTok{ }\KeywordTok{read.csv}\NormalTok{(}\DataTypeTok{file =} \StringTok{"./data/housing.csv"}\NormalTok{, }\DataTypeTok{header =} \OtherTok{TRUE}\NormalTok{,  }\DataTypeTok{sep =} \StringTok{","}\NormalTok{)}
\NormalTok{dt \textless{}{-}}\StringTok{ }\NormalTok{dt[,}\KeywordTok{c}\NormalTok{(}\StringTok{"Level"}\NormalTok{, }\StringTok{"lnPrice"}\NormalTok{, }\StringTok{"Top25"}\NormalTok{)]}

\KeywordTok{set.seed}\NormalTok{(}\DecValTok{120511}\NormalTok{)}
\NormalTok{train\_id \textless{}{-}}\StringTok{ }\KeywordTok{sample}\NormalTok{(}\DecValTok{1}\OperatorTok{:}\KeywordTok{nrow}\NormalTok{(dt), }\DataTypeTok{size =}\NormalTok{ (}\DecValTok{2}\OperatorTok{/}\DecValTok{3}\NormalTok{)}\OperatorTok{*}\KeywordTok{nrow}\NormalTok{(dt), }\DataTypeTok{replace =} \OtherTok{FALSE}\NormalTok{)}
\NormalTok{train\_dt \textless{}{-}}\StringTok{ }\NormalTok{dt[train\_id,]; test\_dt \textless{}{-}}\StringTok{ }\NormalTok{dt[}\OperatorTok{{-}}\NormalTok{train\_id,]}

\KeywordTok{head}\NormalTok{(train\_dt)}
\end{Highlighting}
\end{Shaded}

\begin{verbatim}
##      Level   lnPrice Top25
## 1099     4  9.903538     0
## 2        4 11.512935     1
## 1398     4 11.775297     0
## 405      2 12.429220     0
## 579      1 11.289794     0
## 1157     1 10.596660     0
\end{verbatim}

\hypertarget{model}{%
\subsection{Model}\label{model}}

The outcome variable is \texttt{Level} taking \(\{1, 2, 3, 4\}\).
Consider the following regression equation of a latent variable:
\begin{equation*}
  y_i^* = \mathbf{x}_i \beta + u_i,
\end{equation*} where a vector of explanatory variables are
\texttt{lnPrice} and \texttt{Top25}, and \(u_i\) is an error term. The
relationship between the latent variable \(y_i^*\) and the observed
outcome variable is \begin{equation*}
  Level =
  \begin{cases}
    1 &\text{if}\quad -\infty < y_i^* \le a_1  \\
    2 &\text{if}\quad a_1 < y_i^* \le a_2 \\
    3 &\text{if}\quad a_2 < y_i^* \le a_3 \\
    4 &\text{if}\quad a_3 < y_i^* < +\infty
  \end{cases}.
\end{equation*}

Consider the probability of realization of \(y_i\), that is,
\begin{equation*}
  \begin{split}
  \mathbb{P}(y_i = k | \mathbf{x}_i) 
  &= \mathbb{P}(a_{k-1} - \mathbf{x}_i \beta < u_i \le a_k - \mathbf{x}_i \beta | \mathbf{x}_i)  \\
  &= G(a_k - \mathbf{x}_i \beta) - G(a_{k-1} - \mathbf{x}_i \beta),
  \end{split}
\end{equation*} where \(a_{4} = +\infty\) and \(a_0 = -\infty\). Then,
the likelihood function is defined by \begin{equation*}
  p((y_i|\mathbf{x}_i), i = 1, \ldots, n; \beta, a_1, \ldots, a_3)
  = \prod_{i=1}^n \prod_{k=1}^4 (G(a_k - \mathbf{x}_i \beta) - G(a_{k-1} - \mathbf{x}_i \beta))^{I_{ik}}.
\end{equation*} where \(I_{ik}\) is a indicator variable taking 1 if
\(y_i = k\). Finally, the log-likelihood function is \begin{equation*}
  M(\beta, a_1, a_2, a_3) = \sum_{i=1}^n \sum_{k=1}^4 I_{ik} \log(G(a_k - \mathbf{x}_i \beta) - G(a_{k-1} - \mathbf{x}_i \beta)).
\end{equation*} Usually, \(G(a)\) assumes the standard normal
distribution, \(\Phi(a)\), or the logistic distribution,
\(1/(1 + \exp(-a))\). We estimate \(\theta = (\beta, a_1, a_2, a_3)'\)
to minimize the log-likelihood function, that is, \begin{equation*}
  \hat{\theta} \in \mathop{\rm arg~min}\limits_{(\beta, a_1, a_2, a_3)} M(\beta, a_1, a_2, a_3).
\end{equation*}

In \texttt{R}, the library (package) \texttt{MASS} provides the
\texttt{polr} function which estimates the ordered probit and logit
model. Although we can use the \texttt{nlm} function when we define the
log-likelihood function, we do not report this method.

\begin{Shaded}
\begin{Highlighting}[]
\KeywordTok{library}\NormalTok{(MASS)}

\NormalTok{model \textless{}{-}}\StringTok{ }\KeywordTok{factor}\NormalTok{(Level) }\OperatorTok{\textasciitilde{}}\StringTok{ }\NormalTok{lnPrice }\OperatorTok{+}\StringTok{ }\NormalTok{Top25}
\NormalTok{oprobit \textless{}{-}}\StringTok{ }\KeywordTok{polr}\NormalTok{(model, }\DataTypeTok{data =}\NormalTok{ train\_dt, }\DataTypeTok{method =} \StringTok{"probit"}\NormalTok{)}
\NormalTok{ologit \textless{}{-}}\StringTok{ }\KeywordTok{polr}\NormalTok{(model, }\DataTypeTok{data =}\NormalTok{ train\_dt, }\DataTypeTok{method =} \StringTok{"logistic"}\NormalTok{)}
\end{Highlighting}
\end{Shaded}

\hypertarget{interepretation-and-model-fitness}{%
\subsection{Interepretation and Model
Fitness}\label{interepretation-and-model-fitness}}

Table \ref{housing} shows results. In both models, the latent variable
\(y_i^*\) is increasing in \texttt{Top25}. This means that high-earners
have higer value of latent variable \(y_i^*\). Since the cutoff values
are increasing in the observed \(y_i\), we can conclude that
high-earners are more likely to live on the upper floor.

To evaluate model fitness, we use the \emph{percent correctly
predicted}, which is the percentage of unit whose predicted \(y_i\)
matches the actual \(y_i\). First, we calculate
\(\mathbf{x}_i \hat{\beta}\). If this value is in
\((-\infty, \hat{a}_1]\), \((\hat{a}_1, a_2]\),
\((\hat{a}_2, \hat{a}_3]\), and \((\hat{a}_3, +\infty)\), then we take
\(\hat{y}_i = 1\), \(\hat{y}_i = 2\), \(\hat{y}_i = 3\) and
\(\hat{y}_i = 4\), respectively. Using the training data (in-sample) and
the test data (out-of-sample), we calculate this index.

\begin{Shaded}
\begin{Highlighting}[]
\KeywordTok{library}\NormalTok{(tidyverse) }\CommentTok{\#use case\_when()}
\CommentTok{\# coefficients}
\NormalTok{bp \textless{}{-}}\StringTok{ }\KeywordTok{matrix}\NormalTok{(}\KeywordTok{coef}\NormalTok{(oprobit), }\DataTypeTok{nrow =} \DecValTok{2}\NormalTok{); bl \textless{}{-}}\StringTok{ }\KeywordTok{matrix}\NormalTok{(}\KeywordTok{coef}\NormalTok{(ologit), }\DataTypeTok{nrow =} \DecValTok{2}\NormalTok{)}
\CommentTok{\# cutoff value}
\NormalTok{ap \textless{}{-}}\StringTok{ }\NormalTok{oprobit}\OperatorTok{$}\NormalTok{zeta; al \textless{}{-}}\StringTok{ }\NormalTok{ologit}\OperatorTok{$}\NormalTok{zeta}
\NormalTok{seap \textless{}{-}}\StringTok{ }\KeywordTok{sqrt}\NormalTok{(}\KeywordTok{diag}\NormalTok{(}\KeywordTok{vcov}\NormalTok{(oprobit)))[}\DecValTok{3}\OperatorTok{:}\DecValTok{5}\NormalTok{]; seal \textless{}{-}}\StringTok{ }\KeywordTok{sqrt}\NormalTok{(}\KeywordTok{diag}\NormalTok{(}\KeywordTok{vcov}\NormalTok{(ologit)))[}\DecValTok{3}\OperatorTok{:}\DecValTok{5}\NormalTok{]}
\CommentTok{\# in{-}sample prediction}
\NormalTok{indt \textless{}{-}}\StringTok{ }\KeywordTok{as.matrix}\NormalTok{(train\_dt[,}\KeywordTok{c}\NormalTok{(}\StringTok{"lnPrice"}\NormalTok{, }\StringTok{"Top25"}\NormalTok{)])}
\NormalTok{in\_xbp \textless{}{-}}\StringTok{ }\NormalTok{indt }\OperatorTok{\%*\%}\StringTok{ }\NormalTok{bp; in\_xbl \textless{}{-}}\StringTok{ }\NormalTok{indt }\OperatorTok{\%*\%}\StringTok{ }\NormalTok{bl}

\NormalTok{in\_hatYp \textless{}{-}}\StringTok{ }\KeywordTok{case\_when}\NormalTok{(}
\NormalTok{  in\_xbp }\OperatorTok{\textless{}=}\StringTok{ }\NormalTok{ap[}\DecValTok{1}\NormalTok{] }\OperatorTok{\textasciitilde{}}\StringTok{ }\DecValTok{1}\NormalTok{,}
\NormalTok{  in\_xbp }\OperatorTok{\textless{}=}\StringTok{ }\NormalTok{ap[}\DecValTok{2}\NormalTok{] }\OperatorTok{\textasciitilde{}}\StringTok{ }\DecValTok{2}\NormalTok{,}
\NormalTok{  in\_xbp }\OperatorTok{\textless{}=}\StringTok{ }\NormalTok{ap[}\DecValTok{3}\NormalTok{] }\OperatorTok{\textasciitilde{}}\StringTok{ }\DecValTok{3}\NormalTok{,}
  \OtherTok{TRUE} \OperatorTok{\textasciitilde{}}\StringTok{ }\DecValTok{4}
\NormalTok{)}

\NormalTok{in\_hatYl \textless{}{-}}\StringTok{ }\KeywordTok{case\_when}\NormalTok{(}
\NormalTok{  in\_xbl }\OperatorTok{\textless{}=}\StringTok{ }\NormalTok{al[}\DecValTok{1}\NormalTok{] }\OperatorTok{\textasciitilde{}}\StringTok{ }\DecValTok{1}\NormalTok{,}
\NormalTok{  in\_xbl }\OperatorTok{\textless{}=}\StringTok{ }\NormalTok{al[}\DecValTok{2}\NormalTok{] }\OperatorTok{\textasciitilde{}}\StringTok{ }\DecValTok{2}\NormalTok{,}
\NormalTok{  in\_xbl }\OperatorTok{\textless{}=}\StringTok{ }\NormalTok{al[}\DecValTok{3}\NormalTok{] }\OperatorTok{\textasciitilde{}}\StringTok{ }\DecValTok{3}\NormalTok{,}
  \OtherTok{TRUE} \OperatorTok{\textasciitilde{}}\StringTok{ }\DecValTok{4}
\NormalTok{)}

\NormalTok{inpred\_p \textless{}{-}}\StringTok{ }\KeywordTok{round}\NormalTok{(}\KeywordTok{sum}\NormalTok{(train\_dt}\OperatorTok{$}\NormalTok{Level }\OperatorTok{==}\StringTok{ }\NormalTok{in\_hatYp)}\OperatorTok{/}\KeywordTok{nrow}\NormalTok{(train\_dt), }\DecValTok{3}\NormalTok{)}
\NormalTok{inpred\_l \textless{}{-}}\StringTok{ }\KeywordTok{round}\NormalTok{(}\KeywordTok{sum}\NormalTok{(train\_dt}\OperatorTok{$}\NormalTok{Level }\OperatorTok{==}\StringTok{ }\NormalTok{in\_hatYl)}\OperatorTok{/}\KeywordTok{nrow}\NormalTok{(train\_dt), }\DecValTok{3}\NormalTok{)}

\CommentTok{\# out{-}of{-}sample prediction}
\NormalTok{outdt \textless{}{-}}\StringTok{ }\KeywordTok{as.matrix}\NormalTok{(test\_dt[,}\KeywordTok{c}\NormalTok{(}\StringTok{"lnPrice"}\NormalTok{, }\StringTok{"Top25"}\NormalTok{)])}
\NormalTok{out\_xbp \textless{}{-}}\StringTok{ }\NormalTok{outdt }\OperatorTok{\%*\%}\StringTok{ }\NormalTok{bp; out\_xbl \textless{}{-}}\StringTok{ }\NormalTok{outdt }\OperatorTok{\%*\%}\StringTok{ }\NormalTok{bl}

\NormalTok{out\_hatYp \textless{}{-}}\StringTok{ }\KeywordTok{case\_when}\NormalTok{(}
\NormalTok{  out\_xbp }\OperatorTok{\textless{}=}\StringTok{ }\NormalTok{ap[}\DecValTok{1}\NormalTok{] }\OperatorTok{\textasciitilde{}}\StringTok{ }\DecValTok{1}\NormalTok{,}
\NormalTok{  out\_xbp }\OperatorTok{\textless{}=}\StringTok{ }\NormalTok{ap[}\DecValTok{2}\NormalTok{] }\OperatorTok{\textasciitilde{}}\StringTok{ }\DecValTok{2}\NormalTok{,}
\NormalTok{  out\_xbp }\OperatorTok{\textless{}=}\StringTok{ }\NormalTok{ap[}\DecValTok{3}\NormalTok{] }\OperatorTok{\textasciitilde{}}\StringTok{ }\DecValTok{3}\NormalTok{,}
  \OtherTok{TRUE} \OperatorTok{\textasciitilde{}}\StringTok{ }\DecValTok{4}
\NormalTok{)}

\NormalTok{out\_hatYl \textless{}{-}}\StringTok{ }\KeywordTok{case\_when}\NormalTok{(}
\NormalTok{  out\_xbl }\OperatorTok{\textless{}=}\StringTok{ }\NormalTok{al[}\DecValTok{1}\NormalTok{] }\OperatorTok{\textasciitilde{}}\StringTok{ }\DecValTok{1}\NormalTok{,}
\NormalTok{  out\_xbl }\OperatorTok{\textless{}=}\StringTok{ }\NormalTok{al[}\DecValTok{2}\NormalTok{] }\OperatorTok{\textasciitilde{}}\StringTok{ }\DecValTok{2}\NormalTok{,}
\NormalTok{  out\_xbl }\OperatorTok{\textless{}=}\StringTok{ }\NormalTok{al[}\DecValTok{3}\NormalTok{] }\OperatorTok{\textasciitilde{}}\StringTok{ }\DecValTok{3}\NormalTok{,}
  \OtherTok{TRUE} \OperatorTok{\textasciitilde{}}\StringTok{ }\DecValTok{4}
\NormalTok{)}

\NormalTok{outpred\_p \textless{}{-}}\StringTok{ }\KeywordTok{round}\NormalTok{(}\KeywordTok{sum}\NormalTok{(test\_dt}\OperatorTok{$}\NormalTok{Level }\OperatorTok{==}\StringTok{ }\NormalTok{out\_hatYp)}\OperatorTok{/}\KeywordTok{nrow}\NormalTok{(test\_dt), }\DecValTok{3}\NormalTok{)}
\NormalTok{outpred\_l \textless{}{-}}\StringTok{ }\KeywordTok{round}\NormalTok{(}\KeywordTok{sum}\NormalTok{(test\_dt}\OperatorTok{$}\NormalTok{Level }\OperatorTok{==}\StringTok{ }\NormalTok{out\_hatYl)}\OperatorTok{/}\KeywordTok{nrow}\NormalTok{(test\_dt), }\DecValTok{3}\NormalTok{)}
\end{Highlighting}
\end{Shaded}

As a result, the percent correctly predicted is almost 16\% when we use
the in-sample data. When we use the test data, this index slightly
increases. Overall, out model seems not to be good because the percent
correctly predicted is low.

\begin{Shaded}
\begin{Highlighting}[]
\NormalTok{seap \textless{}{-}}\StringTok{ }\KeywordTok{sprintf}\NormalTok{(}\StringTok{"(\%1.3f)"}\NormalTok{, seap); seal \textless{}{-}}\StringTok{ }\KeywordTok{sprintf}\NormalTok{(}\StringTok{"(\%1.3f)"}\NormalTok{, seal)}

\KeywordTok{library}\NormalTok{(stargazer)}
\KeywordTok{stargazer}\NormalTok{(}
\NormalTok{  oprobit, ologit,}
  \DataTypeTok{report =} \StringTok{"vcs"}\NormalTok{, }\DataTypeTok{keep.stat =} \KeywordTok{c}\NormalTok{(}\StringTok{"n"}\NormalTok{),}
  \DataTypeTok{omit =} \KeywordTok{c}\NormalTok{(}\StringTok{"Constant"}\NormalTok{),}
  \DataTypeTok{add.lines =} \KeywordTok{list}\NormalTok{(}
    \KeywordTok{c}\NormalTok{(}\StringTok{"Cutoff value at 1|2"}\NormalTok{, }\KeywordTok{round}\NormalTok{(ap[}\DecValTok{1}\NormalTok{], }\DecValTok{3}\NormalTok{), }\KeywordTok{round}\NormalTok{(al[}\DecValTok{1}\NormalTok{], }\DecValTok{3}\NormalTok{)),}
    \KeywordTok{c}\NormalTok{(}\StringTok{""}\NormalTok{, seap[}\DecValTok{1}\NormalTok{], seal[}\DecValTok{1}\NormalTok{]),}
    \KeywordTok{c}\NormalTok{(}\StringTok{"Cutoff value at 2|3"}\NormalTok{, }\KeywordTok{round}\NormalTok{(ap[}\DecValTok{2}\NormalTok{], }\DecValTok{3}\NormalTok{), }\KeywordTok{round}\NormalTok{(al[}\DecValTok{2}\NormalTok{], }\DecValTok{3}\NormalTok{)),}
    \KeywordTok{c}\NormalTok{(}\StringTok{""}\NormalTok{, seap[}\DecValTok{2}\NormalTok{], seal[}\DecValTok{2}\NormalTok{]),}
    \KeywordTok{c}\NormalTok{(}\StringTok{"Cutoff value at 3|4"}\NormalTok{, }\KeywordTok{round}\NormalTok{(ap[}\DecValTok{3}\NormalTok{], }\DecValTok{3}\NormalTok{), }\KeywordTok{round}\NormalTok{(al[}\DecValTok{3}\NormalTok{], }\DecValTok{3}\NormalTok{)),}
    \KeywordTok{c}\NormalTok{(}\StringTok{""}\NormalTok{, seap[}\DecValTok{3}\NormalTok{], seal[}\DecValTok{3}\NormalTok{]),}
    \KeywordTok{c}\NormalTok{(}\StringTok{"Percent correctly predicted (in{-}sample)"}\NormalTok{, inpred\_p, inpred\_l),}
    \KeywordTok{c}\NormalTok{(}\StringTok{"Percent correctly predicted (out{-}of{-}sample)"}\NormalTok{, outpred\_p, outpred\_l)}
\NormalTok{  ),}
  \DataTypeTok{omit.table.layout =} \StringTok{"n"}\NormalTok{, }\DataTypeTok{table.placement =} \StringTok{"t"}\NormalTok{,}
  \DataTypeTok{title =} \StringTok{"Floor Level of House: Ordered Probit and Logit Model"}\NormalTok{,}
  \DataTypeTok{label =} \StringTok{"housing"}\NormalTok{,}
  \DataTypeTok{type =} \StringTok{"latex"}\NormalTok{, }\DataTypeTok{header =} \OtherTok{FALSE}
\NormalTok{)}
\end{Highlighting}
\end{Shaded}

\begin{table}[t] \centering 
  \caption{Floor Level of House: Ordered Probit and Logit Model} 
  \label{housing} 
\begin{tabular}{@{\extracolsep{5pt}}lcc} 
\\[-1.8ex]\hline 
\hline \\[-1.8ex] 
 & \multicolumn{2}{c}{\textit{Dependent variable:}} \\ 
\cline{2-3} 
\\[-1.8ex] & \multicolumn{2}{c}{Level} \\ 
\\[-1.8ex] & \textit{ordered} & \textit{ordered} \\ 
 & \textit{probit} & \textit{logistic} \\ 
\\[-1.8ex] & (1) & (2)\\ 
\hline \\[-1.8ex] 
 lnPrice & $-$0.007 & $-$0.013 \\ 
  & (0.019) & (0.031) \\ 
  & & \\ 
 Top25 & 0.133 & 0.202 \\ 
  & (0.080) & (0.132) \\ 
  & & \\ 
\hline \\[-1.8ex] 
Cutoff value at 1|2 & -0.371 & -0.611 \\ 
 & (0.227) & (0.363) \\ 
Cutoff value at 2|3 & 0.02 & 0.014 \\ 
 & (0.226) & (0.362) \\ 
Cutoff value at 3|4 & 0.719 & 1.163 \\ 
 & (0.227) & (0.365) \\ 
Percent correctly predicted (in-sample) & 0.161 & 0.161 \\ 
Percent correctly predicted (out-of-sample) & 0.175 & 0.175 \\ 
Observations & 1,074 & 1,074 \\ 
\hline 
\hline \\[-1.8ex] 
\end{tabular} 
\end{table}

\hypertarget{empirical-application-of-multinomial-model-gender-discremination-in-job-position}{%
\section{Empirical Application of Multinomial Model: Gender
Discremination in Job
Position}\label{empirical-application-of-multinomial-model-gender-discremination-in-job-position}}

\hypertarget{background-and-data-1}{%
\subsection{Background and Data}\label{background-and-data-1}}

Recently, many developed countries move toward women's social
advancement, for example, an increase of number of board member. In this
application, we explore whether the gender discremination existed in the
U.S. bank industry. Our hypothesis is that women are less likely to be
given a higher position than male.

We use a built-in dataset called \texttt{BankWages} in the library
\texttt{AER}. This datase contains the following variables:

\begin{itemize}
\tightlist
\item
  \texttt{job}: three job position. The rank of position is
  \texttt{custodial\ \textless{}\ admin\ \textless{}\ manage}.
\item
  \texttt{education}: years of education
\item
  \texttt{gender}: a dummy variable of female
\end{itemize}

Again, we split data into two subsets: the \emph{training} data and the
\emph{test} data. The training data, which is used for estimation and
model fitness, is randoly drawn from the original data. The sample size
of this subset is two thirds of total observations of the original one
(\(N = 316\)). The test data, which is used for model prediction,
consists of observations which the training data does not include
(\(N = 158\)).

To use the multinomial logit model in \texttt{R}, we need to transform
outcome variable into the form \texttt{factor}, which is special
variable form in \texttt{R}. The variable form \texttt{factor} is
similar to dummy variables. For example,
\texttt{factor(dt\$job,\ levels\ =\ c("admin",\ "custodial",\ "manage"))}
transforms the variable form \texttt{job} from the form
\texttt{character} into the form \texttt{factor}. Moreover, when we use
\texttt{job} as explanatory variables, \texttt{R} automatically makes
two dummy variables of \texttt{custodial} and \texttt{manage}.

\begin{Shaded}
\begin{Highlighting}[]
\KeywordTok{library}\NormalTok{(AER)}
\KeywordTok{data}\NormalTok{(BankWages)}
\NormalTok{dt \textless{}{-}}\StringTok{ }\NormalTok{BankWages}
\NormalTok{dt}\OperatorTok{$}\NormalTok{job \textless{}{-}}\StringTok{ }\KeywordTok{as.character}\NormalTok{(dt}\OperatorTok{$}\NormalTok{job)}
\NormalTok{dt}\OperatorTok{$}\NormalTok{job \textless{}{-}}\StringTok{ }\KeywordTok{factor}\NormalTok{(dt}\OperatorTok{$}\NormalTok{job, }\DataTypeTok{levels =} \KeywordTok{c}\NormalTok{(}\StringTok{"admin"}\NormalTok{, }\StringTok{"custodial"}\NormalTok{, }\StringTok{"manage"}\NormalTok{))}
\NormalTok{dt \textless{}{-}}\StringTok{ }\NormalTok{dt[,}\KeywordTok{c}\NormalTok{(}\StringTok{"job"}\NormalTok{, }\StringTok{"education"}\NormalTok{, }\StringTok{"gender"}\NormalTok{)]}

\KeywordTok{set.seed}\NormalTok{(}\DecValTok{120511}\NormalTok{)}
\NormalTok{train\_id \textless{}{-}}\StringTok{ }\KeywordTok{sample}\NormalTok{(}\DecValTok{1}\OperatorTok{:}\KeywordTok{nrow}\NormalTok{(dt), }\DataTypeTok{size =}\NormalTok{ (}\DecValTok{2}\OperatorTok{/}\DecValTok{3}\NormalTok{)}\OperatorTok{*}\KeywordTok{nrow}\NormalTok{(dt), }\DataTypeTok{replace =} \OtherTok{FALSE}\NormalTok{)}
\NormalTok{train\_dt \textless{}{-}}\StringTok{ }\NormalTok{dt[train\_id,]; test\_dt \textless{}{-}}\StringTok{ }\NormalTok{dt[}\OperatorTok{{-}}\NormalTok{train\_id,]}

\KeywordTok{head}\NormalTok{(train\_dt)}
\end{Highlighting}
\end{Shaded}

\begin{verbatim}
##        job education gender
## 75   admin        15 female
## 2    admin        16   male
## 374  admin        15   male
## 405  admin        12 female
## 67  manage        16   male
## 92   admin         8 female
\end{verbatim}

\hypertarget{model-1}{%
\subsection{Model}\label{model-1}}

The outcome variable \(y_i\) takes three values \(\{0, 1, 2\}\). Note
that the labelling of the choices is arbitrary. Then, the multinomial
logit model has the following response probabilities \begin{equation*}
  P_{ij} = \mathbb{P}(y_i = j | \mathbf{x}_i) =
  \begin{cases}
    \frac{\exp(\mathbf{x}_i \beta_j)}{1 + \sum_{k=1}^2 \exp(\mathbf{x}_i \beta_k)} &\text{if}\quad j = 1, 2  \\
    \frac{1}{1 + \sum_{k=1}^2 \exp(\mathbf{x}_i \beta_k)}  &\text{if}\quad j = 0
  \end{cases}.
\end{equation*} The log-likelihood function is \begin{equation*}
  M_n(\beta_1, \beta_2) = \sum_{i=1}^n \sum_{j=0}^3 d_{ij} \log (P_{ij}),
\end{equation*} where \(d_{ij}\) is a dummy variable taking 1 if
\(y_i = j\).

In \texttt{R}, some packages provide the multinomial logit model. In
this application, we use the \texttt{multinom} function in the library
\texttt{nnet}.

\begin{Shaded}
\begin{Highlighting}[]
\KeywordTok{library}\NormalTok{(nnet)}
\NormalTok{est\_mlogit \textless{}{-}}\StringTok{ }\KeywordTok{multinom}\NormalTok{(job }\OperatorTok{\textasciitilde{}}\StringTok{ }\NormalTok{education }\OperatorTok{+}\StringTok{ }\NormalTok{gender, }\DataTypeTok{data =}\NormalTok{ train\_dt)}
\end{Highlighting}
\end{Shaded}

\hypertarget{interpretations-and-model-fitness}{%
\subsection{Interpretations and Model
Fitness}\label{interpretations-and-model-fitness}}

Table \ref{job} summarizes the result of multinomial logit model. The
coefficient represents the change of \(\log(P_{ij}/P_{i0})\) in
corresponding covariate beucase the response probabilities yields
\begin{equation*}
    \frac{P_{ij}}{P_{i0}} = \exp(\mathbf{x}_i \beta_j)  \Leftrightarrow
    \log \left( \frac{P_{ij}}{P_{i0}} \right) = \mathbf{x}_i \beta_j.
\end{equation*} For example, eduction decreases the log-odds between
\texttt{custodial} and \texttt{admin} by -0.562. This implies that those
who received higher education are more likely to obtain the position
\texttt{admin}. Highly-educated workers are also more likely to obtain
the position \texttt{manage}. Moreover, a female dummy decrease the
log-odds between \texttt{manage} and \texttt{admin} by -0.748, which
implies that females are less likely to obtain higher position
\texttt{manage}. From this result, we conclude that the U.S. bank
disencouraged females to assign higher job position.

To evalue model fitness and prediction, we use two indices: the
\emph{pseudo R-squared} and \emph{percent correctly predicted}. The
\emph{preudo R-sqaured} is calculated by \(1 - L_1/L_0\) where \(L_1\)
is the value of log-likelihood for estimated model and \(L_0\) is the
value of log-likelihood in the model with only an intercept. \texttt{R}
snippet for calculation of pseudo R-sqaured is as follows: Note that
\texttt{nnet:::logLik.multinom()} returns the value of log-likelihood.

\begin{Shaded}
\begin{Highlighting}[]
\NormalTok{loglik1 \textless{}{-}}\StringTok{ }\KeywordTok{as.numeric}\NormalTok{(nnet}\OperatorTok{:::}\KeywordTok{logLik.multinom}\NormalTok{(est\_mlogit))}
\NormalTok{est\_mlogit0 \textless{}{-}}\StringTok{ }\KeywordTok{multinom}\NormalTok{(job }\OperatorTok{\textasciitilde{}}\StringTok{ }\DecValTok{1}\NormalTok{, }\DataTypeTok{data =}\NormalTok{ train\_dt)}
\NormalTok{loglik0 \textless{}{-}}\StringTok{ }\KeywordTok{as.numeric}\NormalTok{(nnet}\OperatorTok{:::}\KeywordTok{logLik.multinom}\NormalTok{(est\_mlogit0))}
\NormalTok{pr2 \textless{}{-}}\StringTok{ }\KeywordTok{round}\NormalTok{(}\DecValTok{1} \OperatorTok{{-}}\StringTok{ }\NormalTok{loglik1}\OperatorTok{/}\NormalTok{loglik0, }\DecValTok{3}\NormalTok{)}
\end{Highlighting}
\end{Shaded}

The second index is the \emph{precent correctly predicted}. The
predicted outcome is the outcome with the highest estimated probability.
Using the training data (in-sample) and the test data (out-of-sample),
we calculate this index. \texttt{R} snippet for calculation of this
index is as follows.

\begin{Shaded}
\begin{Highlighting}[]
\CommentTok{\# in{-}sample prediction}
\NormalTok{inpred \textless{}{-}}\StringTok{ }\KeywordTok{predict}\NormalTok{(est\_mlogit, }\DataTypeTok{newdata =}\NormalTok{ train\_dt, }\StringTok{"probs"}\NormalTok{)}
\NormalTok{inpred \textless{}{-}}\StringTok{ }\KeywordTok{colnames}\NormalTok{(inpred)[}\KeywordTok{apply}\NormalTok{(inpred, }\DecValTok{1}\NormalTok{, which.max)]}
\NormalTok{inpcp \textless{}{-}}\StringTok{ }\KeywordTok{round}\NormalTok{(}\KeywordTok{sum}\NormalTok{(inpred }\OperatorTok{==}\StringTok{ }\NormalTok{train\_dt}\OperatorTok{$}\NormalTok{job)}\OperatorTok{/}\KeywordTok{length}\NormalTok{(inpred), }\DecValTok{3}\NormalTok{)}
\CommentTok{\# out{-}of{-}sample prediction}
\NormalTok{outpred \textless{}{-}}\StringTok{ }\KeywordTok{predict}\NormalTok{(est\_mlogit, }\DataTypeTok{newdata =}\NormalTok{ test\_dt, }\StringTok{"probs"}\NormalTok{)}
\NormalTok{outpred \textless{}{-}}\StringTok{ }\KeywordTok{colnames}\NormalTok{(outpred)[}\KeywordTok{apply}\NormalTok{(outpred, }\DecValTok{1}\NormalTok{, which.max)]}
\NormalTok{outpcp \textless{}{-}}\StringTok{ }\KeywordTok{round}\NormalTok{(}\KeywordTok{sum}\NormalTok{(outpred }\OperatorTok{==}\StringTok{ }\NormalTok{test\_dt}\OperatorTok{$}\NormalTok{job)}\OperatorTok{/}\KeywordTok{length}\NormalTok{(outpred), }\DecValTok{3}\NormalTok{)}
\end{Highlighting}
\end{Shaded}

As a result, our model is good in terms of fitness and prediction
because the percent correctly predicted is high (83.9\% of in-sample
data and 88.0\% of out-of-sample data), and the pseudo R-sqaured is
0.523.

\begin{Shaded}
\begin{Highlighting}[]
\KeywordTok{stargazer}\NormalTok{(}
\NormalTok{  est\_mlogit,}
  \DataTypeTok{covariate.labels =} \KeywordTok{c}\NormalTok{(}\StringTok{"Education"}\NormalTok{, }\StringTok{"Female = 1"}\NormalTok{),}
  \DataTypeTok{report =} \StringTok{"vcs"}\NormalTok{, }\DataTypeTok{omit.stat =} \KeywordTok{c}\NormalTok{(}\StringTok{"aic"}\NormalTok{),}
  \DataTypeTok{add.lines =} \KeywordTok{list}\NormalTok{(}
    \KeywordTok{c}\NormalTok{(}\StringTok{"Observations"}\NormalTok{, }\KeywordTok{length}\NormalTok{(inpred), }\StringTok{""}\NormalTok{),}
    \KeywordTok{c}\NormalTok{(}\StringTok{"Percent correctly predicted (in{-}sample)"}\NormalTok{, inpcp, }\StringTok{""}\NormalTok{),}
    \KeywordTok{c}\NormalTok{(}\StringTok{"Percent correctly predicted (out{-}of{-}sample)"}\NormalTok{, outpcp, }\StringTok{""}\NormalTok{),}
    \KeywordTok{c}\NormalTok{(}\StringTok{"Log{-}likelihood"}\NormalTok{, }\KeywordTok{round}\NormalTok{(loglik1, }\DecValTok{3}\NormalTok{), }\StringTok{""}\NormalTok{),}
    \KeywordTok{c}\NormalTok{(}\StringTok{"Pseudo R{-}sq"}\NormalTok{, pr2, }\StringTok{""}\NormalTok{)}
\NormalTok{  ),}
  \DataTypeTok{omit.table.layout =} \StringTok{"n"}\NormalTok{, }\DataTypeTok{table.placement =} \StringTok{"t"}\NormalTok{,}
  \DataTypeTok{title =} \StringTok{"Multinomial Logit Model of Job Position"}\NormalTok{,}
  \DataTypeTok{label =} \StringTok{"job"}\NormalTok{,}
  \DataTypeTok{type =} \StringTok{"latex"}\NormalTok{, }\DataTypeTok{header =} \OtherTok{FALSE}  
\NormalTok{)}
\end{Highlighting}
\end{Shaded}

\begin{table}[t] \centering 
  \caption{Multinomial Logit Model of Job Position} 
  \label{job} 
\begin{tabular}{@{\extracolsep{5pt}}lcc} 
\\[-1.8ex]\hline 
\hline \\[-1.8ex] 
 & \multicolumn{2}{c}{\textit{Dependent variable:}} \\ 
\cline{2-3} 
\\[-1.8ex] & custodial & manage \\ 
\\[-1.8ex] & (1) & (2)\\ 
\hline \\[-1.8ex] 
 Education & $-$0.547 & 1.322 \\ 
  & (0.116) & (0.229) \\ 
  & & \\ 
 Female = 1 & $-$10.507 & $-$0.891 \\ 
  & (31.352) & (0.524) \\ 
  & & \\ 
 Constant & 4.634 & $-$21.448 \\ 
  & (1.269) & (3.605) \\ 
  & & \\ 
\hline \\[-1.8ex] 
Observations & 316 &  \\ 
Percent correctly predicted (in-sample) & 0.839 &  \\ 
Percent correctly predicted (out-of-sample) & 0.88 &  \\ 
Log-likelihood & -102.964 &  \\ 
Pseudo R-sq & 0.523 &  \\ 
\hline 
\hline \\[-1.8ex] 
\end{tabular} 
\end{table}

\end{document}
